\maketitle

\begin{remark}
	This text is based on the lecture notes on Analytic Geometry~\cite{Analytic},
	by Peter Scholze.
  The final section is copy-pasted from those lecture notes almost verbatim.
  This text is meant as a blueprint for the Liquid Tensor Experiment.
\end{remark}

\section{Breen--Deligne data}

The goal of this section is to a give a precise statement of the Breen--Deligne resolution.
We first give the statement, and provide details later.

\begin{theoremx}[Breen--Deligne]
  \label{BD_orig}
  For an abelian group $A$, there is a resolution, functorial in~$A$, of the form
  \[
    \ldots \to \bigoplus_{i=1}^{n_i} \mathbb Z[A^{r_{ij}}] \to \ldots
    \to \mathbb Z[A^3] \oplus \mathbb Z[A^2] \to \mathbb Z[A^2] \to \mathbb Z[A] \to A \to 0.
  \]
\end{theoremx}

I (Johan Commelin) have not figure out the details.
But it seems to be possible to avoid the $\bigoplus_{i=1}^{n_i}$,
so we will aim for something like the following statement.

\begin{theoremx}
  \label{BD_reso}
  For an abelian group $A$, there is a resolution, functorial in~$A$, of the form
  \[
    \ldots \to \mathbb Z[A^{n_{i}}] \to \ldots \to \mathbb Z[A^2] \to \mathbb Z[A] \to A \to 0.
  \]
\end{theoremx}

What does a homomorphism $f \colon \mathbb Z[A^m] \to \mathbb Z[A^n]$
that is functorial in~$A$ look like?

Well, such a ``universal map'' is specified
by what it does to the generators $(a_1, a_2, a_3, \dots, a_m)$.
It can send such an element to an arbitrary element of $\mathbb Z[A^n]$,
but it has to be ``universal''.

In the end, this means that $f$ will be a $\mathbb Z$-linear combination of
``basic universal maps'', where a ``basic universal map'' is one that
sends $(a_1, a_2, \dots, a_m)$ to $(b_1, \dots, b_n)$,
where $b_i$ is a $\mathbb Z$-linear combination $c_{i,1} \cdot a_1 + \dots + c_{i,m} \cdot a_m$.
So a ``basic universal map'' is specified by the $n \times m$-matrix $c$.

\begin{definition}
  \label{basic_universal_map}
  \lean{breen_deligne.basic_universal_map}
  \leanok
  A \emph{basic universal map} from exponent $m$ to $n$,
  is an $n \times m$-matrix with coefficients in~$\mathbb Z$.
\end{definition}

\begin{definition}
  \label{universal_map}
  \lean{breen_deligne.universal_map}
  \leanok
  \uses{basic_universal_map}
  A \emph{universal map} from exponent $m$ to $n$,
  is a formal $\mathbb Z$-linear combination of basic universal maps from exponent $m$ to $n$.
\end{definition}

We point out that basic universal maps can be composed by matrix multiplication,
and this formally induces a composition of universal maps.

\begin{definition}
  \label{sigma_add}
  \lean{breen_deligne.σ_add}
  \leanok
  \uses{universal_map}
  The addition on $A^n$ induces a universal map
  $\sigma_\alpha \colon \mathbb Z[A^{2n}] \to \mathbb Z[A^n]$,
  namely the formal generator $(I_n I_n)$, where $I_n$ denotes the $n \times n$ identity matrix.
\end{definition}

\begin{definition}
  \label{sigma_proj}
  \lean{breen_deligne.σ_proj}
  \leanok
  \uses{universal_map}
  The formal sum of the two projections $A^{2n} \to A^n$
  induces a universal map $\sigma_\pi \colon \mathbb Z[A^{2n}] \to \mathbb Z[A^n]$,
  namely the formal sum $(I_n 0) + (0 I_n)$,
  where $I_n$ denotes the $n \times n$ identity matrix, and $0$ the $n \times n$ zero matrix.
\end{definition}

\begin{definition}
  \label{BD_double}
  \lean{breen_deligne.universal_map.double}
  \leanok
  \uses{universal_map}
  Let $f$ be a universal map from exponent~$m$ to~$n$.
  Then $f \oplus f$ denotes the universal map from exponent~$2m$ to~$2n$,
  that applies $f$ componentwise.
  In other words, if $f$ is a generator (i.e.\ a basic universal map) 
  then $f \oplus f$ is
  \[
    \begin{pmatrix}
    f & 0 \\
    0 & f
    \end{pmatrix}.
  \]
\end{definition}

\begin{definition}
  \label{BD_data}
  \lean{breen_deligne.data}
  \lean{breen_deligne.is_complex}
  \leanok
  \uses{sigma_add, sigma_proj}
  A tuple $(n, f)$ of \emph{Breen--Deligne data}
  consists of a sequence of exponents $n_0, n_1, n_2, \dots \in \mathbb N$,
  and universal maps $f_i$ from exponent $n_{i+1}$ to $n_i$.

  Such a tuples is a \emph{complex} if for all $i$ we have $f_i \circ f_{i+1} = 0$.
\end{definition}

\begin{definition}
  \label{BD_data_double}
  % There is no corresponding Lean code. Maybe we should add it.
  \uses{BD_data, BD_double}
  If $(n, f)$ is a tuple of Breen--Deligne data,
  then $(n, f) \oplus (n, f)$ is the tuple
  consisting of exponents $2n_i$ and universal maps $f_i \oplus f_i$.
\end{definition}

The two universal map $\sigma_\alpha$ and~$\sigma_\pi$ explained in the examples above,
can be viewed as universal maps of complexes: $(n,f) \oplus (n,f) \to (n,f)$.

\begin{definition}
  \label{BD_homotopy}
  \lean{breen_deligne.homotopy}
  \leanok
  \uses{BD_data, BD_double}
  A \emph{homotopy} for a tuple $(n, f)$ of Breen--Deligne data
  is a homotopy between the maps of complexes
  \[
    \sigma_\alpha, \sigma_\pi \colon (n,f) \oplus (n,f) \to (n,f)
  \]
  In other words, it consists of universal maps $h_i$ from exponent $2n_i$ to $n_{i+1}$,
  such that
  \[
    f_{i+1} \circ h_{i+1} + h_i \circ (f_i \oplus f_i) = \sigma_\alpha - \sigma_\pi.
  \]
\end{definition}

\begin{definition}
  \label{BD_package}
  \lean{breen_deligne.package}
  \leanok
  \uses{BD_data, BD_homotopy}
  A \emph{Breen--Deligne package}
  is a triple $(n, f, h)$,
  such that $(n, f)$ is Breen--Deligne data that is a complex,
  and $h$ is a homotopy for $(n,f)$.
\end{definition}

\begin{definition}
  \label{BD_eg}
  \lean{breen_deligne.eg}
  \leanok
  \uses{BD_package}
  We will now construct an example of a Breen--Deligne package.
  In some sense, it is the ``easiest'' solution to the conditions posed above.
  The exponents will be $n_i = 2^i$, and the homotopies $h_i$ will be the identity.
  Under these constraints, we inductively construct the universal maps $f_i$:
  \[
    f_0 = \sigma_\alpha - \sigma_\pi,
    \quad
    f_{i+1} = (\sigma_\alpha - \sigma_\pi) - (f_i \oplus f_i).
  \]
  We leave it as exercise for the reader, to verify that
  with these definitions $(n, f, h)$ forms a Breen--Deligne package.
\end{definition}

\section{Spaces of convergent power series}

\begin{definition}
  \label{pseudo_normed_group}
  \lean{pseudo_normed_group}
  \leanok
  A \emph{pseudo-normed group} is an abelian group $A$,
  together with an increasing filtration $A_c \subset A$ of subsets $A_c$ indexed by $\mathbb R_{\ge 0}$,
  such that each $A_c$ contains $0$, is closed under negation,
  and $A_{c_1} + A_{c_2} \subset A_{c_1 + c_2}$.
\end{definition}

\begin{definition}
  \label{Mbar}
  \lean{Mbar}
  \leanok
  Let $r > 0$ be a real number, and let $S$ be a finite set.
  Denote by $\overline{\mathcal M}_r(S)$ the set
  \[
    \left\{ \left( \sum_{n \ge 1} a_{n,s} T^n \right)_{s \in S} \middle\vert \sum_{n \ge 1, s \in S} |a_{n,s}| r^n < \infty \right\}
  \]
\end{definition}

\begin{definition}
  \label{Mbar_le}
  \lean{Mbar_le}
  \leanok
  Let $r > 0$ and $c \ge 0$ be real numbers, and let $S$ be a finite set.
  Denote by $\overline{\mathcal M}_r(S)_{\le c}$ the set
  \[
    \left\{ \left( \sum_{n \ge 1} a_{n,s} T^n \right)_{s \in S} \middle\vert \sum_{n \ge 1, s \in S} |a_{n,s}| r^n \le c \right\}
  \]
\end{definition}

\begin{lemma}
  \label{Mbar_png}
  \lean{Mbar.pseudo_normed_group}
  \leanok
  \uses{pseudo_normed_group,Mbar,Mbar_le}
  The space $\overline{\mathcal M}_r(S)$ is a pseudo-normed group,
  with filtration $\overline{\mathcal M}_r(S)_{\le c}$.
\end{lemma}

\begin{proof}
  \leanok
  Obvious.
\end{proof}

\begin{lemma}
  \label{Mbar_le_profinite}
  \lean{Mbar_le.eqv}
  \leanok
  \uses{Mbar_le}
  Let $r > 0$ and $c \ge 0$ be real numbers, and let $S$ be a finite set.
  The space $\overline{\mathcal M}_r(S)_{\le c}$ is the profinite limit of the finite sets
  \[
    \overline{\mathcal M}_r(S)_{\le c, \le N} =
    \left\{ \left( \sum_{n \ge 1} a_{n,s} T^n \right)_{s \in S} \middle\vert
    \sum_{1 \le n \le N, s \in S} |a_{n,s}| r^n \le c \right\}
  \]
  This endows $\overline{\mathcal M}_r(S)_{\le c}$ with the profinite topology.
\end{lemma}

\begin{proof}
  \leanok
  Omitted.
\end{proof}

For the remainder of this section,
let $r > 0, c \ge 0$ be real numbers,
and let $S$ be a finite set.

\begin{definition}
  \label{Mbar_Tinv}
  \lean{Mbar.Tinv}
  \leanok
  \uses{Mbar}
  There is a natural action of $T^{-1}$ on $\overline{\mathcal M}_r(S)$, via
  \[
    T^{-1} \cdot 
    \left( \sum_{n \ge 1} a_{n,s} T^n \right)_{s \in S} =
    \left( \sum_{n \ge 1} a_{n+1,s} T^n \right)_{s \in S}.
  \]
\end{definition}

\begin{definition}
  \label{Mbar_le_Tinv}
  \lean{Mbar_le.Tinv}
  \leanok
  \uses{Mbar_le, Mbar_Tinv}
  The natural action of $T^{-1}$ on $\overline{\mathcal M}_r(S)$ restricts to maps
  \[
    T^{-1} \cdot \_ \colon
    \overline{\mathcal M}_r(S)_{\le c}
    \overline{\mathcal M}_r(S)_{\le c'}
  \]
  for all $c' \ge c/r$.
\end{definition}

\begin{lemma}
  \label{Mbar_le_Tinv_ctu}
  \lean{Mbar_le.continuous_Tinv}
  \leanok
  \uses{Mbar_le_profinite, Mbar_Tinv}
  For all $c' \ge c/r$, the natural maps
  \[
    T^{-1} \cdot \_ \colon
    \overline{\mathcal M}_r(S)_{\le c}
    \overline{\mathcal M}_r(S)_{\le c'}
  \]
  are continuous
\end{lemma}

\begin{proof}
  \leanok
  Omitted.
\end{proof}

\begin{lemma}
  \label{basic_eval_Mbar}
  \lean{breen_deligne.basic_universal_map.eval_png_mem_filtration}
  \leanok
  \uses{Mbar, basic_universal_map}
  Let $f$ be a basic universal map from exponent~$m$ to~$n$.
  We get an induced homomorphism of pseudo normed groups
  \[
    \overline{\mathcal M}_r(S)^m \to \overline{\mathcal M}_r(S)^n
  \]
  bounded by the maximum (over all $i$) of $\sum_j |f_{ij}|$,
  where the $f_{ij}$ are the coefficients of the $n \times m$-matrix representing~$f$.
\end{lemma}

\begin{proof}
  \leanok
  Omitted.
\end{proof}

\begin{definition}
  \label{basic_suitable}
  \lean{breen_deligne.basic_universal_map.suitable}
  \leanok
  \uses{basic_universal_map}
  Let $f$ be a basic universal map from exponent~$m$ to~$n$.
  Let $c_1, c_2 \in \mathbb R_{\ge 0}$.
  We say that $c_1, c_2$ are $f$-suitable, if for all $i$
  \[
    \sum_j |f_{ij}| * c_1 \le c_2.
  \]
  (Note! The order of $c_1$ and $c_2$ matters!)
\end{definition}

\begin{lemma}
  \label{basic_eval_Mbar_le_ctu}
  \lean{breen_deligne.basic_universal_map.eval_Mbar_le_continuous}
  \leanok
  \uses{Mbar_le_profinite, Mbar_png, basic_suitable, basic_eval_Mbar}
  Let $f$ be a basic universal map from exponent~$m$ to~$n$,
  and let $c_1, c_2 \in \mathbb R_{\ge 0}$ be $f$-suitable.
  We get an induced map
  \[
    \overline{\mathcal M}_r(S)_{\le c_1}^m \to \overline{\mathcal M}_r(S)_{\le c_2}^n
  \]
  that is continuous.
\end{lemma}

\begin{proof}
  \leanok
  Omitted.
\end{proof}

\section{Completions of locally constant functions}

\begin{definition}
  \label{Vhat}
  \lean{NormedGroup.LCC}
  \leanok
  Let $V$ be a normed abelian group, and $X$ a compact topological space.
  We denote by $\hat V(X)$ the completion
  of the normed abelian group of locally constant functions $X \to V$
  with respect to the sup norm.

  This construction is functorial in bounded group homomorphisms $V \to V'$
  and contravariantly functorial in continuous maps $X \to X'$.
\end{definition}

We continue to use the notation of before:
let $r, c \in \mathbb R_{\ge 0}$, and let $S$ be a finite set.

\begin{definition}
  \label{Vhat_basic_eval_Mbar}
  \lean{breen_deligne.basic_universal_map.eval_Mbar_pow}
  \leanok
  \uses{basic_eval_Mbar_le_ctu, Vhat}
  Let $f$ be a basic universal map from exponent~$m$ to~$n$,
  and let $c_1, c_2 \in \mathbb R_{\ge 0}$ be $f$-suitable.
  We get an induced map
  \[
    \hat V(f) \colon
    \hat V(\overline{\mathcal M}_r(S)_{\le c_1}^m) \to
    \hat V(\overline{\mathcal M}_r(S)_{\le c_2}^n).
  \]
\end{definition}

\begin{definition}
  \label{universal_suitable}
  \lean{breen_deligne.universal_map.suitable}
  \leanok
  \uses{universal_map, basic_suitable}
  Let $f$ be a universal map from exponent~$m$ to~$n$.
  Let $c_1, c_2 \in \mathbb R_{\ge 0}$.
  We say that $c_1, c_2$ are $f$-suitable, if for all basic universal maps $g$
  that occur in the formal sum $f$,
  the nonnegative reals $c_1$ and $c_2$ are $g$-suitable.
  (Note! The order of $c_1$ and $c_2$ matters!)
\end{definition}

\begin{definition}
  \label{Vhat_eval_Mbar}
  \lean{breen_deligne.universal_map.eval_Mbar_pow}
  \leanok
  \uses{universal_suitable, Vhat_basic_eval_Mbar}
  Let $f$ be a universal map from exponent~$m$ to~$n$,
  and let $c_1, c_2 \in \mathbb R_{\ge 0}$ be $f$-suitable.
  We get an induced map
  \[
    \colon
    \hat V(\overline{\mathcal M}_r(S)_{\le c_1}^m) \to
    \hat V(\overline{\mathcal M}_r(S)_{\le c_2}^n).
  \]
  that is the sum $\sum n_g V(g)$,
  if $f$ is the formal sum $\sum n_g g$
  of basic universal maps.
\end{definition}

\begin{definition}
  \label{normed_with_aut}
  \lean{normed_with_aut}
  \leanok
  Let $r \in \mathbb R_{\ge 0}$.
  An $r$-normed $\mathbb Z[T^{\pm 1}]$-module
  is a normed abelian group $V$
  endowed with an automorphism $T \colon V \to V$ such that
  for all $v \in V$ we have $\|T(v)\| = r\|v\|$.
\end{definition}

\begin{lemma}
  \label{Vhat_T_aut}
  \lean{NormedGroup.normed_with_aut_LCC}
  \leanok
  \uses{Vhat, normed_with_aut}
  Let $r \in \mathbb R_{\ge 0}$,
  and let $V$ be an $r$-normed $\mathbb Z[T^{\pm 1}]$-module.
  Let $X$ be a compact space.
  Then $\hat V(X)$ is naturally an $r$-normed $\mathbb Z[T^{\pm 1}]$-module,
  with the action of $T$ given by post-composition.
\end{lemma}

\begin{proof}
  \leanok
  Omitted.
\end{proof}

\begin{definition}
  \label{Vhat_Mbar_Tinv}
  \lean{LCC_Mbar_pow_Tinv}
  \leanok
  \uses{Mbar_le_Tinv_ctu, Vhat_T_aut}
  Let $r, r' c \in \mathbb R_{>0}$,
  and let $V$ be an $r$-normed $\mathbb Z[T^{\pm 1}]$-module.
  Assume $r' \le 1$.
  Let $S$ be a finite set.

  There are two natural actions of $T^{-1}$ on
  $\hat V(\overline{\mathcal M}_{r'}(S)_{\le c})$.
  The first comes from Definition~\ref{Mbar_le_Tinv},
  which gives a continuous map
  \[
    \overline{\mathcal M}_{r'}(S)_{\le c*r'} \to
    \overline{\mathcal M}_{r'}(S)_{\le c}
  \]
  and thus a map
  \[
    (T^{-1})^* \colon
    \hat V(\overline{\mathcal M}_{r'}(S)_{\le c}) \to
    \hat V(\overline{\mathcal M}_{r'}(S)_{\le c*r'}).
  \]
  The other comes from Lemma~\ref{Vhat_normed_with_aut},
  using the $r$-normed $\mathbb Z[T^{\pm 1}]$-module $V$.
  We get a map
  \[
    [T^{-1}] \colon
    \hat V(\overline{\mathcal M}_{r'}(S)_{\le c}) \to
    \hat V(\overline{\mathcal M}_{r'}(S)_{\le c}),
  \]
  that we can compose with the map
  \[
    \hat V(\overline{\mathcal M}_{r'}(S)_{\le c}) \to
    \hat V(\overline{\mathcal M}_{r'}(S)_{\le c*r'}),
  \]
  obtained from the natural inclusion
  \[
    \overline{\mathcal M}_{r'}(S)_{\le c*r'} \to
    \overline{\mathcal M}_{r'}(S)_{\le c}.
  \]
  We thus end up with two maps
  \[
    (T^{-1})^*, [T^{-1}] \colon
    \hat V(\overline{\mathcal M}_{r'}(S)_{\le c}) \to
    \hat V(\overline{\mathcal M}_{r'}(S)_{\le c*r'}).
  \]
  and we define
  $\hat V(\overline{\mathcal M}_{r'}(S)_{\le c})^{T^{-1}}$
  to be the equalizer of $(T^{-1})^*$, and $[T^{-1}]$.
  In other words, the kernel of $(T^{-1})^* - [T^{-1}]$.
\end{definition}

\begin{definition}
  \label{Vhat_eval_Mbar_Tinv}
  \lean{breen_deligne.universal_map.eval_Mbar_pow_Tinv}
  \leanok
  \uses{Vhat_Mbar_Tinv, Vhat_eval_Mbar}
  Let $r, r' \in \mathbb R_{>0}$,
  and let $V$ be an $r$-normed $\mathbb Z[T^{\pm 1}]$-module.
  Assume $r' \le 1$.
  Let $S$ be a finite set.
  Let $f$ be a universal map from exponent~$m$ to~$n$,
  and let $c_1, c_2 \in \mathbb R_{\ge 0}$ be $f$-suitable.

  The natural map from Definition~\ref{Vhat_eval_Mbar}
  restricts to a map
  \[
    \hat V(\overline{\mathcal M}_{r'}(S)_{\le c_2}^n)^{T^{-1}} \to
    \hat V(\overline{\mathcal M}_{r'}(S)_{\le c_1}^m)^{T^{-1}}
  \]
\end{definition}

\section{Lecture IX: End of proof}

\begin{definition}
  \label{system_of_complexes}
  \lean{system_of_complexes}
  \leanok
  A system of complexes of complete normed abelian groups
  is for each sufficiently large $c$ (i.e.~all $c\geq c_0$ for some $c_0>0$),
  a complex
  \[
  C_c^\bullet: C_c^0\to C_c^1\to\ldots
  \]
  together with maps of complexes
  $\mathrm{res}_{c',c}^i: C_{c'}^\bullet\to C_c^\bullet$,
  for $c'>c$,
  satisfying the obvious associativity condition.
\end{definition}

\begin{definition}
  \label{admissible}
  \lean{system_of_complexes.admissible}
  \leanok
  \uses{system_of_complexes}
  A system of complexes is \emph{admissible}
  if all differentials and maps $\mathrm{res}_{c',c}^i$ are norm-nonincreasing.
\end{definition}

\begin{definition}
  \label{is_bdd_exact_for_bdd_degree_above_idx}
  \lean{system_of_complexes.is_bdd_exact_for_bdd_degree_above_idx}
  \leanok
  \uses{system_of_complexes}
  Let $(C_c^\bullet)_c$ be a system of complexes.
  For integers $m\geq 0$ and constants $k>0$, $c_0'>0$,
  the datum $(C_c^\bullet)_c$ is
  $\leq k$-exact in degrees $\leq m$ and for $c\geq c_0'$ if the following condition is satisfied.
  For all $c\geq c_0'$ and all $x\in C_{kc}^i$ with $i\leq m$
  there is some $y\in C_c^{i-1}$ (which is defined to be $0$ when $i=0$) such that
  \[
    \|\mathrm{res}_{kc,c}^i(x)-d_c^{i-1}(y)\|_{C_c^i}\leq k\|d_{kc}^i(x)\|_{C_{kc}^{i+1}}.
  \]
\end{definition}

We apply this to
\[
  C_c^\bullet \colon
  \widehat{V}(\overline{\mathcal M}_{r'}(S)_{\leq c})^{T^{-1}} \to
  \widehat{V}(\overline{\mathcal M}_{r'}(S)_{\leq c_1c}^2)^{T^{-1}} \to \ldots
\]
given by mapping \eqref{eq:doublecomplex} into $\hat{V}$ and using Lemma~\ref{lem:Tinv}.
Let us make this precise.

\begin{definition}
  \label{BD_suitable}
  \lean{breen_deligne.package.suitable}
  \leanok
  \uses{BD_package, universal_suitable}
  Let $(n, f, h)$ be a Breen--Deligne package,
  and let $c = (c_0, c_1, \dots)$ be a sequence of nonnegative real numbers.
  We say that $c$ is $(n,f,h)$-suitable,
  if for all $i$, the constants $c_{i+1}, c_i$ are $f_i$-suitable.

  (Note! The order $c_{i+1}, c_i$ is contravariant
  compared to Definition~\ref{universal_suitable}.
  This is because of the contravariance of $\hat V(\_)$;
  see Definition~\ref{Vhat_eval_Mbar_Tinv}.)
\end{definition}

\begin{definition}
  \label{Mbar_system}
  \lean{Mbar_system}
  \leanok
  \uses{BD_suitable, Vhat_eval_Mbar_Tinv, system_of_complexes}
  Let $r, r' \in \mathbb R_{>0}$,
  and let $V$ be an $r$-normed $\mathbb Z[T^{\pm 1}]$-module.
  Assume $r' \le 1$.
  Let $S$ be a finite set.
  Let $(n,f,h)$ be a Breen--Deligne package,
  and let $c' = (c'_0, c'_1, c'_2, \dots)$ be a sequence of constants in $\mathbb R_{\ge 0}$
  that is $(n,f,h)$-suitable.

  For every $c \in \mathbb R_{\ge 0}$,
  the maps from Definition~\ref{Vhat_eval_Mbar_Tinv}
  induced by the universal maps $f_i$ from the Breen--Deligne package $(n,f,h)$
  assemble into a complex of normed abelian groups
  \[
    \dots \to
    \hat V(\overline{\mathcal M}_{r'}(S)_{\le c_i}^{n_i})^{T^{-1}}
    \to
    \hat V(\overline{\mathcal M}_{r'}(S)_{\le c_{i+1}}^{n_{i+1}})^{T^{-1}}
    \to \dots \to 0.
  \]
  Together, these complexes fit into a system of complexes
  with the natural restriction maps.
\end{definition}

Now we state the following result, which is our main goal.

\textbf{N.b.:}
It differs from Theorem~9.4 of \cite{Analytic} only in one aspect:
we assume that the sets $S$ are finite, rather than profinite.

\begin{theorem}
  \label{first_target}
  \lean{first_target}
  \leanok
  \uses{Mbar_system}
  Let $(n,f,h)$ be a Breen--Deligne package,
  and let $c' = (c'_0, c'_1, c'_2, \dots)$ be a sequence of constants in $\mathbb R_{\ge 0}$
	Fix radii $1>r'>r>0$.
  For any $m$ there is some $k$ and $c_0$ such that for all finite sets $S$ and all $r$-normed $\mathbb Z[T^{\pm 1}]$-modules $V$,
  the system of complexes
  \[
    C_c^\bullet \colon
    \widehat{V}(\overline{\mathcal M}_{r'}(S)_{\leq c})^{T^{-1}} \to
    \widehat{V}(\overline{\mathcal M}_{r'}(S)_{\leq c'_1c}^2)^{T^{-1}}
    \to \ldots
  \]
  is $\leq k$-exact in degrees $\leq m$ for $c\geq c_0$.
\end{theorem}

\begin{remark}
  \label{boundary-text}
  \textbf{Note: the text below is copied almost verbatim from \cite{Analytic}.
  Small parts have been formalized.
  We expect that the text will be rewritten and expanded as the formalization project progresses.}
\end{remark}

% Let us first check that this implies Theorem~\ref{explicit1}.

% \begin{proof}[Theorem~\ref{first_target} implies Theorem~\ref{explicit1}] By the preceding discussion, one can compute
% \[
% R\Hom_{\mathbb Z[T^{-1}]}(\overline{\mathcal M}_{r'}(S),\widehat{V})
% \]
% as the derived inverse limit of $C_c^\bullet$ over all $c>0$; equivalently, all $c\geq c_0$. Theorem~\ref{first_target} implies that for any $m\geq 0$ the pro-system of cohomology groups $H^m(C_c^\bullet)$ is pro-zero (as $H^m(C_{kc}^\bullet)\to H^m(C_c^\bullet)$ is zero). Thus, the derived inverse limit vanishes, as desired.
% \end{proof}

% We remark that Theorem~\ref{first_target} reduces formally to the case that $S$ is finite; we make this reduction.

We will prove Theorem~\ref{first_target} by induction on $m$.
Unfortunately, the induction requires us to prove a stronger statement.
Before stating it, we need another definition.

\begin{definition}
  \label{polyhedral_lattice}
  A \emph{polyhedral lattice} is a finite free abelian group~$\Lambda$
  equipped with a norm $||\cdot||_\Lambda \colon \Lambda\otimes \mathbb R\to \mathbb R$
  (so $\Lambda\otimes \mathbb R$ is a Banach space)
  that is given by the supremum of finitely many linear functions on $\Lambda$ with rational coefficients;
  equivalently, the ``unit ball''
  $\{\lambda\in \Lambda\otimes \mathbb R\mid ||\lambda||_\Lambda\leq 1\}$ is a rational polyhedron.
\end{definition}

Endow $\Hom(\Lambda,\overline{\mathcal M}_{r'}(S))$ with the subspaces
\[
\Hom(\Lambda,\overline{\mathcal M}_{r'}(S))_{\leq c} = \{f: \Lambda\to \overline{\mathcal M}_{r'}(S)\mid \forall x\in \Lambda, f(x)\in \overline{\mathcal M}_{r'}(S)_{\leq c||x||}\}.
\]
As $\Lambda$ is polyhedral, it is enough to check the given condition for finitely many $x$.

We can then define double complexes like \eqref{eq:doublecomplex}.
Lemma~\ref{lem:Tinv} stays true with the same constants.
Now we claim the following generalization of Theorem~\ref{first_target}.

\begin{theorem}
  \label{explicit}
  \uses{polyhedral_lattice}
  Fix radii $1>r'>r>0$. For any $m$ there is some $k$
  such that for all polyhedral lattices $\Lambda$
  there is a constant $c_0(\Lambda)>0$
  such that for all finite sets $S$
  and all $r$-normed $\mathbb Z[T^{\pm 1}]$-modules $V$,
  the system of complexes
  \[
  C_{\Lambda,c}^\bullet \colon
  \widehat{V}(\Hom(\Lambda,\overline{\mathcal M}_{r'}(S))_{\leq c})^{T^{-1}} \to
  \widehat{V}(\Hom(\Lambda,\overline{\mathcal M}_{r'}(S))_{\leq c_1c}^2)^{T^{-1}} \to \ldots
  \]
  is $\leq k$-exact in degrees $\leq m$ for $c\geq c_0(\Lambda)$.
\end{theorem}

\begin{proof}
  \proves{first_target}
  \uses{explicit}
  Use $\Lambda = \mathbb Z$, and the isomorphism $\Hom(\mathbb Z, A) \cong A$.
\end{proof}

We note that the constants $c_1,c_2,\ldots$ implicit in the choice of the complex
are chosen once and for all (after fixing $r$ and $r'$),
and it can be ensured that the transition maps in the complex are norm-nonincreasing.
Indeed, if the $c_i$ are suitable as in Definition~\ref{BD_suitable},
the maps
\[
\widehat{V}(\Hom(\Lambda,\overline{\mathcal M}_{r'}(S))_{\leq c_ic}^{a_i})\to \widehat{V}(\Hom(\Lambda,\overline{\mathcal M}_{r'}(S))_{\leq c_{i+1}c}^{a_{i+1}})
\]
will have bounded norm, independently of $V$ (as they are a certain universal finite sum of maps induced by maps between the profinite sets in paranthesis, each of which induces a map of norm bounded by $1$), so on the subspace of $T^{-1}$-invariants, one can shrink the norm down to $1$ by shrinking $c_{i+1}$. We make and fix this choice of the $c_i$ for the statement of Theorem~\ref{explicit}, and the rest of the proof.

\begin{proposition}
  \label{spectral}
  \uses{admissible}
  \uses{is_bdd_exact_for_bdd_degree_above_idx}
  Fix an integer $m\geq 0$ and a constant $k$.
  Then there exists an $\epsilon>0$ and a constant $k_0$,
  depending (only) on $k$ and $m$, with the following property.

  Consider an admissible system of double complexes $M^{p,q}_c$, $p,q\geq 0$, $c\geq c_0$,
  of complete normed abelian groups as well as some $k'\geq k_0$ and some $H>0$, such that
  \begin{center}
    \begin{tikzcd}
      M^{0,0}_c \ar[r]{d'^{0,0}_c}\ar[d]{d^{0,0}_c} & M^{0,1}_c\ar[r]{d'^{0,1}_c}\ar[d]{d^{0,1}_c} & M^{0,2}_c\ar[r]{d'^{0,2}_c}\ar[d]{d^{0,2}_c} & \ldots\\
  M^{1,0}_c\ar[r]{d'^{1,0}_c}\ar[d]{d^{1,0}_c} & M^{1,1}_c\ar[r]{d'^{1,1}_c}\ar[d]{d^{1,1}_c} & M^{1,2}_c\ar[r]{d'^{1,2}_c}\ar[d]{d^{1,2}_c} & \ldots\\
  M^{2,0}_c\ar[r]{d'^{2,0}_c}\ar[d]{d^{2,0}_c} & M^{2,1}_c\ar[r]{d'^{2,1}_c}\ar[d]{d^{2,1}_c} & \ddots\\
  \vdots & \vdots
    \end{tikzcd}
  \end{center}
  \begin{enumerate}
  \item for $j=0,\ldots,m$, the columns $M^{p,j}_c$ are $\leq k$-exact in degrees $\leq m$ for $c\geq c_0$;
  \item for $i=0,\ldots,m+1$, the rows $M^{i,q}_c$ are $\leq k$-exact in degrees $\leq m-1$ for $c\geq c_0$;
  \item for $q=0,\ldots,m$ and $c\geq c_0$, there is a map $h^q_{k'c}: M^{0,q+1}_{k'c}\to M^{1,q}_c$ with
  \[
  ||h^q_{k'c}(x)||_{M^{1,q}_c}\leq H||x||_{M^{0,q+1}_{k'c}}
  \]
  for all $x\in M^{0,q+1}_c$, and such that for all $c\geq c_0$ and $x\in M^{0,q}_{k'^2c}$, one has
  \begin{equation}\label{eq:homotopicmapsmall}
  ||\mathrm{res}_{k'^2c,k'c}^{1,q}(d^{0,q}(x))\pm h^q_{k'^2c}(d'^{0,q}_{k'^2c}(x))\pm d'^{1,q-1}_{k'c}(h^{q-1}_{k'^2c}(x))||_{M^{1,q}_{k'c}}\leq \epsilon ||\mathrm{res}_{k'^2c,c}^{0,q}(x)||_{M^{0,q}_c}.
  \end{equation}
  \end{enumerate}
  Then the first row is $\leq \max(k'^2,2k_0H)$-exact in degrees $\leq m$ for $c\geq c_0$.
\end{proposition}

We note that the bound on the homotopy is of a peculiar nature, in that the bound only depends on a deep restriction of $x$.

\begin{proof}
  \proves{spectral}
  \uses{snakelemma}
  First, we treat the case $m=0$.
  If $m=0$, we claim that one can take $\epsilon=\tfrac 1{2k}$ and $k_0=k$.
  We have to prove exactness at the first step.
  Let $x_{k'^2c}\in M^{0,0}_{k'^2c}$ and
  denote $x_{k'c}=\mathrm{res}_{k'^2c,k'c}^{0,0}(x)$
  and $x_c=\mathrm{res}_{k'^2c,c}^{0,0}(x)$.
  Then by assumption (1) (and $k'\geq k$), we have
  \[
  ||x_c||_{M^{0,0}_c}\leq k||d^{0,0}_{k'c}(x_{k'c})||_{M^{1,0}_{k'c}}.
  \]
  On the other hand, by (3),
  \[
  ||\mathrm{res}_{k'^2c,k'c}^{1,0}(d^{0,0}_{k'^2c}(x))\pm h^0_{k'^2c}(d'^{0,0}_{k'^2c}(x))||_{M^{1,0}_{k'c}}\leq \epsilon ||x_c||_{M^{0,0}_c}.
  \]
  In particular, noting that $\mathrm{res}_{k'^2c,k'c}^{1,0}(d^{0,0}_{k'^2c}(x)) = d^{0,0}_{k'c}(x_{k'c})$, we get
  \[
  ||x_c||_{M^{0,0}_c}\leq k||d^{0,0}_{k'c}(x_{k'c})||_{M^{1,0}_{k'c}}\leq k\epsilon ||x_c||_{M^{0,0}_c} + kH ||d'^{0,0}_{k'^2c}(x)||_{M^{0,1}_{k'^2c}}.
  \]
  Thus, taking $\epsilon=\tfrac 1{2k}$ as promised, this implies
  \[
  ||x_c||_{M^{0,0}_c}\leq 2kH ||d'^{0,0}_{k'^2c}(x)||_{M^{0,1}_{k'^2c}}.
  \]
  This gives the desired $\leq \max(k'^2,2k_0H)$-exactness in degrees $\leq m$ for $c\geq c_0$.

  Now we argue by induction on $m$.
  Consider the complex $N^{p,q}$ given by $M^{p,q+1}$ for $q\geq 1$
  and $N^{p,0} = M^{p,1}/\overline{M^{p,0}}$
  (the quotient by the closure of the image, which is also the completion of $M^{p,1}/M^{p,0}$),
  equipped with the quotient norm.
  Using the normed version of the snake lemma,
  Proposition~\ref{snakelemma} in the appendix to this lecture,
  one checks that this satisfies the assumptions for $m-1$,
  with $k$ replaced by $\max(k^4,k^3+k+1)$.
\end{proof}

Finally, we can prove the key combinatorial lemma,
ensuring that any element of $\Hom(\Lambda,\overline{\mathcal M}_{r'}(S))$
can be decomposed into $N$ elements whose norm is roughly $\tfrac 1N$ of the original element.
As preparation, we have the following simple result.

\begin{lemma}
  \label{combi_aux}
  Let $\Lambda$ be a finite free abelian group,
  let $N$ be a positive integer,
  and let $\lambda_1,\ldots,\lambda_m\in \Lambda$ be elements.
  Then there is a finite subset $A\subset \Lambda^\vee$
  such that for all $x\in \Lambda^\vee=\Hom(\Lambda,\mathbb Z)$
  there is some $x'\in A$ such that $x-x'\in N\Lambda^\vee$
  and for all $i=1,\ldots,m$,
  the numbers $x'(\lambda_i)$ and $(x-x')(\lambda_i)$ have the same sign,
  i.e.~are both nonnegative or both nonpositive.
\end{lemma}

\begin{proof}
  It suffices to prove the statement for all $x$ such that $\lambda_i(x)\geq 0$ for all $i$;
  indeed, applying this variant to all $\pm \lambda_i$, one gets the full statement.

  Thus, consider the submonoid $\Lambda^\vee_+\subset \Lambda^\vee$
  of all $x$ that pair nonnegatively with all $\lambda_i$.
  This is a finitely generated monoid by standard results;
  let $y_1,\ldots,y_M$ be a set of generators.
  Then we can take for $A$ all sums $n_1y_1+\ldots+n_My_M$ where all $n_j\in \{0,\ldots,N-1\}$.
\end{proof}

Now we have the key lemma:

\begin{lemma}
  \label{combi}
  \uses{polyhedral_lattice}
  Let $\Lambda$ be a polyhedral lattice.
  Then for all positive integers $N$ there is a constant $d$
  such that for all $c>0$ one can write any
  $x\in \Hom(\Lambda,\overline{\mathcal M}_{r'}(S))_{\leq c}$ as
  \[
    x=x_1+\ldots+x_N
  \]
  where all $x_i\in \Hom(\Lambda,\overline{\mathcal M}_{r'}(S))_{\leq c/N+d}$.
\end{lemma}

\begin{proof}
  \proves{combi}
  \uses{combi_aux}
  Pick $\lambda_1,\ldots,\lambda_m\in \Lambda$ generating the norm. We fix a finite subset $A\subset \Lambda^\vee$ satisfying the conclusion of the previous lemma. Write
  \[
  x=\sum_{n\geq 1, s\in S} x_{n,s} T^n [s]
  \]
  with $x_{n,s}\in \Lambda^\vee$. Then we can decompose
  \[
  x_{n,s} = N x_{n,s}^0 + x_{n,s}^1
  \]
  where $x_{n,s}^1\in A$ and we have the same-sign property of the last lemma. Letting $x^0 = \sum_{n\geq 1, s\in S} x_{n,s}^0 T^n [s]$, we get a decomposition
  \[
  x = Nx^0 + \sum_{a\in A} a x_a
  \]
  with $x_a\in \overline{\mathcal M}_{r'}(S)$ (with the property that in the
  basis given by the $T^n [s]$, all coefficients are $0$ or $1$). Crucially,
  we know that for all $i=1,\ldots,m$, we have
  \[
  ||x(\lambda_i)|| = N ||x^0(\lambda_i)|| + \sum_{a\in A} |a(\lambda_i)| ||x_a(\lambda_i)||
  \]
  by using the same sign property of the decomposition.

  Using this decomposition of $x$, we decompose each term into $N$ summands.
  This is trivial for the first term $Nx^0$,
  and each summand of the second term reduces to the similar problem for $\Lambda=\mathbb Z$.
  In that case, one can take $d=1$,
  as follows by decomposing any sum with terms of size at most $1$
  into $N$ such partial sums whose sums differ by at most $1$.
  (It follows that in general one can take for $d$
  the supremum over all $i$ of $\sum_{a\in A} |a(\lambda_i)|$.)
\end{proof}



\begin{proof}[Proof of Theorem~\ref{explicit}]
\proves{explicit}
\uses{spectral,combi,basehomotopy,homotopyNelements}
  We argue by induction on $m$, so assume the result for $m-1$
  (this is no assumption for $m=0$, so we do not need an induction start).
  This gives us some $k>1$ for which the statement of Theorem~\ref{explicit} holds true for $m-1$;
  if $m=0$, simply take any $k>1$.
  In the proof below, we will increase $k$ further in a way that depends only on $m$ and $r$.
  After this modified choice of $k$, we fix $\epsilon$ and $k_0$ as provided by Proposition~\ref{spectral}.
  Moreover, we let $k'$ be the supremum of $k_0$ and the $c_i'$ from
  Lemma~\ref{basehomotopy} (and~\ref{homotopyNelements}) for $i=0,\ldots,m$.
  Finally, choose a positive integer $b$ so that $2k'(\tfrac r{r'})^b\leq \epsilon$,
  and let $N$ be the minimal power of $2$ that satisfies
\[
k'/N\leq (r')^b.
\]
Then in particular $r^bN\leq \frac 2{k'}(\tfrac{r}{r'})^b\leq \epsilon$.

We consider the diagonal embedding
\[
\Lambda\hookrightarrow \Lambda' = \Lambda^N,
\]
where we endow $\Lambda'$ with the norm
\[
||(\lambda_1,\ldots,\lambda_N)||_{\Lambda'} = \tfrac 1N(||\lambda_1||_\Lambda+\ldots+||\lambda_N||_\Lambda).
\]
For any $m\geq 1$, let $\Lambda'^{(m)}$ be given by $\Lambda'^m / \Lambda\otimes (\mathbb Z^m)_{\sum=0}$; then $\Lambda'^{(\bullet)}$ is cosimplicial polyhedral lattice, the \v{C}ech conerve of $\Lambda\to \Lambda'$. For $m=0$, we set $\Lambda'^{(0)} = \Lambda$. It is clear that all of these are polyhedral lattices.

In particular, for any $c>0$, we have
\[
\Hom(\Lambda'^{(m)},\overline{\mathcal M}_{r'}(S))_{\leq c} = \Hom(\Lambda',\overline{\mathcal M}_{r'}(S))_{\leq c}^{m/\Hom(\Lambda,\overline{\mathcal M}_{r'}(S))_{\leq c}},
\]
the $m$-fold fibre product of $\Hom(\Lambda',\overline{\mathcal M}_{r'}(S))_{\leq c}$ over $\Hom(\Lambda,\overline{\mathcal M}_{r'}(S))_{\leq c}$; and
\[
\Hom(\Lambda',\overline{\mathcal M}_{r'}(S))_{\leq c} = \Hom(\Lambda,\overline{\mathcal M}_{r'}(S))_{\leq c/N}^N,
\]
with the map to $\Hom(\Lambda,\overline{\mathcal M}_{r'}(S))_{\leq c}$ given by the sum map.

Consider the collection of double complexes $C_{\Lambda'^{(\bullet)},c}^\bullet$ associated to this cosimplicial polyhedral lattice by Dold-Kan. Up to rescaling the norms in the complex for $\Lambda'^{(m)}$ by a universal constant (something like $(m+2)!$), the differentials are strictly compatible with norms (as they are an alternating sum of $m+1$ face maps, all of which are of norm $\leq 1$), so this collection of normed double complexes is admissible. By induction, the first condition of Proposition~\ref{spectral} is satisfied for all $c\geq c_0$ with $c_0$ large enough (depending on $\Lambda$ but not $V$ or $S$). By Lemma~\ref{combi}, and noting that $\Hom(\Lambda'^{(\bullet)},\overline{\mathcal M}_{r'}(S))_{\leq c}$ is the \v{C}ech nerve of
\[
\Hom(\Lambda,\overline{\mathcal M}_{r'}(S))_{\leq c/N}^N\xrightarrow{\sum} \Hom(\Lambda,\overline{\mathcal M}_{r'}(S))_{\leq c},
\]
also the second condition is satisfied, with $k$ the maximum of the previous $k$ and some constant depending only on $m$ and $r$, provided we take $c_0$ large enough so that $(k-1)r'c_ic_0/N$ is at least the $d$ of Lemma~\ref{combi} for all $i=0,\ldots,m$ (so this choice of $c_0$ again depends on $\Lambda$). Indeed, then one can splice a surjection of profinite sets between the maps
\[
\Hom(\Lambda,\overline{\mathcal M}_{r'}(S))_{\leq c_ic/N}^{Na}\to\Hom(\Lambda,\overline{\mathcal M}_{r'}(S))_{\leq c_ic}^a
\]
and
\[
\Hom(\Lambda,\overline{\mathcal M}_{r'}(S))_{\leq kc_ic/N}^{Na}\to \Hom(\Lambda,\overline{\mathcal M}_{r'}(S))_{\leq kc_ic}^a,
\]
and so the transition map between the columns of that double complex factors over a similar complex arising from a simplicial hypercover of profinite sets, so the constants are bounded by Proposition~\ref{prop:normedcompletion}, Lemma~\ref{lem:Tinv}, and Proposition~\ref{snakelemma}. At this point, we have finalized our choice of $k$ (and, as promised, this choice depended only on $m$ and $r$), and so we also finalized the constants $\epsilon$, $k'$ and $N$ from the first paragraph of the proof.

Finally, to check the third condition, we use Lemma~\ref{homotopyNelements} to find, in degrees $\leq m$, a homotopy between the two maps from the first row
\[
\widehat{V}(\Hom(\Lambda,\overline{\mathcal M}_{r'}(S))_{\leq c})^{T^{-1}}\to \widehat{V}(\Hom(\Lambda,\overline{\mathcal M}_{r'}(S))_{\leq c_1c}^2)^{T^{-1}}\to \ldots
\]
to the second row
\[
\widehat{V}(\Hom(\Lambda,\overline{\mathcal M}_{r'}(S))_{\leq c/N}^N)^{T^{-1}}\to \widehat{V}(\Hom(\Lambda,\overline{\mathcal M}_{r'}(S))_{\leq c_1c/N}^{2N})^{T^{-1}}\to \ldots
\]
respectively induced by the addition $\Hom(\Lambda,\overline{\mathcal M}_{r'}(S))_{\leq c/N}^N\to \Hom(\Lambda,\overline{\mathcal M}_{r'}(S))_{\leq c}$ (which is the map that forms part of the double complex), and the map that is the sum of the $N$ maps induced by the $N$ projection maps
\[
\Hom(\Lambda,\overline{\mathcal M}_{r'}(S))_{\leq c/N}^N\to \Hom(\Lambda,\overline{\mathcal M}_{r'}(S))_{\leq c/N}\subset \Hom(\Lambda,\overline{\mathcal M}_{r'}(S))_{\leq c}.
\]
By Lemma~\ref{homotopyNelements}, we can find this homotopy between the complex for $k'c$ and the complex for $c$, by our choice of $k'\geq c_i'$ for $i=0,\ldots,m$. As $N$ is fixed, the homotopy is the universal homotopy from Lemma~\ref{homotopyNelements}, and in particular its norm is bounded by some universal constant $H$.

Finally, it remains to establish the estimate \eqref{eq:homotopicmapsmall} on the homotopic map. We note that this takes $x\in \widehat{V}(\Hom(\Lambda,\overline{\mathcal M}_{r'}(S))_{\leq k'^2c_ic}^{a_i})^{T^{-1}}$ (with $i=q$ in the notation of \eqref{eq:homotopicmapsmall}) to the element
\[
y\in \widehat{V}(\Hom(\Lambda,\overline{\mathcal M}_{r'}(S))_{\leq k'c_ic/N}^{Na_i})^{T^{-1}}
\]
that is the sum of the $N$ pullbacks along the $N$ projection maps $\Hom(\Lambda,\overline{\mathcal M}_{r'}(S))_{\leq k'c_ic/N}^{Na_i}\to \Hom(\Lambda,\overline{\mathcal M}_{r'}(S))_{\leq k'^2c_ic}^{a_i}$. We note that these actually take image in $\Hom(\Lambda,\overline{\mathcal M}_{r'}(S))_{\leq c_ic}^{a_i}$ as $N\geq k'$, so this actually gives a well-defined map
\[
\widehat{V}(\Hom(\Lambda,\overline{\mathcal M}_{r'}(S))_{\leq c_ic}^{a_i})^{T^{-1}}\to \widehat{V}(\Hom(\Lambda,\overline{\mathcal M}_{r'}(S))_{\leq k'c_ic/N}^{Na_i})^{T^{-1}}.
\]
We need to see that this map is of norm $\leq \epsilon$. Now note that by our choice of $N$, we actually have $k'c_ic/N\leq (r')^b c_ic$, so this can be written as the composite of the restriction map
\[
\widehat{V}(\Hom(\Lambda,\overline{\mathcal M}_{r'}(S))_{\leq c_ic}^{a_i})^{T^{-1}}\to \widehat{V}(\Hom(\Lambda,\overline{\mathcal M}_{r'}(S))_{\leq (r')^b c_ic}^{a_i})^{T^{-1}}
\]
and
\[
\widehat{V}(\Hom(\Lambda,\overline{\mathcal M}_{r'}(S))_{\leq (r')^b c_ic}^{a_i})^{T^{-1}}\to \widehat{V}(\Hom(\Lambda,\overline{\mathcal M}_{r'}(S))_{\leq k'c_ic/N}^{Na_i})^{T^{-1}}.
\]
The first map has norm exactly $r^b$, by $T^{-1}$-invariance, and as multiplication by $T$ scales the norm with a factor of $r$ on $\widehat{V}$.\footnote{Here is where we use $r'>r$, ensuring different scaling behaviour of the norm on source and target.} The second map has norm at most $N$ (as it is a sum of $N$ maps of norm $\leq 1$). Thus, the total map has norm $\leq r^bN$. But by our choice of $N$, we have $r^bN\leq \epsilon$, giving the result.

Thus, we can apply Proposition~\ref{spectral}, and get the desired $\leq \max(k'^2,2k_0H)$-exactness in degrees $\leq m$ for $c\geq c_0$, where $k'$, $k_0$ and $H$ were defined only in terms of $k$, $m$, $r'$ and $r$, while $c_0$ depends on $\Lambda$ (but not on $V$ or $S$). This proves the inductive step.
\end{proof}

\begin{question} Can one make the constants explicit, and how large are they?\footnote{A back of the envelope calculation seems to suggest that $k$ is roughly doubly exponential in $m$, and that $N$ has to be taken of roughly the same magnitude.} Modulo the Breen-Deligne resolution, all the arguments give in principle explicit constants; and actually the proof of the existence of the Breen-Deligne resolution should be explicit enough to ensure the existence of bounds on the $c_i$ and $c_i'$.
\end{question}

This completes the proof of all results announced so far.

\newpage

\section*{Appendix to Lecture IX: Some normed homological algebra}

In this appendix, we gather a few results about homological algebra with normed abelian groups, the proofs of which are just obtained by keeping track of constants in the standard proofs.


\begin{proposition}
  \label{snakelemma}
  \uses{is_bdd_exact_for_bdd_degree_above_idx, admissible}
  Let $M^\bullet_c$ and $M'^\bullet_c$ be two admissible collections
  of complexes of complete normed abelian groups, where $c\geq c_0$.
  Let $f^\bullet_c: M^\bullet_c\to M'^\bullet_c$ be a collection of maps
  between these collections of complexes
  that is strictly compatible with the norm and commutes with restriction maps,
  and assume that it satisfies
  \[
  ||\mathrm{res}^i_{kc,c}(x)||_{M^i_c}\leq k||f^i_{kc}(x)||_{M'^i_{kc}}
  \]
  for all $i=0,\ldots,m+1$ and all $x\in M^i_{kc}$.
  Let $N^\bullet_c=M'^\bullet_c/\overline{M^\bullet_c}$
  (which equals the completion of $M'^{\bullet}_c/M^\bullet_c$)
  be the collection of quotient complexes, with the quotient norm;
  this is again an admissible collection of complexes.

  Assume that $M^\bullet_c$ and $M'^\bullet_c$ are $\leq k$-exact in degrees $\leq m$ for $c\geq c_0$.
  Then $N^\bullet_c$ is $\leq \max(k^4,k^3+k+1)$-exact in degrees $\leq m-1$ for $c\geq c_0$.
\end{proposition}

\begin{proof}
  We make the following preliminary observation.
  Take any $i=0,\ldots,m+1$ and $m'_{kc}\in M'^i_{kc}$ with image $n_{kc}\in N^i_{kc}$.
  By the definition of the quotient norm,
  for any $\epsilon>0$ we can find some $m_{kc}\in M^i_{kc}$
  such that $||m'_{kc}-f^i_{kc}(m_{kc})||\leq ||n_{kc}||+\epsilon$.
  We would like to replace this by the stronger assertion that we can find $m_{kc}\in M^i_{kc}$ such that
\[
||m'_{kc}-f^i_{kc}(m_{kc})||\leq (1+\epsilon)||n_{kc}||.
\]
This is obviously possible as long as $||n_{kc}||>0$, but in case $||n_{kc}||=0$, it may not be possible, because $M^\bullet_c\to M'^\bullet_c$ may not have closed image.

However, we claim that, letting $m'_c\in M'^i_c$ be the restriction of $m'_{kc}\in M'^i_{kc}$, with image $n_c\in N^i_c$, we can always find some $m_c\in M^i_c$ such that
\[
||m'_c-f^i_c(m_c)||\leq (1+\epsilon)||n_{kc}||.
\]
By the above, we only need to prove this when $||n_{kc}||=0$. Choose a sequence $m_{kc,0},m_{kc,1},\ldots$ in $M^i_{kc}$ such that $||m'_{kc}-f^i_{kc}(m_{kc,j})||\to 0$ for $j\to \infty$. In particular, $||f^i_{kc}(m_{kc,j}-m_{kc,j'})||\to 0$ for $j,j'\to \infty$. By the displayed bound in the statement of the proposition, this ensure that $||m_{c,j}-m_{c,j'}||\to 0$ where $m_{c,j}\in M^i_c$ is the image of $m_{kc,j}$. Thus, we get a Cauchy sequence in $M^i_c$ whose limit $m_c\in M^i_c$ will satisfy $||m'_c-f^i_c(m_c)||=0$ (i.e.~$m'_c=f^i_c(m_c)$).

Now we start the proof of the proposition. Let $n^i_{k^4c}\in N^i_{k^4c}$ for $i\leq m-1$, with image $n^{i+1}_{k^4c}\in N^{i+1}_{k^4c}$, and let $C:=||n^{i+1}_{k^4c}||_{N^{i+1}_{k^4c}}$. We need to find an element $n^i_c\in N^{i-1}_c$ such that
\[
||n^i_c - d^{i-1}_{N,c}(n^{i-1}_c)||_{N^i_c}\leq (k^3+k+1)C,
\]
where we change the subscript when applying restriction maps.

Pick any preimage $m'^i_{k^4c}\in M'^i_{k^4c}$ of $n^i_{k^4c}$, and let $m'^{i+1}_{k^4c}\in M'^{i+1}_{k^4c}$ be its image. By the preliminary observation, we can find $m^{i+1}_{k^3c}\in M^{i+1}_{k^3c}$ such that
\[
m'^{i+1}_{k^3c} = f^{i+1}_{k^3c}(m^{i+1}_{k^3c}) + m''^{i+1}_{k^3c}
\]
with $||m''^{i+1}_{k^3c}||_{M'^{i+1}_{k^3c}}\leq (1+\epsilon)C$, where we choose $\epsilon$ so that $(k^3+k)(1+\epsilon)\leq k^3+k+1$.

Let $m^{i+2}_{k^3c}\in M^{i+2}_{k^3c}$ be the image of $m^{i+1}_{k^3c}$. Applying the differential to the last displayed equation, and using that this kills $m'^{i+1}_{k^3c}$, and that $f^\bullet_{k^3c}$ is a map of complexes, we see that
\[
f^{i+2}_{k^3c}(m^{i+2}_{k^3c}) = -m''^{i+2}_{k^3c},
\]
where similarly $m''^{i+2}_{k^3c}$ is the differential of $m''^{i+1}_{k^3c}$. We get
\[\begin{aligned}
||m^{i+2}_{k^2c}||_{M^{i+2}_{k^2c}}&\leq k||f^{i+2}_{k^3c}(m^{i+2}_{k^3c})||_{M'^{i+2}_{k^3c}} = k||m''^{i+2}_{k^3c}||_{M'^{i+2}_{k^3c}}\\
&\leq k||m'^{i+1}_{k^3c}||_{M'^{i+1}_{k^3c}}\leq k(1+\epsilon)C.
\end{aligned}\]
On the other hand, we can find some $m^i_{kc}\in M^i_{kc}$ such that
\[
||m^{i+1}_{kc}-d^i_{kc}(m^i_{kc})||\leq k||m^{i+2}_{k^2c}||_{M^{i+2}_{k^2c}}\leq k^2(1+\epsilon)C.
\]
Now let $m'^i_{kc,\mathrm{new}} = m'^i_{kc}-f^i_{kc}(m^i_{kc})\in M'^i_{kc}$; this is a lift of $n^i_{kc}$. Then the image $m'^{i+1}_{kc,\mathrm{new}}$ in $M'^{i+1}_{kc}$ satisfies
\[
m'^{i+1}_{kc,\mathrm{new}} = m'^{i+1}_{kc}-f^{i+1}_{kc}(m^{i+1}_{kc}) + f^{i+1}_{kc}(m^{i+1}_{kc}-d^i_{kc}(m^i_{kc})) = m''^{i+1}_{kc} + f^{i+1}_{kc}(m^{i+1}_{kc}-d^i_{kc}(m^i_{kc})).
\]
In particular,
\[
||m'^{i+1}_{kc,\mathrm{new}}||_{M'^{i+1}_{kc}}\leq (1+\epsilon)C+ k^2(1+\epsilon)C.
\]
Now we can find $m'^{i-1}_c\in M'^{i-1}_c$ such that
\[
||m'^i_{c,\mathrm{new}} - d'^{i-1}_c(m'^{i-1}_c)||_{M'^i_c}\leq k||m'^{i+1}_{kc,\mathrm{new}}||_{M'^{i+1}_{kc}}\leq (k^3+k)(1+\epsilon)C.
\]
In particular, letting $n^{i-1}_c\in N^{i-1}_c$ be the image of $m'^{i-1}_c$, we get
\[
||n^i_c - d^{i-1}_{N,c}(n^{i-1}_c)||_{N^i_c}\leq (k^3+k)(1+\epsilon)C,
\]
so by our choice of $\epsilon$ this gives the desired result.
\end{proof}

We need the following results about the Breen-Deligne resolution for normed abelian groups. Let us consider here abelian groups $M$ (in any topos) equipped with an increasing filtration $M_{\leq c}\subset M$ by subobjects indexed by the positive real numbers, such that $0\in M_{\leq c}$, $-M_{\leq c} = M_{\leq c}$ and $M_{\leq c}+M_{\leq c'}\subset M_{\leq c+c'}$; we need no further conditions. Let us call these pseudo-normed abelian groups.

Fix a choice of a functorial Breen-Deligne resolution
\[
C(M):\ldots \to \mathbb Z[M^{a_i}]\to\ldots\to\mathbb
Z[M^{a_1}]\to\mathbb Z[M^{a_0}]\to M\to 0
\]
of an abelian group $M$; purely for notational convenience, we can and do assume that each term is of the form $\mathbb Z[M^{a_i}]$ (as opposed to a finite direct sum of such). The possibility of doing this follows from the proof of \cite[Theorem 4.10]{Condensed}, noting that a functor of the form $A\mapsto \mathbb Z[A^n]\oplus \mathbb Z[A^m]$ admits a surjection from the functor $A\mapsto \mathbb Z[A^{n+m}]\oplus \mathbb Z$; this gives a resolution where all terms are of the form $\mathbb Z[A^{a_i}]\oplus \mathbb Z^m$. Now pass to the quotient of these complexes corresponding to the map $0\to A$; this gives a complex all of whose terms are of the form $\mathbb Z[A^{a_i}]/\mathbb Z$. Noting that $\mathbb Z[A^{a_i}]$ is functorially isomorphic to $\mathbb Z[A^{a_i}]/\mathbb Z\oplus \mathbb Z$ (via splitting $0\to A^{a_i}\to 0$), we can then add an acyclic complex of $\mathbb Z$'s in each degree to get a resolution all of whose terms are of the form $\mathbb Z[A^{a_i}]$.

%% \begin{lemma}\label{lem:constantsdeligne} There are universal constants $c_0=1,c_1,c_2,\ldots$ so that the Breen-Deligne resolution admits the subcomplex
%% \[
%% C(M)_{\leq c}: \ldots \to \mathbb Z[M^{a_i}_{\leq c_ic}]\to\ldots\to\mathbb
%% Z[M^{a_1}_{\leq c_1c}]\to\mathbb Z[M^{a_0}_{\leq c}]
%% \]
%% for all pseudo-normed abelian group objects in any topos as above, and all $c>0$.
%% \end{lemma}

%% \begin{proof} Each differential in the Breen-Deligne resolution is a finite sum of maps induced by maps $M^{a_{i+1}}\to M^{a_i}$ given by some $a_i\times a_{i+1}$-matrix of integers. Given $c_i$, one can thus find some $c_{i+1}$ so that $M^{a_{i+1}}_{\leq c_{i+1}c}$ maps into $M^{a_i}_{\leq c_ic}$ for each of those finitely many maps, which gives the claim.
%% \end{proof}

We also need some homotopies. More precisely, we start with the following homotopy.

\begin{lemma}\label{basehomotopy} For an abelian group $M$, the maps $\sigma_1,\sigma_2$ from
\[
C(M^2): \ldots \to \mathbb Z[M^{2a_i}]\to\ldots\to\mathbb Z[M^{2a_1}]\to\mathbb Z[M^{2a_0}]
\]
to
\[
C(M): \ldots \to \mathbb Z[M^{a_i}]\to\ldots\to\mathbb Z[M^{a_1}]\to\mathbb Z[M^{a_0}],
\]
induced by addition $M^2\to M$, respectively the sum of the two maps induced by two projections $M^2\to M$, are homotopic, via some functorial homotopy
\[
h_i: \mathbb Z[M^{2a_i}]\to \mathbb Z[M^{a_{i+1}}].
\]

If $M$ is a pseudo-normed abelian group object in any topos, then $\sigma_1$ and $\sigma_2$ are well-defined as maps of complexes from
\[
C(M^2)_{\leq c/2}: \ldots \to \mathbb Z[M^{2a_i}_{\leq c_ic/2}]\to\ldots\to\mathbb
Z[M^{2a_1}_{\leq c_1c/2}]\to\mathbb Z[M^{2a_0}_{\leq c/2}]
\]
to
\[
C(M)_{\leq c}: \ldots \to \mathbb Z[M^{a_i}_{\leq c_ic}]\to\ldots\to\mathbb
Z[M^{a_1}_{\leq c_1c}]\to\mathbb Z[M^{a_0}_{\leq c}]
\]
for all $c>0$. In that case, for all $i\geq 0$ there are universal constants $c_i'$ such that $h_i$ defines well-defined maps
\[
\mathbb Z[M^{2a_i}_{\leq c_ic/2}]\to \mathbb Z[M^{a_{i+1}}_{\leq
c_i'c_{i+1}c}]
\]
for all $c>0$.
\end{lemma}

\begin{proof} This is a consequence of the proof of the existence of the Breen-Deligne resolution, proved in the same way as \cite[Proposition 4.17]{Condensed}. The existence of the constants $c_i'$ is again formal, as in the last lemma.
\end{proof}

Now we need the following generalization to adding $N$ elements.

\begin{lemma}\label{homotopyNelements} Let $N$ be a power of $2$. The maps of complexes $\sigma_1,\sigma_2$ from
\[
C(M^N): \ldots \to \mathbb Z[M^{Na_i}]\to\ldots\to\mathbb Z[M^{Na_1}]\to\mathbb Z[M^{Na_0}]
\]
to
\[
C(M): \ldots \to \mathbb Z[M^{a_i}]\to\ldots\to\mathbb Z[M^{a_1}]\to\mathbb Z[M^{a_0}],
\]
induced by addition $M^N\to M$, respectively the sum of the $N$ maps induced by the $N$ projections $M^N\to M$, are homotopic, via some functorial homotopy
\[
h_i^N: \mathbb Z[M^{Na_i}]\to \mathbb Z[M^{a_{i+1}}]
\]
which moreover satisfies the following bound, with the same constants $c_0',c_1',\ldots$ as in the previous lemma:

If $M$ is a pseudo-normed abelian group object in any topos, then $\sigma_1$ and $\sigma_2$ are well-defined as maps of complexes from
\[
C(M^N)_{\leq c/N}: \ldots \to \mathbb Z[M^{Na_i}_{\leq c_ic/N}]\to\ldots\to\mathbb Z[M^{Na_1}_{\leq c_1c/N}]\to\mathbb Z[M^{Na_0}_{\leq c/N}]
\]
to
\[
C(M)_{\leq c}: \ldots \to \mathbb Z[M^{a_i}_{\leq c_ic}]\to\ldots\to\mathbb Z[M^{a_1}_{\leq c_1c}]\to\mathbb Z[M^{a_0}_{\leq c}]
\]
for all $c>0$. In that case, $h_i^N$ defines well-defined maps
\[
\mathbb Z[M^{Na_i}_{\leq c_ic/N}]\to \mathbb Z[M^{a_{i+1}}_{\leq c_i'c_{i+1}c}]
\]
for all $c>0$.
\end{lemma}

\begin{proof} Let $N=2^m$. For each $j=0,\ldots,m-1$, the two maps from $C(M^{2^{j+1}})$ to $C(M^{2^j})$ from the previous lemma are homotopic, and we use the homotopy from that lemma. Composing homotopies (which amounts concretely to a certain sum) we get the desired homotopy from $C(M^{2^m})$ to $C(M)$. It follows directly from this construction that the constants $c_i'$ are unchanged.
\end{proof}
