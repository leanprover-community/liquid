\maketitle

\subsection{What is the Liquid Tensor Experiment?}

A collaborative project to formally verify parts of condensed mathematics,
a new subject developed by Dustin Clausen and Peter Scholze.
We use a computer program called Lean,
in which we enter all the definitions, theorems and proofs,
upon which Lean checks the mathematics down to the axioms.

The name ``Liquid Tensor Experiment'' is a homage
to the progressive rock band Liquid Tension Experiment,
as well as a reference to the liquid mathematics that is formalized in this project.

\subsection{How it all started}

In December 2020, Peter Scholze published a blogpost:
\url{https://xenaproject.wordpress.com/2020/12/05/liquid-tensor-experiment/}.
In this blogpost, he challenged the proof assistant communities
to formally verify one of the main results of his lecture notes on
Analytic Geometry, \url{www.math.uni-bonn.de/people/scholze/Analytic.pdf}.

In the Lean community, we took up this challenge.
Within half a year, we reached the first major milestone.
Scholze wrote about his experiences in a follow-up blogpost,
\url{https://xenaproject.wordpress.com/2021/06/05/half-a-year-of-the-liquid-tensor-experiment-amazing-developments/}.

\subsection{Where can I learn more about Lean?}

A good place to start is \url{https://leanprover-community.github.io/learn.html}.
There is also a Zulip chatroom, where there is always a helpful person to say hello and help you along,
\url{https://leanprover.zulipchat.com}.

\subsection{Liquid Tensor Experiment in the media}

Recently our project was mentioned in various media.
You can read about it in Nature,
\url{https://www.nature.com/articles/d41586-021-01627-2}.

\subsection*{Introduction}
\label{intro}

The goal of this document is to provide a detailed account
of the proof of the following theorem,
along side a computer verification in the Lean theorem prover.

\begin{theoremx}[Clausen--Scholze]
  \label{main-goal}
  Let $0 < p' < p \le 1$ be real numbers,
  let $S$ be a profinite set,
  and let $V$ be a $p$-Banach space.
  Let $\Rm(S)$ be the space of $p'$-measures on $S$.
  Then
  \[
    \Ext^i_{\Cond(\Ab)}(\Rm(S), V)=0
  \]
  for $i \ge 1$.
\end{theoremx}

This document consists of two parts, and there is some duplication between the two parts.
The first half gives a detailed and self-contained proof of the highly technical Theorem~\ref{first_target}.
The second half is meant to be readable in a stand-alone fashion, and therefore repeats some material of the first half.
It is concerned with deducing Theorem~\ref{main-goal} from Theorem~\ref{first_target}.

This document can be consumed in PDF format,
but it is designed first and foremost for interactive reading.
An online copy is available at
\texttt{https://leanprover-community.github.io/liquid/}.
This online version includes hyperlinks to the Lean code that formally verifies the proofs,
as well as two dependency graphs (one for each part of the document) that visualize the global structure of the proof.

For more information about the Lean interactive proof assistant, and formal verification of mathematics,
we refer to \texttt{https://leanprover-community.github.io}.


\section{Breen--Deligne data}

The goal of this section is to a give a precise statement of the Breen--Deligne resolution.
We first give the statement, and provide details later.

\begin{theoremx}[Breen--Deligne]
  \label{BD_orig}
  For an abelian group $A$, there is a resolution, functorial in~$A$, of the form
  \[
    \ldots \to \bigoplus_{i=1}^{n_i} \mathbb Z[A^{r_{ij}}] \to \ldots
    \to \mathbb Z[A^3] \oplus \mathbb Z[A^2] \to \mathbb Z[A^2] \to \mathbb Z[A] \to A \to 0.
  \]
\end{theoremx}

I (Johan Commelin) have not figured out the details.
But it seems to be possible to avoid the $\bigoplus_{i=1}^{n_i}$,
so we will aim for something like the following statement.

\begin{theoremx}
  \label{BD_reso}
  For an abelian group $A$, there is a resolution, functorial in~$A$, of the form
  \[
    \ldots \to \mathbb Z[A^{n_{i}}] \to \ldots \to \mathbb Z[A^2] \to \mathbb Z[A] \to A \to 0.
  \]
\end{theoremx}

What does a homomorphism $f \colon \mathbb Z[A^m] \to \mathbb Z[A^n]$
that is functorial in~$A$ look like? We should perhaps say more precisely
what we mean by this. The idea is that $m$ and $n$ are fixed, and
for each abelian group $A$ we have a group homomorphism
$f_A\colon \mathbb Z[A^m] \to \mathbb Z[A^n]$ such that if $\phi:A\to B$
is a group homomorphism inducing $\phi_i:\Z[A^i]\to\Z[B^i]$ for each
natural number $i$ then the obvious square commutes: $\phi_n \circ f_A = f_B \circ \phi_m$.

The map $f_A$ is specified by what it does to the generators
$(a_1, a_2, a_3, \dots, a_m)\in A^m$. It can send such an element
to an arbitrary element of $\mathbb Z[A^n]$, but one can check
that universality implies that $f_A$ will be a $\mathbb Z$-linear combination of
``basic universal maps'', where a ``basic universal map'' is one that
sends $(a_1, a_2, \dots, a_m)$ to $(t_1, \dots, t_n)$,
where $t_i$ is a $\mathbb Z$-linear combination $c_{i,1} \cdot a_1 + \dots + c_{i,m} \cdot a_m$.
So a ``basic universal map'' is specified by the $n \times m$-matrix $c$.

\begin{definition}
  \label{basic_universal_map}
  \lean{breen_deligne.basic_universal_map}
  \leanok
  A \emph{basic universal map} from exponent $m$ to $n$,
  is an $n \times m$-matrix with coefficients in~$\mathbb Z$.
\end{definition}

\begin{definition}
  \label{universal_map}
  \lean{breen_deligne.universal_map}
  \leanok
  \uses{basic_universal_map}
  A \emph{universal map} from exponent $m$ to $n$,
  is a formal $\mathbb Z$-linear combination of basic universal maps from exponent $m$ to $n$.
\end{definition}

We point out that basic universal maps can be composed by matrix multiplication,
and this formally induces a composition of universal maps. As mentioned above, one can also check (this has been formalised in Lean) that this construction gives a bijection between universal maps from exponent $m$ to $n$ and functorial collections $f_A:\Z[A^m]\to\Z[A^n]$.

\begin{definition}
  \label{sigma_add}
  \lean{breen_deligne.σ_add}
  \leanok
  \uses{universal_map}
  The addition on $A^n$ induces a universal map
  $\sigma_\alpha \colon \mathbb Z[(A^n)^2] \to \mathbb Z[A^n]$,
  namely the formal generator $(I_n I_n)$, where $I_n$ denotes the $n \times n$ identity matrix.
  (Here $\alpha$ stands for ``addition''.)
\end{definition}

\begin{definition}
  \label{sigma_proj}
  \lean{breen_deligne.σ_proj}
  \leanok
  \uses{universal_map}
  The formal sum of the two projections $(A^n)^2 \to A^n$
  induces a universal map $\sigma_\pi \colon \mathbb Z[(A^n)^2] \to \mathbb Z[A^n]$,
  namely the formal sum $(I_n 0) + (0 I_n)$,
  where $I_n$ denotes the $n \times n$ identity matrix, and $0$ the $n \times n$ zero matrix.
  (Here $\pi$ stands for ``projections''.)
\end{definition}

\begin{definition}
  \label{BD_double}
  \lean{breen_deligne.universal_map.double}
  \leanok
  \uses{universal_map}
  Let $f$ be a universal map from exponent~$m$ to~$n$.
  Then $f \oplus f$ denotes the universal map from exponent~$2m$ to~$2n$,
  that applies $f$ componentwise.
  If $f$ is a generator (i.e.\ a basic universal map)
  then $f \oplus f$ is
  \[
    \begin{pmatrix}
    f & 0 \\
    0 & f
    \end{pmatrix}.
  \]
\end{definition}

\begin{definition}
  \label{BD_data}
  \lean{breen_deligne.data}
  \lean{breen_deligne.is_complex}
  \leanok
  \uses{sigma_add, sigma_proj}
  A tuple $(n, f)$ of \emph{Breen--Deligne data}
  consists of a sequence of exponents $n_0, n_1, n_2, \dots \in \mathbb N$,
  and universal maps $f_i$ from exponent $n_{i+1}$ to $n_i$.

  Such a tuples is a \emph{complex} if for all $i$ we have $f_i \circ f_{i+1} = 0$.

  A \emph{universal morphism} of Breen--Deligne data (or complexes) $(m,f)\to (n,g)$ is a collection of universal maps $\phi_i$ from exponent $m_i$ to $n_i$ such that $g_i\circ \phi_{i+1}=\phi_i\circ f_i$ as universal maps from exponent $m_{i+1}$ to $n_i$ (i.e., the squares commute). 
\end{definition}

\begin{definition}
  \label{BD_data_double}
  \lean{breen_deligne.data.double}
  \leanok
  \uses{BD_data, BD_double}
  If $(n, f)$ is a tuple of Breen--Deligne data,
  then $(n, f) \oplus (n, f)$ is the tuple
  consisting of exponents $2n_i$ and universal maps $f_i \oplus f_i$.
\end{definition}

The two universal map $\sigma_\alpha$ and~$\sigma_\pi$ explained in the examples above, can be checked to induce universal maps of complexes: $(n,f) \oplus (n,f) \to (n,f)$.

\begin{definition}
  \label{BD_homotopy}
  \lean{breen_deligne.homotopy}
  \leanok
  \uses{BD_data, BD_double}
  A \emph{homotopy} for a tuple $(n, f)$ of Breen--Deligne data
  is a homotopy between the maps of complexes
  \[
    \sigma_\alpha, \sigma_\pi \colon (n,f) \oplus (n,f) \to (n,f)
  \]
  In other words, it consists of universal maps $h_i$ from exponent $2n_i$ to $n_{i+1}$,
  such that $f_0\circ h_0=\sigma_\alpha-\sigma_\pi$ as universal maps from exponent $2n_0$ to $n_0$, and for all $i\geq0$ we have
  \[
    f_{i+1} \circ h_{i+1} + h_i \circ (f_i \oplus f_i) = \sigma_\alpha - \sigma_\pi
    \]
    as universal maps from exponent $2n_{i+1}$ to $n_{i+1}$. Note that the first condition is morally the $i=-1$ case of the displayed equation, if we set $h_{-1}=0$.
\end{definition}

\begin{definition}
  \label{BD_package}
  \lean{breen_deligne.package}
  \leanok
  \uses{BD_data, BD_homotopy}
  A \emph{Breen--Deligne package}
  is a triple $(n, f, h)$,
  such that $(n, f)$ is Breen--Deligne data that is a complex,
  and $h$ is a homotopy for $(n,f)$.
\end{definition}

\begin{definition}
  \label{BD_eg}
  \lean{breen_deligne.eg}
  \leanok
  \uses{BD_package}
  We will now construct an example of a Breen--Deligne package.
  In some sense, it is the ``easiest'' solution to the conditions posed above.
  The exponents will be $n_i = 2^i$, and the homotopies $h_i$ will be the identity.
  Under these constraints, we recursively construct the universal maps $f_i$:
  \[
    f_0 = \sigma_\alpha - \sigma_\pi,
    \quad
    f_{i+1} = (\sigma_\alpha - \sigma_\pi) - (f_i \oplus f_i).
  \]
  We leave it as exercise for the reader, to verify that
  with these definitions $(n, f, h)$ forms a Breen--Deligne package.
\end{definition}

We now make three definitions that will make precise
some conditions between constants that will be needed
when we construct Breen--Deligne complexes of normed abelian groups.

\begin{definition}
  \label{basic_suitable}
  \lean{breen_deligne.basic_universal_map.suitable}
  \leanok
  \uses{basic_universal_map}
  Let $f$ be a basic universal map from exponent~$m$ to~$n$.
  Let $c_1, c_2 \in \mathbb R_{\ge 0}$.
  We say that $(c_1, c_2)$ is \emph{$f$-suitable}, if for all $i$
  \[
    \sum_j c_1|f_{ij}| \le c_2.
  \]
\end{definition}

To orient the reader: later on we will be considering maps on normed abelian groups induced from universal maps, and this inequality will guarantee that if $||m||\leq c_1$$ then $||f(m)||\leq c_2$$.

\begin{definition}
  \label{universal_suitable}
  \lean{breen_deligne.universal_map.suitable}
  \leanok
  \uses{universal_map, basic_suitable}
  Let $f$ be a universal map from exponent~$m$ to~$n$.
  Let $c_1, c_2 \in \mathbb R_{\ge 0}$.
  We say that $(c_1, c_2)$ is \emph{$f$-suitable}, if for all basic universal maps $g$
  that occur in the formal sum $f$,
  the pair of nonnegative reals $(c_1, c_2)$ is $g$-suitable.
\end{definition}

\begin{definition}
  \label{BD_suitable}
  \lean{breen_deligne.package.suitable}
  \leanok
  \uses{BD_package, universal_suitable}
  Let $(n, f, h)$ be a Breen--Deligne package,
  and let $c = (c_0, c_1, \dots)$ be a sequence of nonnegative real numbers.
  We say that $c$ is \emph{$(n,f,h)$-suitable},
  if for all $i$, the pair $(c_{i+1}, c_i)$ is $f_i$-suitable.

  (Note! The order $(c_{i+1}, c_i)$ is contravariant
  compared to Definition~\ref{universal_suitable}.
  This is because of the contravariance of $\hat V(\_)$;
  see Definition~\ref{eval_CLCFPTinv}.)
\end{definition}

% vim: ts=2 et sw=2 sts=2

\section{Variants of normed groups}

\begin{definition}
  \label{pseudo_normed_group}
  \lean{pseudo_normed_group,profinitely_filtered_pseudo_normed_group,
  profinitely_filtered_pseudo_normed_group_hom}
  \leanok
  A \emph{pseudo-normed group} is an abelian group $(M,+)$,
  together with an increasing filtration $M_c \subseteq M$ of subsets $M_c$ indexed by $\mathbb R_{\ge 0}$,
  such that each $M_c$ contains $0$, is closed under negation,
  and $M_{c_1} + M_{c_2} \subseteq M_{c_1 + c_2}$. An example would be $M=\mathbb{R}$ or $M=\mathbb{Q}_p$ with $M_c :=\{x\,:\,|x|\leq c\}$.

  A pseudo-normed group~$M$ is \emph{profinitely filtered}
  if each of the sets $M_c$ is endowed with a topological space structure
  making it a profinite set, such that following maps are all continuous:
  \begin{itemize}
    \item the inclusion $M_{c_1} \to M_{c_2}$ (for $c_1 \le c_2$);
    \item the negation $M_c \to M_c$;
    \item the addition $M_{c_1} \times M_{c_2} \to M_{c_1 + c_2}$.
  \end{itemize}

  
  A \emph{morphism} of profinitely filtered pseudo-normed groups $M \to N$
  is a group homomorphism $f$ that is
  \begin{itemize}
    \item \emph{bounded}:
      there is a constant $C$
      such that $x \in M_c$ implies $f(x) \in N_{Cc}$;
    \item \emph{continuous}:
      for one (or equivalently all) constants $C$ as above,
      the induced map $M_c \to N_{Cc}$ is
      a morphism of profinite sets, i.e. continuous.
  \end{itemize}

  The reason the two definitions are equivalent is that a continuous injection between profinite sets must be a topological embedding.
\end{definition}

\begin{definition}
  \label{profinitely_filtered_pseudo_normed_group_with_Tinv}
  \lean{profinitely_filtered_pseudo_normed_group_with_Tinv,
  profinitely_filtered_pseudo_normed_group_with_Tinv_hom}
  \leanok
  \uses{pseudo_normed_group}
  Let $r'$ be a positive real number.
  A profinitely filtered pseudo-normed group $M$
  has an \emph{$r'$-action of $T^{-1}$}
  if it comes endowed with a distinguished morphism
  of profinitely filtered pseudo-normed groups
  $T^{-1} \colon M \to M$
  that is bounded by $r'^{-1}$:
  if $x \in M_c$ then $T^{-1}x \in M_{c/r'}$.

  A morphism $M \to N$
  of profinitely filtered pseudo-normed groups with $r'$-action of $T^{-1}$
  is a morphism of profinitely filtered pseudo-normed groups $f$
  that commutes with the action of $T^{-1}$
  and is \emph{strict}: if $x \in M_c$ then $f(x) \in N_c$.
\end{definition}

\begin{definition}
  \label{normed_with_aut}
  \lean{normed_with_aut}
  \leanok
  Let $r > 0$ be a real number.
  An \emph{$r$-normed $\mathbb Z[T^{\pm 1}]$-module}
  is a normed abelian group $V$
  endowed with an automorphism $T \colon V \to V$ such that
  for all $v \in V$ we have $\|T(v)\| = r\|v\|$.
\end{definition}

% vim: ts=2 et sw=2 sts=2

\section{Spaces of convergent power series}

We will now construct the central example of
profinitely filtered pseudo-normed groups with $r'$-action of $T^{-1}$.

\begin{definition}
  \label{Mbar}
  \lean{Mbar,Mbar_le,Mbar.pseudo_normed_group}
  \uses{pseudo_normed_group}
  \leanok
  Let $r' > 0$ be a real number, and let $S$ be a finite set.
  Denote by $\overline{\mathcal M}_{r'}(S)$ the set
  \[
    \left\{ \left( \sum_{n \ge 1} a_{n,s} T^n \in T\Z[[T]]\right)_{s \in S} \,\middle\vert\, \sum_{n \ge 1, s \in S} |a_{n,s}| (r')^n < \infty \right\}.
  \]

  Note that $\overline{\mathcal M}_{r'}(S)$ is naturally a pseudo-normed group
  with filtration given by
  \[
    \overline{\mathcal M}_{r'}(S)_{\le c} =
    \left\{ \left( \sum_{n \ge 1} a_{n,s} T^n \right)_{s \in S} \middle\vert \sum_{n \ge 1, s \in S} |a_{n,s}| (r')^n \le c \right\}.
  \]
\end{definition}

\begin{lemma}
  \label{Mbar_profinite_filtered}
  \lean{Mbar.profinitely_filtered_pseudo_normed_group}
  \leanok
  \uses{Mbar}
  Let $r' > 0$ and $c \ge 0$ be real numbers, and let $S$ be a finite set.
  The space $\overline{\mathcal M}_{r'}(S)_{\le c}$ is the profinite limit of the finite sets
  \[
    \overline{\mathcal M}_{r'}(S)_{\le c, \le N} =
    \left\{ \left( \sum_{n \ge 1} a_{n,s} T^n \right)_{s \in S} \middle\vert
    \sum_{1 \le n \le N, s \in S} |a_{n,s}| (r')^n \le c \right\}
  \]
  This endows $\overline{\mathcal M}_{r'}(S)_{\le c}$ with the profinite topology.
  In particular, it is a profinitely filtered pseudo-normed group.
\end{lemma}

\begin{proof}
  \leanok
  Formalised, but omitted from this text.
\end{proof}

For the remainder of this section,
let $r' > 0, c \ge 0$ be real numbers,
and let $S$ be a finite set.

\begin{definition}
  \label{Mbar_Tinv}
  \lean{Mbar.Tinv}
  \leanok
  \uses{Mbar}
  There is a natural action of $T^{-1}$ on $\overline{\mathcal M}_{r'}(S)$, via
  \[
    T^{-1} \cdot
    \left( \sum_{n \ge 1} a_{n,s} T^n \right)_{s \in S} =
    \left( \sum_{n \ge 1} a_{n+1,s} T^n \right)_{s \in S}.
  \]
\end{definition}

\begin{lemma}
  \label{Mbar_with_Tinv}
  \lean{Mbar.profinitely_filtered_pseudo_normed_group_with_Tinv}
  \leanok
  \uses{Mbar_profinitely_filtered, Mbar_Tinv}
  The natural action of $T^{-1}$ on $\overline{\mathcal M}_{r'}(S)$
  restricts to continuous maps
  \[
    T^{-1} \cdot \_ \colon
    \overline{\mathcal M}_r(S)_{\le c} \to
    \overline{\mathcal M}_r(S)_{\le c/r'}.
  \]
  In particular, $\overline{\mathcal M}_{r'}(S)$
  has an $r'$-action of $T^{-1}$.
\end{lemma}

\begin{proof}
  \leanok
  Formalised, but omitted from this text.
\end{proof}

% vim: ts=2 et sw=2 sts=2

\section{Some normed homological algebra}%
\label{sec:some_normed_homological_algebra}

It will be convenient to use the following definition genearalizing the notion
of a bound on the norm of a right inverse of a normed group morphism (note that
the morphisms we consider have no reason to have a right inverse, even when
they are surjective).

\begin{definition}
  \label{def:surjective_with_bound}
  Let $G$ and $H$ be semi-normed groups, let $K$ be a subgroup of $H$ and $C$ be
  a positive real number.
  A morphism $f : G → H$ is $C$-surjective onto $K$ if, for all $x$ in $K$,
  there exists some $g$ in $G$ such that $f(g) = x$ and $\|g\| ≤ C\|x\|$.
  If $K = H$ we simply say $f$ is $C$-surjective.
\end{definition}

The following controlled surjectivity lemma will be used to prove
Proposition~\ref{prop:completeexact} and Lemma~\ref{lem:Tinv}.

\begin{lemma}
  \label{lem:closure_surjective}
  \uses{def:surjective_with_bound}
  \lean{controlled_closure_of_complete}\leanok
  Let $G$ and $H$ be normed groups. Let $K$ be a subgroup of
  $H$ and $f$ a morphism from $G$ to $H$.  Assume that $G$ is complete and
  $f$ is $C$-surjective onto $K$. Then $f$ is $(C + ε)$-surjective onto
  the topological closure of $K$ for every positive $ε$.
\end{lemma}

\begin{proof}\leanok
  Let $x$ be any element of the closure of $K$.
  First note the conclusion is trivial when $x = 0$, so we can assume
  $x ≠ 0$. Then write $x$ as a sum
  $\sum_{i \ge 0} x_i$ with all $x_i \in K$, $\|x - x_0\| ≤ ε_0$ and
  $‖x_i‖\leq \epsilon_i$ for $i>0$ for some sequence of positive numbers
  $\epsilon_i$ to be chosen later.
  By assumption, we can then lift each $x_i$ to $g_i$ such that
  $f(g_i) = x_i$ and $‖g_i‖\leq C‖x_i‖$, and then set
  $g = \sum g_i$. Because $G$ is complete,
  this sum converges provided the $ε_i$ sequence converges fast enough to zero.
  We then have $f(g) = x$ and
  \[
    ‖g‖ ≤ C\sum_{i \geq 0} ‖x_i‖ ≤
    C(\|x\| + ε_0) + C\sum_{i>0} ε_i ≤
    (C + ε)‖x‖
  \]
  where the last inequality holds provided the $ε_i$ sequence converges fast
  enough to zero. For instance $ε_i = ε∥x∥/(2^{i+1}C)$ satisfies all
  our constraints on the $ε_i$ sequence (in particular they are positive
  because $x ≠ 0$).
\end{proof}

We often use the following quantitative exactness property:

\begin{proposition}\label{prop:completeexact}
Let $M_0\xrightarrow{d_0} M_1\xrightarrow{d_1} M_2\xrightarrow{d_2} M_3$ be a
four-term complex of bounded maps of normed abelian groups. Assume that, for
some positive constants $C$ and $D$, $d_0$ is $C$-surjective onto $\ker d_1$
and $d_1$ is $D$-surjective onto $\ker d_2$.

Then
$\widehat{M}_0\xrightarrow{\widehat{d}_0} \widehat{M}_1\xrightarrow{\widehat{d}_1} \widehat{M}_2\xrightarrow{\widehat{d}_2} \widehat{M}_3$
is a complex and, for every positive $ε$, $\widehat{d}_0$ is $(C + ε)$-surjective
onto $\ker \widehat{d}_1$.
\end{proposition}

While the above statement in terms of complexes is what we will use,
the heart of the statement is the following lemma.

\begin{lemma}\label{lem:controlled_exactness}
  \lean{controlled_exactness}\leanok
  Let $f : M_0 → M_1$ and $g : M_1 → M_2$ be bounded maps between normed groups.
  Assume there are positive constants $C$ and $D$ such that:
  \begin{itemize}
    \item
      $f$ is $C$-surjective onto $\ker g$.
    \item
      $g$ is $D$-surjective onto its image.
  \end{itemize}
  Then for every positive $ε$, $\widehat{f}$ is $(C + ε)$-surjective onto
  $\ker \widehat{g}$.
\end{lemma}

\begin{proof}\leanok
  Since $f$ is $C$-surjective onto $\ker g$, $\widehat{f}$ is $C$-surjective
  onto $\ker g$ seen as a subset of $\widehat{M_1}$. Hence this lemma will
  follow directly from Lemma~\ref{lem:closure_surjective}
  once we'll have proven that $\ker g$ is dense in $\ker \widehat{g}$.
  Let $\widehat y$ be an element of $\ker \widehat{g}$.
  Pick any $\delta > 0$ and take $y\in M_1$ such that
  $‖\widehat{y}-y‖\leq \delta$. Let $z=g(y)\in M_2$, which has norm
  $‖z‖=‖g(y)‖=‖g(y-\widehat{y})‖$ bounded by
  $C_{g}\delta$, where $C_{g}$ is the norm of $g$. We can thus find some
  $y'\in M_1$ with $‖y'‖\leq DC_{g}\delta$ and $g(y')=z$. Replacing $y$ by
  $y-y'$, we can thus find $y\in \ker(g: M_1\to M_2)$ such that still
  $‖\widehat{y}-y‖\leq (1+DC_{g})\delta$; as $\delta$ was arbitrary, this
  gives the desired density.
\end{proof}

\begin{proof}[Proof of Proposition~\ref{prop:completeexact}]
  \proves{prop:completeexact}
  \uses{lem:controlled_exactness}
  The complex equations $\widehat d_1 ∘ \widehat d_0 = 0$ and
  $\widehat d_2 ∘ \widehat d_1 = 0$ follow from $d_1 ∘ d_0 = 0$ and
  $d_2 ∘ d_1 = 0$ by extension of identities.

  In order to get the quantitative estimates, we apply
  Lemma~\ref{lem:controlled_exactness} to $f = d_0$ and $g = d_1$. By exactness
  at $M_2$, the assumption we made on elements of $\ker d_2$ translates to the
  assumption made by the lemma on the image of $d_1$.
\end{proof}

\begin{definition}
  \label{system_of_complexes}
  \lean{system_of_complexes}
  \leanok
  A \emph{system of complexes} of normed abelian groups
  is for each sufficiently large $c$ (i.e.~all $c\geq c_0$ for some $c_0>0$),
  a complex
  \[
  C_c^\bullet: C_c^0\to C_c^1\to\ldots
  \]
  of normed abelian groups together with maps of complexes
  $\mathrm{res}_{c',c}: C_{c'}^\bullet\to C_c^\bullet$,
  for $c' ≥ c \geq c_0$,
  satisfying $\mathrm{res}_{c,c}=\mathrm{id}$ and the obvious associativity condition.
  We use notation $(C_c^\bullet)_{c\geq c_0}$ for a system of complexes,
  although we will frequently omit any mention of the lower bound $c_0$
  and just write $C_\bullet^\bullet$.
\end{definition}

By convention, for every system of complexes $(C_c^\bullet)_{c\geq c_0}$,
we will set $C^{-1}_c = 0$ for all $c\geq c_0$.
This will come up each time we write $C^{i-1}_c$ and $i$ could be $0$.

In this section, given $x ∈ C^•_{c'}$ and $c_0\leq c ≤ c'$ we will use the notation
$x_{|c} := \mathrm{res}_{c', c}(x)$.

\begin{definition}
  \label{admissible}
  \lean{system_of_complexes.admissible}
  \leanok
  \uses{system_of_complexes}
  A system of complexes is \emph{admissible}
  if all differentials and maps $\mathrm{res}_{c',c}^i$ are norm-nonincreasing.
\end{definition}

Throughout the rest of this section, $k$ (and $k'$, $k''$) will denote reals at least 1,
$m$ will be a non-negative integer, and $K$, $K'$, $K''$ will denote non-negative reals.

\begin{definition}
  \label{is_bounded_exact}
  \lean{system_of_complexes.is_bounded_exact}
  \leanok
  \uses{system_of_complexes}
  Let $(C_c^\bullet)_{c\geq c_0}$ be a system of complexes.
  For an integer $m\geq 0$ and reals $k \ge 1$, $c_0'\geq c_0$ and $K\geq0$,
  we say the datum $(C_c^\bullet)_{c\geq c_0}$ is
  \emph{$\leq k$-exact in degrees $\leq m$ and for $c\geq c_0'$ with bound $K$} if the following condition is satisfied.
  For all $c\geq c_0'$ and all $x\in C_{kc}^i$ with $i\leq m$
  there is some $y\in C_c^{i-1}$ such that
  \[
    ‖x_{|c} - dy‖ ≤ K ‖dx‖.
  \]
\end{definition}

We will also need a version where the inequality is relaxed by some arbitrary small additive constant.

\begin{definition}
  \label{is_weak_bounded_exact}
  \lean{system_of_complexes.is_weak_bounded_exact}
  \leanok
  \uses{system_of_complexes}
  Let $(C_c^\bullet)_{c\geq c_0}$ be a system of complexes.
  For an integer $m\geq 0$ and reals $k \ge 1$, $c_0'\geq c_0$ and $K\geq0$,
  the datum $(C_c^\bullet)_c$ is
  \emph{weakly $\leq k$-exact in degrees $\leq m$ and for $c\geq c_0'$ with bound $K$} if the following condition is satisfied.
  For all $c\geq c_0'$, all $x\in C_{kc}^i$ with $i\leq m$ and any $ε > 0$
  there is some $y\in C_c^{i-1}$ such that
  \[
    ‖x_{|c} - dy‖ ≤ K ‖dx‖ + ε.
  \]
\end{definition}

We first note that the difference between those two definitions is only about
cocyles if we are ready to lose a tiny something on the norm bound $K$.



\begin{lemma}
  \label{is_bounded_exact_of_weakly}
  \lean{system_of_complexes.is_weak_bounded_exact.to_exact}
  \uses{is_bounded_exact, is_weak_bounded_exact}
  Let $C_\bullet^\bullet$ be a system of complexes. If $C_\bullet^\bullet$ is
  weakly $\leq k$-exact in degrees $\leq m$ and for $c\geq c_0'$ with bound $K$ and if,
  for all $c\geq c_0'$ and all $x\in C_{kc}^i$ with $i\leq m$ such that $dx = 0$
  there is some $y\in C_c^{i-1}$  such that
  $x_{|c} = dy$ then, for every positive $δ$,
  $C_\bullet^\bullet$ is $\leq k$-exact in degrees $\leq m$ and for $c\geq c_0'$ with
  bound $K + δ$.
\end{lemma}

\begin{proof}
  \leanok
  Let $δ$ be some positive real number.
  Let $x$ be an element of $C_{kc}^i$ for some $c ≥ c_0'$ and $i ≤ m$. If $dx = 0$
  then the assumption we made about exact elements is exactly what we want.

  Assume now that $dx ≠ 0$. The weak exactness assumption applied to $ε = δ‖dx‖$
  gives some $y\in C_c^{i-1}$ such that
  \begin{align*}
    ‖x_{|c} - dy‖ &≤ K‖dx‖ + δ‖dx‖ \\
                  &= (K + δ)‖dx‖
  \end{align*}
\end{proof}

A more important observation is that, in both definitions, we can also ask some
control on the norm of $y$ if we are ready to square the restriction depth factor $k$.

\begin{lemma}
  \label{is_weak_bdd_exact_controlled_y}
  \uses{is_weak_bounded_exact}
  % This statement exists in some comment somewhere
  % \lean{system_of_complexes.is_weak_bounded_exact.controlled_y}
  Let $C_\bullet^\bullet$ be a system of complexes which is
  weakly $\leq k$-exact in degrees $\leq m$ and for $c\geq c_0'$ with bound $K$.
  For all $c\geq c_0'$, all $x\in C_{k²c}^i$ with $i\leq m$, all $ε > 0$ and all $δ > 0$
  there is some $y\in C_c^{i-1}$ such that
  \[
    ‖x_{|c} - dy‖ ≤  K ‖dx‖ + ε
    \quad \text{and} \quad
    ‖y‖ ≤ K(K + 1)‖x‖ + δ.
  \]
\end{lemma}

\begin{proof}
  Fix $x$, $ε$ and $δ$. The weak exactness assumption applied to $x$ and some $η$ to be chosen later gives us
  $w ∈ C_{kc}^{i-1}$ such that
  \[
    ‖x_{|kc} - dw‖ ≤  K ‖dx‖ + η.
  \]
  Then the weak exactness assumption applied to $w$ and some $τ$ to be chosen later gives us
  $z ∈ C_{c}^{i-2}$ such that
  \[
    ‖w_{|c} - dz‖ ≤  K ‖dw‖ + τ.
  \]
  We set $y = w_{|c} - dz$. Since $dy = dw_{|c}$, we get the first required estimate as long
  as $η ≤ ε$. And we have:
  \begin{align*}
    ‖y‖ &≤ K ‖dw‖ + τ \\
        &≤ K (‖x_{|kc}‖ + K‖dx‖ + η) + τ \\
        &≤ K(K + 1) ‖x‖ + Kη + τ
  \end{align*}
  which is fine as long as $Kη + τ ≤ δ$.
  So we set $η = \min(ε, δ/(2K))$ (interpreted as $ε$ if $K=0$) and $τ = δ/2$.
\end{proof}

\begin{lemma}
  \label{completion_is_weakexact}
  \uses{is_weak_bounded_exact, admissible}
  % This statement exists in some comment somewhere
  % \lean{system_of_complexes.is_weak_bounded_exact.completion}
  Let $(M_c^\bullet)_{c\geq c_0}$ be an admissible collection
  of complexes of normed abelian groups.

  Assume that $M^\bullet_c$ is weakly $\leq k$-exact in degrees $\leq m$ for $c\geq c_0$ with bound $K$.
  Then the completion $\overline{M^\bullet_c}$ is weakly $\leq k^2$-exact in degrees $\leq m$ for $c\geq c_0$ with bound $K$.
\end{lemma}

\begin{proof}
  \uses{is_weak_bdd_exact_controlled_y}
Let $x \in \overline{M^i_{k^2c}}$, where $c \geq c_0$ and $i \leq m$ and let $\epsilon > 0$.
We can write $x = \sum_j x^j$ where
\begin{itemize}
 \item $x^j \in M^i_{k^2c}$ for all $j \geq 0$,
 \item $‖x - x^0‖ ≤ ε_0$ for some positive $ε_0$ to be chosen later. This implies that $‖dx - dx^0‖ ≤ ε_0$ and
   in particular $‖dx^0‖ ≤ ‖dx‖ + ε_0$,
 \item $‖x^j‖ ≤ ε_j$ if $j > 0$, for some positive $ε_j$ to be chosen later. This implies $‖dx^j‖ ≤ ε_j$
   for all $j > 0$.
\end{itemize}

Using Lemma~\ref{is_weak_bdd_exact_controlled_y}, we get a sequence $y^j$ in $M^{i-1}_c$ such that
  \[
    ‖x^j_{|c} - dy^j‖ ≤  K ‖dx^j‖ + δ_j
    \quad \text{and} \quad
    ‖y^j‖ ≤ K(K + 1)‖x^j‖ + τ_j.
  \]
for positive sequences $δ$ and $τ$ to be chosen later.

Since $M^{i-1}_c$ is complete, the series $\sum y^j$ converges as soon as we can guarantee that
$\sum ‖y^j‖$ converges. Our estimates ensure this convergence as soon as the
sum of the $K(K + 1)ε_j + τ_j$ converges so here we only need $ε$ and $τ$ to be
summable.

We then set $y = ∑ y^j$ and compute:
\begin{align*}
  ‖x_{|c} - dy‖ &= \left\|∑_{j ≥ 0} x^j_{|c} - dy^j\right\| \\
    &≤ ∑_{j ≥ 0} \left\|x^j_{|c} - dy^j\right\| \\
    &≤ ∑_{j ≥ 0} K‖dx^j‖ + δ_j \\
    &≤ K‖dx‖ + Kε_0 + δ_0 +  ∑_{j > 0} (Kε_j + δ_j)
\end{align*}

So everything is fine as long as $∑_{j ≥ 0} (Kε_j + δ_j) ≤ ε$, say $ε_j = ε2^{-j-2}/K$ and $δ_j = ε2^{-j-2}$.
\end{proof}

\begin{lemma}
  \label{weakexact_implies_exact}
  \lean{system_of_complexes.is_weak_bounded_exact.strong_of_complete}
  \leanok
  \uses{is_weak_bounded_exact, is_bounded_exact, admissible}
  Let $(M^\bullet_c)_{c\geq c_0}$ be an admissible collection
  of complexes of complete normed abelian groups.

  Assume that $M^\bullet_c$ is weakly $\leq k$-exact in degrees $\leq m$ for $c\geq c_0$ with bound $K$.
  Then $M^\bullet_c$, for every $δ > 0$, it is $\leq k^2$-exact in degrees $\leq m$ for $c\geq c_0$
  with bound $K+δ$.
\end{lemma}

\begin{proof}
  \uses{is_bounded_exact_of_weakly}
  \leanok
  Lemma~\ref{is_bounded_exact_of_weakly} ensures we only need to care about cocycles
  of $M$. More precisely, let $x$ be a cocycle in $M^i_{k^2c}$ for some $i ≤ m$ and $c ≥ c_0$.
  We need to find $y \in M^{i-1}_c$ such that $dy = x_{|c}$.

  By weak $\leq k$-exactness applied to $x$ and a sequence $ε_j$ to be chosen later, we can find
  a sequence $w^j \in M^{i-1}_{kc}$ such that
  \[
    ‖x_{kc} - dw^j‖ ≤ ε_j.
  \]
  Then, by weak $\leq k$-exactness applied to each $w^{j + 1} - w^j$ and a sequence $δ_j$ to be chosen later, we can find
  a sequence $z^j \in M^{i-2}_{c}$ such that
  \[
    ‖(w^{j+1} - w^j)_{|c} - dz^j‖ ≤ K‖dw^{j+1} - dw^j‖ + δ_j.
  \]
  We set $y^j := w^j_{|c} - \sum_{l=0}^{j-1} dz^l ∈ M^{i-1}_c$.


  We have
  \begin{align*}
    ‖y^{j + 1} - y^j‖ &=  \left\|(w^{j + 1} - w^j)_{|c} - dz^j\right\| \\
                      &≤  K‖dw^{j+1} - dw^j‖ + δ_j \\
                      &≤  2Kε_j + δ_j.
  \end{align*}
  So $y^j$ is a Cauchy sequence as long as we make sure $2Kε_j + δ_j ≤ 2^{-j}$ for instance.
  Since $M^{i-1}_c$ is complete, this sequence converges to some $y$.
  Because $dy^j = dw^j_{|c}$, we get that $‖x_{|c} - dy^j‖ ≤ ε_j$ and in the limit $x_{|c} = dy$.
\end{proof}

\begin{proposition}
  \label{weaksnakelemma}
  \leanok
  \lean{weak_normed_snake}
  \uses{is_weak_bounded_exact, admissible}
  Let $(M^\bullet_c)_{c\geq c_0}$ and $(M'^\bullet_c)_{c\geq c_0}$ be two admissible collections
  of complexes of complete normed abelian groups.
  For each $c\geq c_0$ let $f^\bullet_c: M^\bullet_c\to M'^\bullet_c$ be a collection of maps
  between these collections of complexes
  that are norm-nonincreasing and which all commute with all restriction maps,
  and assume that there exists these maps satisfy
  \[
    ‖x_{|c}‖ ≤ K''‖f(x)‖
  \]
  for all $i ≤ m+1$ and all $x\in M^i_{k''c}$.
  Let $N^\bullet_c=M'^\bullet_c/M^\bullet_c$
  be the collection of quotient complexes, with the quotient norm;
  this is again an admissible collection of complexes.

  Assume that $M^\bullet_c$ (resp. $M'^\bullet_c$) is weakly $\leq k$-exact
  (resp. $≤ k'$-exact) in degrees $\leq m$ for $c\geq c_0$ with bound $K$
  (resp. $K'$).
  Then $N^\bullet_c$ is weakly $\leq kk'k''$-exact in degrees $\leq m-1$ for $c\geq c_0$
  with bound $K'(KK'' + 1)$.
\end{proposition}

\begin{proof}
\leanok
\def\ndn{\left\|dn\right\|}
Let $n \in N^i_{kk'k''c}$ for $i\leq m-1$.
We fix $ε > 0$. We need to find an element $y \in N^{i-1}_c$ such that
\[
  ‖n_{|c} - dy‖ \leq K'(KK'' + 1)‖dn‖ + \epsilon.
\]

Pick any preimage $m' \in M'^i_{kk'k''c}$ of $n$. In particular $dm'$ is
a preimage of $dn$.
By definition of the quotient norm,
we can find $m_1 ∈ M^{i+1}_{kk'k''c}$ and $m_1'' ∈ (M')^{i+1}_{kk'k''c}$ such that
\[
dm' = f(m_1) + m_1''
\]
with $‖m_1''‖ \leq ‖dn‖ + ε_1$, for some positive $ε_1$ to be chosen later.

Applying the differential to the last displayed equation, and using that this
kills the image of $d$, and that $f$ is a map of complexes, we see that
\[
f(dm_1) = -dm_1''.
\]
Using the norm bound on $f$, we get
\[\begin{aligned}
  ‖dm_{1|kk'c}‖ &≤ K''‖f(dm_1)‖ = K''‖dm_1''‖\\
                &≤ K''‖m_1''‖ ≤ K''‖dn‖ + K''ε_1.
\end{aligned}\]
On the other hand, weak exactness of $M$ applied to $m_{1|kk'c}$
gives $m_0 ∈ M^i_{k'c}$ such that
\[
  ‖m_{1|kk'c|k'c} - dm_0‖ \leq K‖dm_{1|kk'c}‖ + ε_1
\]
which combines with the previous estimate to give:
\[
  ‖m_{1|k'c} - dm_0‖ \leq K K'' \left\|d n\right\| + (KK'' + 1)ε_1.
\]
Now let $m'_{\mathrm{new}} = m'_{|k'c} - f(m_0) \in M'^i_{k'c}$; this is a lift of $n_{|k'c}$.
Then
\[
dm'_{\mathrm{new}} = dm'_{|k'c} - f(m_{1|k'c}) + f(m_{1|k'c} - dm_0) = m''_{1|k'c} + f(m_{1|k'c} - dm_0).
\]
In particular,
\[
‖dm'_{\mathrm{new}}‖ ≤ (KK'' + 1)\ndn + (KK'' + 2) ε_1.
\]
Now weak exactness of $M'$ gives $x \in M'^{i-1}_c$ such that
\[
  ‖m'_{\mathrm{new}|c} - dx‖ ≤ K'‖dm'_{\mathrm{new}}‖ + ε_1 \leq
    K'((K K'' + 1) \ndn + (KK'' + 2) ε_1) + ε_1.
\]
In particular, letting $y \in N^{i-1}_c$ be the image of $x$, we get
\[
  ‖n_{|c} - dy‖ ≤ K'(K K'' + 1)\ndn + (K'(K  K'' + 2) + 1) ε_1,
\]
which is exactly what we wanted if we choose
$ε_1 = ε/(K'(K  K'' + 2) + 1)$.
\end{proof}


\begin{proposition}
  \label{snakelemma}
  \uses{is_bounded_exact, admissible}
  Let $(M^\bullet_c)_{c\geq c_0}$ and $(M'^\bullet_c)_{c\geq c_0}$ be two admissible collections
  of complexes of complete normed abelian groups.
  For $c\geq c_0$ let $f^\bullet_c: M^\bullet_c\to M'^\bullet_c$ be a collection of maps
  between these collections of complexes
  that is strictly compatible with the norm and commutes with restriction maps,
  and assume that it satisfies
  \[
    ‖x_{|c}‖ ≤ K''‖f(x)‖
  \]
  for all $i ≤ m+1$ and all $x\in M^i_{k''c}$.
  Let $N^\bullet_c=M'^\bullet_c/\overline{M^\bullet_c}$
  be the collection of quotient complexes, with the quotient norm;
  this is again an admissible collection of complexes.

  Assume that $M^\bullet_c$ (resp. $M'^\bullet_c$) is $\leq k$-exact
  (resp. $≤ k'$-exact) in degrees $\leq m$ for $c\geq c_0$ with bound $K$
  (resp. $K'$).
  Then, for every $δ > 0$, $N^\bullet_c$ is $\leq (kk'k'')^2$-exact in
  degrees $\leq m-1$ for $c\geq c_0$
  with bound $K'(KK'' + 1) + δ$.
\end{proposition}

\begin{proof}
  \uses{weaksnakelemma, weakexact_implies_exact}
  The exactness assumptions on $M$ and $M'$ give the corresponding
  weak exactness condition. Hence Proposition~\ref{weaksnakelemma}
  ensures that $N^\bullet_c$ is weakly $\leq kk'k''$-exact in degrees
  $\leq m-1$ for $c\geq c_0$ with bound $K'(KK'' + 1)$.
  Since $N^\bullet_c$ is a complex of complete groups,
  Lemma~\ref{weakexact_implies_exact} gives the required exactness.
\end{proof}

Consider a system of double complexes $M^{p,q}_c$, $p,q\geq 0$, $c\geq c_0$,
\begin{center}
  \begin{tikzcd}
    M^{0,0}_c \ar[r]{d'^{0,0}_c}\ar[d]{d^{0,0}_c} & M^{0,1}_c\ar[r]{d'^{0,1}_c}\ar[d]{d^{0,1}_c} & M^{0,2}_c\ar[r]{d'^{0,2}_c}\ar[d]{d^{0,2}_c} & \ldots\\
M^{1,0}_c\ar[r]{d'^{1,0}_c}\ar[d]{d^{1,0}_c} & M^{1,1}_c\ar[r]{d'^{1,1}_c}\ar[d]{d^{1,1}_c} & M^{1,2}_c\ar[r]{d'^{1,2}_c}\ar[d]{d^{1,2}_c} & \ldots\\
M^{2,0}_c\ar[r]{d'^{2,0}_c}\ar[d]{d^{2,0}_c} & M^{2,1}_c\ar[r]{d'^{2,1}_c}\ar[d]{d^{2,1}_c} & \ddots\\
\vdots & \vdots
  \end{tikzcd}
\end{center}
of complete normed abelian groups.

\begin{definition}
  \label{spectral-htpy}
  \lean{system_of_double_complexes.normed_spectral_homotopy}
  \uses{system_of_complexes}
  \leanok
  We say that the system of double complexes $M^{p,q}_c$
  satisfies the \emph{normed spectral homotopy condition}
  for $m \in \N$ and $H, c_0 \in \R_{\ge 0}$
  if the following condition is satisfied:

  For $q=0,\ldots,m$ and $c\geq c_0$,
  there is a map $h^q_{k'c} \colon M^{0,q+1}_{k'c}\to M^{1,q}_c$ with
  \[
    \|h^q_{k'c}(x)\|_{M^{1,q}_c} \leq H\|x\|_{M^{0,q+1}_{k'c}}
  \]
  for all $x\in M^{0,q+1}_{k'c}$,
  and such that for all $c\geq c_0$ and $q=0,\ldots,m$ the ``homotopic'' map
  \[
    \mathrm{res}_{k'^2c,k'c}^{1,q}\circ d^{0,q} +
    h^q_{k'^2c}\circ d'^{0,q}_{k'^2c} +
    d'^{1,q-1}_{k'c}\circ h^{q-1}_{k'^2c} \colon
      M^{0,q}_{k'^2c}\to M^{1,q}_{k'c}
  \]
  factors as a composite of the restriction $\mathrm{res}_{k'^2c,c}^{0,q}$ and a map
  \[
    \delta^{0,q}_c \colon M^{0,q}_c\to M^{1,q}_{k'c}
  \]
  that is a map of complexes (in degrees $\leq m$), and satisfies the estimate
  \begin{equation}\label{eq:homotopicmapsmall}
  \|\delta^{0,q}_c(x)\|_{M^{1,q}_{k'c}}\leq \epsilon \|x\|_{M^{0,q}_c}
  \end{equation}
  for all $x\in M^{0,q}_c$.
\end{definition}

\begin{proposition}
  \label{spectral}
  \uses{admissible, is_weak_bounded_exact, spectral-htpy}
  \lean{system_of_double_complexes.normed_spectral}
  \leanok
  Fix an integer $m\geq 0$ and constants $k$, $K$.
  Then there exists an $\epsilon>0$ and constants $k_0$, $K_0$,
  depending (only) on $k$, $K$ and $m$, with the following property.

  Let $M^{p,q}_c$ be a system of double complexes as above,
  and assume that it is admissible.
  Assume further that there is some $k'\geq k_0$ and some $H>0$, such that
  \begin{enumerate}
	  \item for $i=0,\ldots,m+1$, the rows $M^{i,q}_c$ are weakly $\leq k$-exact in degrees $\leq m-1$ for $c\geq c_0$ with bound $K$;
	  \item for $j=0,\ldots,m$, the columns $M^{p,j}_c$ are weakly $\leq k$-exact in degrees $\leq m$ for $c\geq c_0$ with bound $K$;
    \item it satisfies the normed spectral homotopy condition for $m$, $H$ and $c_0$.
  \end{enumerate}
  Then the first row is weakly $\leq k'^2$ exact in degrees $\leq m$ for $c\geq c_0$ with bound $2K_0H$.
\end{proposition}

We note that the bound on the homotopy is of a peculiar nature, in that the bound only depends on a deep restriction of $x$.

\begin{proof}
  \proves{spectral}
  \uses{weaksnakelemma}
  \leanok
  First, we treat the case $m=0$.
  If $m=0$, we claim that one can take $\epsilon=\tfrac 1{2k}$ and $k_0=k$.
  We have to prove exactness at the first step.
  Let $x_{k'^2c}\in M^{0,0}_{k'^2c}$ and
  denote $x_{k'c}=\mathrm{res}_{k'^2c,k'c}^{0,0}(x)$
  and $x_c=\mathrm{res}_{k'^2c,c}^{0,0}(x)$.
  Then by assumption (2) (and $k'\geq k$), we have
  \[
  ‖x_c‖_{M^{0,0}_c}\leq k‖d^{0,0}_{k'c}(x_{k'c})‖_{M^{1,0}_{k'c}}.
  \]
  On the other hand, by (3),
  \[
  ‖\mathrm{res}_{k'^2c,k'c}^{1,0}(d^{0,0}_{k'^2c}(x))\pm h^0_{k'^2c}(d'^{0,0}_{k'^2c}(x))‖_{M^{1,0}_{k'c}}\leq \epsilon ‖x_c‖_{M^{0,0}_c}.
  \]
  In particular, noting that $\mathrm{res}_{k'^2c,k'c}^{1,0}(d^{0,0}_{k'^2c}(x)) = d^{0,0}_{k'c}(x_{k'c})$, we get
  \[
  ‖x_c‖_{M^{0,0}_c}\leq k‖d^{0,0}_{k'c}(x_{k'c})‖_{M^{1,0}_{k'c}}\leq k\epsilon ‖x_c‖_{M^{0,0}_c} + kH ‖d'^{0,0}_{k'^2c}(x)‖_{M^{0,1}_{k'^2c}}.
  \]
  Thus, taking $\epsilon=\tfrac 1{2k}$ as promised, this implies
  \[
  ‖x_c‖_{M^{0,0}_c}\leq 2kH ‖d'^{0,0}_{k'^2c}(x)‖_{M^{0,1}_{k'^2c}}.
  \]
  This gives the desired $\leq \max(k'^2,2k_0H)$-exactness in degrees $\leq m$ for $c\geq c_0$.

  Now we argue by induction on $m$.
  Consider the complex $N^{p,q}$ given by $M^{p,q+1}$ for $q\geq 1$
  and $N^{p,0} = M^{p,1}/\overline{M^{p,0}}$
  (the quotient by the closure of the image, which is also the completion of $M^{p,1}/M^{p,0}$),
  equipped with the quotient norm.
  Using the normed version of the snake lemma,
  Proposition~\ref{snakelemma} in the appendix to this lecture,
  one checks that this satisfies the assumptions for $m-1$,
  with $k$ replaced by $\max(k^4,k^3+k+1)$.
\end{proof}

% vim: ts=2 et sw=2 sts=2

\section{Completions of locally constant functions}

\begin{definition}
  \label{CLC}
  \lean{NormedGroup.LCC}
  \leanok
  Let $V$ be a semi-normed group, and $X$ a compact topological space.
  We denote by $V(X)$ the normed abelian group of locally constant functions $X \to V$
  with respect to the sup norm.
  With $\hat V(X)$ we denote the completion of $V(X)$.

  These constructions are functorial in bounded group homomorphisms $V \to V'$
  and contravariantly functorial in continuous maps $f \colon X \to X'$.

  Note in particular that $V(f)$ and $\hat V(f)$ are norm-nonincreasing
  morphisms of semi-normed groups.
\end{definition}

\begin{lemma}
  \label{CLC_normed_with_aut}
  \lean{NormedGroup.normed_with_aut_LCC}
  \leanok
  \uses{CLC, normed_with_aut}
  Let $r \in \mathbb R_{> 0}$,
  and let $V$ be an $r$-normed $\mathbb Z[T^{\pm 1}]$-module.
  Let $X$ be a compact space.
  Then $\hat V(X)$ is naturally an $r$-normed $\mathbb Z[T^{\pm 1}]$-module,
  with the action of $T$ given by post-composition.
\end{lemma}

\begin{proof}
  \leanok
  Formalised, but omitted from this text.
\end{proof}

We continue to use the notation of before:
let $r' > 0, c \ge 0$ be real numbers,
and let $M$ be a profinitely filtered pseudo-normed group with $r'$-action by $T^{-1}$.

\begin{lemma}
  \label{basic_eval_FP}
  \lean{breen_deligne.basic_universal_map.eval_FP}
  \lean{breen_deligne.basic_universal_map.eval_FP_comp}
  \leanok
  \uses{profinitely_filtered_pseudo_normed_group_with_Tinv,
    basic_universal_map}
  Let $f$ be a basic universal map from exponent~$m$ to~$n$.
  We get an induced homomorphism of
  profinitely filtered pseudo-normed groups $M^m \to M^n$
  bounded by the maximum (over all $i$) of $\sum_j |f_{ij}|$,
  where the $f_{ij}$ are the coefficients of the $n \times m$-matrix representing~$f$.

  This construction is functorial in~$f$.
\end{lemma}

\begin{proof}
  \leanok
  Omitted.
\end{proof}

\begin{definition}
  \label{basic_eval_LCFP}
  \lean{breen_deligne.basic_universal_map.eval_LCFP}
  \lean{breen_deligne.basic_universal_map.eval_LCFP_comp}
  \leanok
  \uses{basic_universal_suitable, basic_eval_FP}
  Let $f$ be a basic universal map from exponent~$m$ to~$n$,
  and let $(c_2, c_1)$ be $f$-suitable.
  We get an induced map
  \[
    V(f) \colon V(M_{\le c_1}^n) \to V(M_{\le c_2}^m)
  \]
  induced by the morphism of profinitely filtered pseudo-normed groups $M^m \to M^n$.

  This construction is functorial in~$f$.
\end{definition}

\begin{definition}
  \label{eval_LCFP}
  \lean{breen_deligne.universal_map.eval_LCFP}
  \lean{breen_deligne.universal_map.eval_LCFP_comp}
  \leanok
  \uses{universal_suitable, basic_eval_LCFP}
  Let $f = \sum_g n_g g$ be a universal map from exponent~$m$ to~$n$,
  and let $(c_2, c_1)$ be $f$-suitable.
  We get an induced map
  \[
    V(f) \colon V(M_{\le c_1}^n) \to V(M_{\le c_2}^m)
  \]
  that is the sum $\sum n_g V(g)$.

  This construction is functorial in~$f$.
\end{definition}

\begin{definition}
  \label{eval_CLCFP}
  \lean{breen_deligne.universal_map.eval_CLCFP}
  \leanok
  \uses{universal_suitable, eval_LCFP}
  Let $f$ be a universal map from exponent~$m$ to~$n$,
  and let $(c_2, c_1)$ be $f$-suitable.
  We get an induced map
  \[
    \hat V(f) \colon \hat V(M_{\le c_1}^n) \to \hat V(M_{\le c_2}^m)
  \]
  that is the completion of $V(f)$.

  This construction is functorial in~$f$.
\end{definition}

Let $r > 0$, and assume now that $V$ is an $r$-normed $\mathbb Z[T^{\pm 1}]$-module.
Assume $r' \le 1$.

\begin{definition}
  \label{CLCFPTinv}
  \lean{CLCFPTinv}
  \leanok
  \uses{CLC_normed_with_aut}
  There are two natural actions of $T^{-1}$ on $\hat V(M_{\le c})$.
  The first comes from the $r'$-action of $T^{-1}$ on $M$
  which gives a continuous map
  \[
    M_{\le cr'} \to M_{\le c}
  \]
  and thus a map
  \[
    (T^{-1})^* \colon \hat V(M_{\le c}) \to \hat V(M_{\le cr'}).
  \]
  The other comes from Lemma~\ref{CLC_normed_with_aut},
  using the $r$-normed $\mathbb Z[T^{\pm 1}]$-module $V$.
  We get a map
  \[
    [T^{-1}] \colon \hat V(M_{\le c}) \to \hat V(M_{\le c}),
  \]
  that we can compose with the map
  $\hat V(M_{\le c}) \to \hat V(M_{\le cr'})$,
  obtained from the natural inclusion $M_{\le cr'} \to M_{\le c}$.
  We thus end up with two maps
  \[
    (T^{-1})^*, [T^{-1}] \colon \hat V(M_{\le c}) \to \hat V(M_{\le cr'}).
  \]
  and we define $\hat V(M_{\le c})^{T^{-1}}$
  to be the equalizer of $(T^{-1})^*$ and $[T^{-1}]$.
  In other words, the kernel of $(T^{-1})^* - [T^{-1}]$.
\end{definition}

\begin{definition}
  \label{eval_CLCFPTinv}
  \lean{breen_deligne.universal_map.eval_CLCFPTinv}
  \leanok
  \uses{CLCFPTinv, eval_CLCFP}
  Let $f$ be a universal map from exponent~$m$ to~$n$,
  and let $(c_2, c_1)$ be $f$-suitable.

  The natural map from Definition~\ref{eval_CLCFP}
  restricts to a map
  \[
    \hat V(f)^{T^{-1}} \colon \hat V(M_{\le c_1}^n)^{T^{-1}} \to \hat V(M_{\le c_2}^m)^{T^{-1}}
  \]
\end{definition}

% vim: ts=2 et sw=2 sts=2

\subsection{Polyhedral lattices}
\label{sec:polyhedral_lattice}

\begin{definition}
  \label{polyhedral_lattice}
  \lean{polyhedral_lattice}
  \leanok
  A \emph{polyhedral lattice} is a finite free abelian group~$\Lambda$
  equipped with a norm $‖\cdot‖_\Lambda \colon \Lambda\otimes \mathbb R\to \mathbb R$
  such that there exists a finite set $\{\lambda_1, \dots, \lambda_n\} \subset \Lambda$
  that generate the norm:
  that is to say, for every $\lambda \in \Lambda$ there exist
  $c_1, \dots, c_n \in \mathbb Q$ such that
  $\lambda = \sum c_i \lambda_i$ and $\|\lambda\| = c_i\|\lambda_i\|$.

  Equivalently (but not verified in Lean):
  the norm is given by the supremum of finitely many linear functions on $\Lambda$;
  or once more,
  equivalently, the ``unit ball''
  $\{\lambda\in \Lambda\otimes \mathbb R\mid ‖\lambda‖_\Lambda\leq 1\}$ is a polyhedron.
\end{definition}

Finally, we can prove the key combinatorial lemma,
ensuring that any element of $\Hom(\Lambda,\Lbar_{r'}(S))$
can be decomposed into $N$ elements whose norm is roughly $\tfrac 1N$ of the original element.

\begin{definition}
  \label{splittable}
  \uses{pseudo_normed_group}
  \lean{pseudo_normed_group.splittable}
  \leanok
  Let $M$ be a pseudo-normed group, $N \in \mathbb N$, and $d \in \mathbb R_{\ge 0}$.
  We say that $M$ is \emph{$N$-splittable} with error term~$d$,
  if for all $c$ and $x \in M_c$,
  there exists a decomposition
  \[
    x = x_1 + x_2 + \dots + x_N,
  \]
  with $x_i \in M_{c/N + d}$.
\end{definition}

\begin{proposition}
  \label{Lbar-splittable}
  \lean{lem98.main}
  \leanok
  \uses{polyhedral_lattice}
  Let $\Lambda$ be a polyhedral lattice, and $S$ a profinite set.
  Then for all positive integers $N$ there is a constant $d$
  such that for all $c>0$ one can write any
  $x\in \Hom(\Lambda,\Lbar_{r'}(S))_{\leq c}$ as
  \[
    x=x_1+\ldots+x_N
  \]
  where all $x_i\in \Hom(\Lambda,\Lbar_{r'}(S))_{\leq c/N+d}$.

  In other words, for all $N$, there exists a $d$ such that
  $\Hom(\Lambda, \Lbar_{r'}(S))$ is $N$-splittable with error term~$d$.
\end{proposition}

\begin{proof}
  \uses{combi}
  \leanok
  The desired statement is equivalent to the surjectivity of the map of profinite sets
  \[
    \Hom(\Lambda,\Lbar_{r'}(S))_{\leq c/N+d}^N\times_{\Hom(\Lambda,\Lbar_{r'}(S))_{\leq c+Nd}} \Hom(\Lambda,\Lbar_{r'}(S))_{\leq c} \to
    \Hom(\Lambda,\Lbar_{r'}(S))_{\leq c}.
  \]
  Note that, as a functor of $S$, both sides commute with cofiltered limits, so it is enough to handle finite $S$, by Tychonoff.
  But that is exactly the following Lemma~\ref{combi}.
\end{proof}

\begin{lemma}
  \label{combi}
  \lean{lem98_finite}
  \leanok
  \uses{polyhedral_lattice}
  Let $\Lambda$ be a polyhedral lattice, and $S$ a finite set.
  Then for all positive integers $N$ there is a constant $d$
  such that for all $c>0$ one can write any
  $x\in \Hom(\Lambda,\Lbar_{r'}(S))_{\leq c}$ as
  \[
    x=x_1+\ldots+x_N
  \]
  where all $x_i\in \Hom(\Lambda,\Lbar_{r'}(S))_{\leq c/N+d}$.

  In other words, for all $N$, there exists a $d$ such that
  $\Hom(\Lambda, \Lbar_{r'}(S))$ is $N$-splittable with error term~$d$.
\end{lemma}

As preparation for the proof, we have the following results.

\begin{lemma}[Gordan's lemma]
  \label{explicit_gordan}
  \lean{explicit_gordan}
  \leanok
  Let $\Lambda$ be a finite free abelian group,
  and let $\lambda_1, \ldots, \lambda_m \in \Lambda$ be elements.
  Let $M \subset \Hom(\Lambda, \mathbb Z)$ be the submonoid
  \(\{x \mid x(\lambda_i) \ge 0 \text{ for all \(i = 1, \dots, m\)}\}\).
  Then $M$ is finitely generated as monoid.
\end{lemma}

\begin{proof}
  \leanok
  This is a standard result. We omit the proof here. It is done in Lean.
\end{proof}

\begin{lemma}
  \label{combi_aux}
  \lean{lem97}
  \leanok
  Let $\Lambda$ be a finite free abelian group,
  let $N$ be a positive integer,
  and let $\lambda_1,\ldots,\lambda_m\in \Lambda$ be elements.
  Then there is a finite subset $A\subset \Lambda^\vee$
  such that for all $x\in \Lambda^\vee=\Hom(\Lambda,\mathbb Z)$
  there is some $x'\in A$ such that $x-x'\in N\Lambda^\vee$
  and for all $i=1,\ldots,m$,
  the numbers $x'(\lambda_i)$ and $(x-x')(\lambda_i)$ have the same sign,
  i.e.~are both nonnegative or both nonpositive.
\end{lemma}

\begin{proof}
  \uses{explicit_gordan}
  \leanok
  It suffices to prove the statement for all $x$ such that $\lambda_i(x)\geq 0$ for all $i$;
  indeed, applying this variant to all $\pm \lambda_i$, one gets the full statement.

  Thus, consider the submonoid $\Lambda^\vee_+\subset \Lambda^\vee$
  of all $x$ that pair nonnegatively with all $\lambda_i$.
  This is a finitely generated monoid by Lemma~\ref{explicit_gordan};
  let $y_1,\ldots,y_M$ be a set of generators.
  Then we can take for $A$ all sums $n_1y_1+\ldots+n_My_M$ where all $n_j\in \{0,\ldots,N-1\}$.
\end{proof}

\begin{lemma}
  \label{exists_partition}
  \lean{combinatorial_lemma.exists_partition}
  \leanok
  Let $x_0, x_1, \dots$ be a sequence of reals,
  and assume that $\sum_{i=0}^\infty x_i$ converges absolutely.
  For every natural number $N > 0$,
  there exists a partition $\mathbb N = A_1 \sqcup A_2 \sqcup \dots \sqcup A_N$
  such that for each $j = 1,\dots,N$ we have
  $\sum_{i \in A_j} x_i \le (\sum_{i=0}^\infty x_i)/N + 1$
\end{lemma}

\begin{proof}
  \leanok
  Define the $A_j$ recursively:
  assume that the natural numbers $0, \dots, n$
  have been placed into the sets $A_1, \dots, A_N$.
  Then add the number $n+1$ to the set $A_j$ for which
  \[
    \sum_{i=0, i\in A_j}^n x_i
  \]
  is minimal.
\end{proof}

\begin{lemma}
  \label{lem98_int}
  \lean{lem98_int}
  \leanok
  For all natural numbers $N > 0$,
  and for all $x\in \Lbar_{r'}(S)_{\leq c}$
  one can decompose $x$ as a sum
  \[
    x=x_1+\ldots+x_N
  \]
  with all $x_i\in \Lbar_{r'}(S)_{\leq c/N+1}$.
\end{lemma}

\begin{proof}
  \leanok
  \uses{exists_partition}
  Choose a bijection $S \times \mathbb N \cong \mathbb N$,
  and transport the result from Lemma~\ref{exists_partition}.
\end{proof}

\begin{proof}[{Proof of Lemma~\ref{combi}}]
  \proves{combi}
  \uses{combi_aux, lem98_int}
  \leanok
  Pick $\lambda_1,\ldots,\lambda_m\in \Lambda$ generating the norm. We fix a finite subset $A\subset \Lambda^\vee$ satisfying the conclusion of the previous lemma. Write
  \[
  x=\sum_{n\geq 1, s\in S} x_{n,s} T^n [s]
  \]
  with $x_{n,s}\in \Lambda^\vee$. Then we can decompose
  \[
  x_{n,s} = N x_{n,s}^0 + x_{n,s}^1
  \]
  where $x_{n,s}^1\in A$ and we have the same-sign property of the last lemma. Letting $x^0 = \sum_{n\geq 1, s\in S} x_{n,s}^0 T^n [s]$, we get a decomposition
  \[
  x = Nx^0 + \sum_{a\in A} a x_a
  \]
  with $x_a\in \Lbar_{r'}(S)$ (with the property that in the
  basis given by the $T^n [s]$, all coefficients are $0$ or $1$). Crucially,
  we know that for all $i=1,\ldots,m$, we have
  \[
  ‖x(\lambda_i)‖ = N ‖x^0(\lambda_i)‖ + \sum_{a\in A} |a(\lambda_i)| ‖x_a‖
  \]
  by using the same sign property of the decomposition.

  Using this decomposition of $x$, we decompose each term into $N$ summands.
  This is trivial for the first term $Nx^0$,
  and each summand of the second term decomposes with $d = 1$ by Lemma~\ref{lem98_int}.
  (It follows that in general one can take for $d$
  the supremum over all $i$ of $\sum_{a\in A} |a(\lambda_i)|$.)
\end{proof}

\begin{definition}
  \label{rescaled-sum}
  \uses{polyhedral_lattice}
  \leanok
  \lean{rescale.polyhedral_lattice, finsupp.polyhedral_lattice}
  Let $\Lambda$ be a polyhedral lattice, and let $N > 0$ be a natural number.
  (We think of $N$ as being fixed once and for all,
  and thus it does not show up in the notation below.)

  By $\Lambda'$ we denote $\Lambda^N$ endowed with the norm
  \[
	  ‖(\lambda_1,\ldots,\lambda_N)‖_{\Lambda'} = \tfrac 1N(‖\lambda_1‖_\Lambda+\ldots+‖\lambda_N‖_\Lambda).
  \]
  This is a polyhedral lattice.
\end{definition}

\begin{lemma}
  \label{polyhedral-quotient}
  \uses{rescaled-sum}
  \lean{polyhedral_lattice.conerve.obj.polyhedral_lattice}
  \leanok
  For any $m\geq 1$, let $\Lambda'^{(m)}$ be given by $\Lambda'^m / \Lambda\otimes (\mathbb Z^m)_{\sum=0}$;
  for $m=0$, we set $\Lambda'^{(0)} = \Lambda$.
  Then $\Lambda'^{(m)}$ is a polyhedral lattice.
\end{lemma}

\begin{proof}
  \leanok
  The proof is done in Lean.
  TODO: write down a proof here.
\end{proof}

\begin{definition}
  \label{cosimplicial-lattice}
  \uses{polyhedral-quotient}
  \lean{PolyhedralLattice.cosimplicial}
  \leanok
  For any $m\geq 1$, let $\Lambda'^{(m)}$ be given by $\Lambda'^m / \Lambda\otimes (\mathbb Z^m)_{\sum=0}$;
  for $m=0$, we set $\Lambda'^{(0)} = \Lambda$.
  Then $\Lambda'^{(\bullet)}$ is a cosimplicial polyhedral lattice,
  the \v{C}ech conerve of $\Lambda\to \Lambda'$.

  In particular, $\Lambda'^{(0)} = \Lambda \to \Lambda' = \Lambda'^{(1)}$
  is the diagonal embedding.
\end{definition}

\begin{definition}
  \label{Hom}
  \uses{polyhedral_lattice, chpng-Tinv}
  \lean{polyhedral_lattice.add_monoid_hom.profinitely_filtered_pseudo_normed_group_with_Tinv}
  \leanok
  Let $\Lambda$ be a polyhedral lattice,
  and $M$ a profinitely filtered pseudo-normed group.

  Endow $\Hom(\Lambda, M)$ with the subspaces
  \[
    \Hom(\Lambda, M)_{\leq c} =
    \{f \colon \Lambda \to M \mid
      \forall x \in \Lambda, f(x) \in M_{\leq c‖x‖} \}.
  \]
  As $\Lambda$ is polyhedral, it is enough to check the given condition on~$f$
  for a finite collection of $x$ that generate the norm.

  These subspaces are profinite subspaces of $M^\Lambda$,
  and thus they make $\Hom(\Lambda, M)$ ito a profinitely filtered pseudo-normed group.

  If $M$ has an action of $T^{-1}$, then so does $\Hom(\Lambda, M)$.
\end{definition}

% vim: ts=2 et sw=2 sts=2


\subsection{Key technical result}

Now we state the following result, which is the key technical result on our to the main goal.

\begin{theorem}
  \label{first_target}\alsoIn{cha:two}
  \lean{thm94}
  \leanok
  \uses{BD_system}
  Let $\BD = (n,f,h)$ be a Breen--Deligne package,
  and let $\kappa = (\kappa_0, \kappa_1, \kappa_2, \dots)$ be a sequence of constants in $\mathbb R_{\ge 0}$
  that is $\BD$-suitable.
	Fix radii $1>r'>r>0$.
  For any $m$ there is some $k$ and $c_0$ such that for all profinite sets $S$ and all $r$-normed $\mathbb Z[T^{\pm 1}]$-modules $V$,
  the system of complexes
  \[
    C^{\BD}_{\kappa}(\Lbar_{r'}(S))_c^\bullet \colon
    \widehat{V}(\Lbar_{r'}(S)_{\leq c})^{T^{-1}} \to
    \widehat{V}(\Lbar_{r'}(S)_{\leq \kappa_1c}^2)^{T^{-1}}
    \to \ldots
  \]
  is $\leq k$-exact in degrees $\leq m$ for $c\geq c_0$.
\end{theorem}

We will prove Theorem~\ref{first_target} by induction on $m$.
Unfortunately, the induction requires us to prove a stronger statement.

\begin{theorem}
  \label{explicit}
  \uses{Hom, BD_system, Lbar_with_Tinv}
  \lean{thm95''.profinite}
  \leanok
  Fix radii $1>r'>r>0$. For any $m$ there is some $k$
  such that for all polyhedral lattices $\Lambda$
  there is a constant $c_0(\Lambda)>0$
  such that for all profinite sets $S$
  and all $r$-normed $\mathbb Z[T^{\pm 1}]$-modules $V$,
  the system of complexes
  \[
  C_{\Lambda,c}^\bullet \colon
  \widehat{V}(\Hom(\Lambda,\Lbar_{r'}(S))_{\leq c})^{T^{-1}} \to
  \widehat{V}(\Hom(\Lambda,\Lbar_{r'}(S))_{\leq \kappa_1c}^2)^{T^{-1}} \to \ldots
  \]
  is $\leq k$-exact in degrees $\leq m$ for $c\geq c_0(\Lambda)$.
\end{theorem}

\begin{proof}[{Proof of Theorem~\ref{first_target}}]
  \proves{first_target}
  \uses{explicit}
  \leanok
  Use $\Lambda = \mathbb Z$, and the isomorphism $\Hom(\mathbb Z, A) \cong A$.
\end{proof}

\textbf{A word on universal constants}:
We fix once and for all, the constants $0 < r < r' \le 1$
a Breen--Deligne package $\BD$,
and a sequence of positive constants $\kappa$ that is very suitable for $(\BD, r, r')$.
Once the full proof is formalized,
we can come back to this place and write a bit more about the other constants.

\textbf{The global strategy}
of the proof is to construct a system of double complexes
such that its first row is the system $C_{\Lambda, \bullet}^\bullet$
occuring in Theorem~\ref{explicit}.
We can then verify the conditions to Proposition~\ref{spectral}
and conclude from there.
For the time being, we will let $M$ denote
an arbitrary profinitely filtered pseudo-normed group with action of $T^{-1}$,
and whenever needed we can specialize to $M = \Lbar_{r'}(S)$.

\textbf{Further choices of constants}:
We will argue by induction on $m$, so assume the result for $m-1$
(this is no assumption for $m=0$, so we do not need an induction start).
This gives us some $k>1$ for which the statement of Theorem~\ref{explicit} holds true for $m-1$;
if $m=0$, simply take any $k>1$.
In the proof below, we will increase $k$ further in a way that depends only on $m$ and $r$.
After this modified choice of $k$, we fix $\epsilon$ and $k_0$ as provided by Proposition~\ref{spectral}.
Fix a sequence $(\kappa'_i)_i$ of nonnegative reals that is adept to $(\BD, \kappa)$.
(Such a sequence exists by Lemma~\ref{exists_adept}.)
Moreover, we let $k'$ be the supremum of $k_0$ and the $c_i'$ for $i=0,\ldots,m+1$.
Finally, choose a positive integer $b$ so that $2k'(\tfrac r{r'})^b\leq \epsilon$,
and let $N$ be the minimal power of $2$ that satisfies
\[
  k'/N\leq (r')^b.
\]
Then in particular $r^bN\leq 2k'(\tfrac{r}{r'})^b\leq \epsilon$.

\begin{definition}
  \label{double_complex}
  \lean{thm95.double_complex}
  \uses{Hom, cosimplicial-lattice, BD_system}
  \leanok
  Let $\Lambda^{(\bullet)}$ be the cosimplicial polyhedral lattice of
  Definition~\ref{cosimplicial-lattice},
  and recall from \ref{Hom} that $\Hom(\Lambda^{(m)}, M)$ is a
  profinitely filtered pseudo-normed group with action of $T^{-1}$.

  Hence $\Hom(\Lambda^{(\bullet)}, M)$ is a simplicial
  profinitely filtered pseudo-normed group with action of $T^{-1}$.

  Now apply the construction of the system of complexes from
  Definition~\ref{BD_system} to obtain a cosimplicial system of complexes
  \[
    C^{\BD}_{\kappa}(\Hom(\Lambda^{(\bullet)}, M))_\bullet^\bullet.
  \]
  Now take the alternating face map cochain complex
  to obtain a system of double complexes, whose objects are
  \[
    \widehat{V}(\Hom(\Lambda^{(m)},M)_{\leq \kappa_ic}^{n_i})^{T^{-1}}.
  \]
  As final step, rescale the norm on the object in row $m$ by $m!$,
  so that all columns become admissible:
  the vertical differential from row $m$ to row $m+1$
  is an alternating sum of $m+1$ maps that are all norm-nonincreasing.
\end{definition}

\begin{lemma}
  \label{canonical_iso_1}
  \uses{double_complex}
  \lean{PolyhedralLattice.Hom_cosimplicial_zero_iso}
  \leanok
  In particular, for any $c>0$, we have
  \[
    \Hom(\Lambda',M)_{\leq c} = \Hom(\Lambda,M)_{\leq c/N}^N,
  \]
  with the map to $\Hom(\Lambda,M)_{\leq c}$ given by the sum map.
\end{lemma}

\begin{proof}
  \leanok
  Omitted (but done in Lean).
\end{proof}

\begin{lemma}
  \label{canonical_iso_2}
  \uses{double_complex}
  \lean{thm95.Cech_nerve_level_iso}\leanok
  Similarly, for any $c>0$, we have
  \[
    \Hom(\Lambda'^{(m)},M)_{\leq c} =
    \Hom(\Lambda',M)_{\leq c}^{m/\Hom(\Lambda,M)_{\leq c}},
  \]
  the $m$-fold fibre product of $\Hom(\Lambda',M)_{\leq c}$
  over $\Hom(\Lambda,M)_{\leq c}$.
\end{lemma}

\begin{proof}
  \leanok
  Omitted (but done in Lean).
\end{proof}

\begin{lemma}
  \label{row_one_iso}
  \uses{canonical_iso_1}
  \lean{thm95.mul_rescale_iso_row_one}
  \leanok
  There is a canonical isomorphism between the first row of the double complex
  \[
    C^{\BD}_{\kappa}(\Hom(\Lambda^{(1)}, M))^\bullet
  \]
  and
  \[
    C^{N \otimes \BD}_{\kappa/N}(\Hom(\Lambda, M))^\bullet
  \]
  which identifies the map induced by
  the diagonal embedding $\Lambda \to \Lambda' = \Lambda^{(1)}$
  with the map induced by $\sigma^N \colon N \otimes \BD \to \BD$.
\end{lemma}

\begin{proof}
  \leanok
  Omitted (but done in Lean).
\end{proof}

\begin{proposition}
  \label{cechcover-exact}
  \uses{CLC}
  \lean{prop819}
  \leanok
  Let $\pi : X \to B$ be a surjective morphism of profinite sets,
  and let $S_\bullet \to S_{-1}$, $S_{-1} := B$, be its augmented \v{C}ech nerve.
  Let $V$ be a semi-normed group.
  Then the complex
  \[
    0\to \widehat{V}(S_{-1})\to \widehat{V}(S_0)\to \widehat{V}(S_1)\to \cdots
  \]
  is exact.
  Furthermore, for all $\epsilon > 0$ and $f \in \ker(\widehat{V}(S_{m}) \to \widehat{V}(S_{m+1}))$,
  there exists some $g\in \widehat{V}(S_{m-1})$ such that $d(g) = f$ and $‖g‖\leq (1+\epsilon) \cdot ‖f‖$.
  In other words, the complex is normed exact in the sense of Definition~\ref{norm_exact_complex}.
\end{proposition}

\begin{proof}
  \uses{lem:controlled_exactness}\leanok
  We argue similarly to~\cite[Theorem 3.3]{Condensed}, as follows.
  By applying Lemma~\ref{lem:controlled_exactness}, we first reduce to a statement which does not involve $\epsilon$ or completions.
  Explicitly, we must show that
  \[ 0 \to V(S_{1}) \to V(S_{0}) \to V(S_{1}) \to \cdots \]
  is exact, and that whenever $f \in \ker(V(S_{m}) \to V(S_{m+1}))$, there exists $g \in V(S_{m-1})$ such that $‖g‖\leq ‖f‖$ and $d(g) = f$.
  The map $V(S_{-1}) \to V(S_{0})$ is the one induced by $S_0 \to S_{-1}$ which agrees with $X \to B$.
  Since $X \to B$ is surjective, we easily see that $V(S_{-1}) \to V(S_{0})$ is injective.

  If $X$ and $B$ are finite, then the remaining assertions follow from the existence of a splitting $\sigma : B \to X$ of $\pi : X \to B$, as follows.
  The map $\sigma$ provides maps $S_{m} \to S_{m+1}$ for all $m \geq -1$, defined explicitly as
  \[ (a_{0},\ldots,a_{m}) \mapsto (\sigma (\pi (a_{0})), a_{0},\ldots,a_{m}) \]
  if $m \geq 0$ and simply as $\sigma$ if $m = -1$.
  Here, for $m \geq 0$, we have identified $S_{m}$ with the $m+1$-fold fibered product $X \times_{B} \cdots \times_{B} X$.
  Applying $V(-)$, these maps induce $h_{m} : V(S_{m+1}) \to V(S_{m})$, which form a contracting homotopy for the complex in question, and which are norm nonincreasing by the definition of $V(-)$.
  If $f \in \ker(V_{m} \to V_{m+1})$ is as above, then $g := h_{m}(f)$ satisfies $d(g) = f$ and $‖g‖\leq ‖f‖$, as required.

  In the general case, write $X = \varprojlim_{i} X_{i}$ where $X_{i}$ vary over the discrete (hence finite) quotients of $X$.
  Since $X \to B$ is surjective, for each $i$ there exists a unique maximal discrete quotient $B_{i}$ of $B$ such that $X \to B$ descends to $X_{i} \to B_{i}$.
  The maps $X_{i} \to B_{i}$ are again surjective, and one has
  \[ (X \to B) = \varprojlim_{i} (X_{i} \to B_{i}). \]
  Let $S_{i,\bullet} \to S_{i,-1}$, $S_{i,-1} := B_{i}$, denote the augmented \v{C}ech nerve of $X_{i} \to B_{i}$.

  The terms in the \v{C}ech nerve are themselves limits, hence we have $S_{m} = \varprojlim_{i} S_{i,m}$, with each $S_{i,m}$ finite.
  The functor $V(-)$, when considered as taking values in abelian groups, sends cofiltered limits to filtered colimits.
  Also, if $f \in V(S_{m})$ is the pullback of $f_{i} \in V(S_{i,m})$, then for a sufficiently small index $j \leq i$, the image of $f : S_{m} \to V$ agrees with the image of $f_{j} : S_{j,m} \to V$, where $f_{j}$ is the image of $f_{i}$ under the map $V(S_{i,m}) \to V(S_{j,m})$ induced by the transition map $S_{j,m} \to S_{i,m}$.

  Now suppose that $f \in \ker(V(S_{m}) \to V(S_{m+1}))$ is given.
  By the discussion above, there exists some $i$ and some $f_{i} \in V(S_{i,m})$ such that $f$ is the pullback of $f_{i}$ with respect to the morphism $S_{m} \to S_{i,m}$ and such that the following additional conditions hold:
  \begin{enumerate}
    \item One has $‖f_{i}‖ = ‖f‖$.
    \item One has $f_{i} \in \ker (V(S_{i,m}) \to V(S_{i,m+1}))$.
  \end{enumerate}
  Let $h_{m} : V(S_{i,m}) \to V(S_{i,m-1})$ be the map constructed in the argument for the finite case $X_{i} \to B_{i}$.
  Put $g_{i} := h_{m} (f_{i})$ and $g$ the image of $g_{i}$ in $V(S_{m-1})$.
  Since the maps $V(S_{i,\bullet}) \to V(S_{\bullet})$ commute with the differentials, we have $d(g) = f$.
  Finally, the map $V(S_{i,m-1}) \to V(S_{m-1})$ is norm nonincreasing as it is induced from $S_{m-1} \to S_{i,m-1}$, so that
  \[ ‖g‖\leq ‖g_{i}‖ \le ‖f_{i}‖ = ‖f‖, \]
  as contended.
\end{proof}

\begin{lemma}
  \label{pre_col_exact}
  \uses{splittable}
  \lean{thm95.FLC_complex.weak_bounded_exact}
  \leanok
  Let $M$ be a profinitely filtered pseudo-normed group with $T^{-1}$-action
  that is $N$-splittable with error term~$d \ge 0$.
  Let $k \ge 1$ be a real number,
  and let $c_0 > 0$ satisfy $d \le \frac{(k - 1) c_0}{N}$.
  For every $c$, consider the Cech nerve of the summation map $M^N_{c/N} \to M_c$.
  By applying the functor $\hat V(\_)$ and taking the alternating face map complex,
  we obtain a system of complexes
  \[
    \hat V(M_{\le c}) \to \hat V(M^N_{\le c/N}) \to \dots
  \]
  This system of complexes
  is weakly $\le k$-exact in degrees $\le m$ and for $c \ge c_0$ with bound~$1$.
\end{lemma}

\begin{proof}
  \uses{cechcover-exact, weak_exact_of_factor_exact, norm_exact_complex}
  \leanok
  For every constant $c$,
  consider the pullback
  \begin{center}
    \begin{tikzcd}
      & M_c \rar & M_{kc} \\
      & X_c \uar \rar & M^N_{kc/N} \uar \\
      M^N_{c/N} \ar{uur} \ar{rru} \ar[dotted]{ur}
    \end{tikzcd}
  \end{center}
  We therefore get morphisms of cochain complexes
  \begin{center}
    \begin{tikzcd}
      \hat V(M_{kc}) \rar\dar & \hat V(M_c) \rar\dar & \hat V(M_c) \dar \\
      \hat V(M^N_{kc/N}) \rar\dar & \hat V(X_c) \rar\dar & \hat V(M^N_{c/N}) \dar \\
      \vdots \rar & \vdots \rar & \vdots
    \end{tikzcd}
  \end{center}
  where all the columns are of the form
  ``alternating face map complex of $\hat V(\_)$ applied to a Cech nerve''.
  Note that the horizontal maps are norm-nonincreasing
  and their compositions are restriction maps.

  Claim: for $c \ge c_0$, the map $X_c \to M_c$ is surjective.

  Indeed, by assumption every $x \in M_c$ can be decomposed into a sum
  $x = x_1 + \dots + x_N$ with $x_i \in M_{c/N+d} \subset M_{kc/N}$,
  since $c \ge c_0$ and $d \le \frac{(k-1)c_0}{N}$.

  By Proposition~\ref{cechcover-exact},
  the middle column is normed exact (in the sence of Definition~\ref{norm_exact_complex}).
  The result follows from Lemma~\ref{weak_exact_of_factor_exact}.
\end{proof}

\begin{proposition}
  \label{col_exact}
  \lean{thm95.col_exact}
  \leanok
  Let $d$ be the constant from Proposition~\ref{Lbar-splittable}.
  Let $k > 1$ and $c_0 > 0$ be real numbers such that
  \[
    (k - 1) * c_0 / N \ge d.
  \]
  Let $m$ be any natural number, and put
  \[
    K = (m + 2) + \frac{r + 1}{r(1 - r)} (m + 2)^2
  \]
  Finally, let $c_0'$ be $\frac{c_0}{r' \cdot n_i}$,
  where $n_i$ is the $i$-th index in our fixed Breen--Deligne data.

  Then $i$-th column in the double complex
  is $(k^2, K)$-weak bounded exact in degrees $\le m$ for $c \ge c_0'$.
\end{proposition}

\begin{proof}\leanok
  \uses{combi,lem:Tinv,canonical_iso_2,weakdualsnakelemma, pre_col_exact}
  Let $M^{(m)}$ denote $\Hom(\Lambda^{(m)},\Lbar_{r'}(S))^{n_i}$.
  We also write $M$ for $M^{(0)} = \Hom(\Lambda,\Lbar_{r'}(S))^{n_i}$
  and $M'$ for $M^{(1)}$.
  By Proposition~\ref{combi}, $M$ is $N$-splittable with error term~$d$.

  Consider the diagram of morphisms of systems of complexes
  \begin{center}
    \begin{tikzcd}
      \hat V(M_c)^{T^{-1}} \rar\dar & \hat V(M_c) \rar["{T^{-1} - [T^{-1}]^*}"]\dar & \hat V(M_c) \dar \\
      \hat V(M'_c)^{T^{-1}} \rar\dar & \hat V(M'_c) \rar["{T^{-1} - [T^{-1}]^*}"]\dar & \hat V(M'_c) \dar \\
      \vdots \dar & \vdots \dar & \vdots \dar \\
      \hat V(M^{(m)}_c)^{T^{-1}} \rar & \hat V(M^{(m)}_c) \rar["{T^{-1} - [T^{-1}]^*}"] & \hat V(M^{(m)}_c)
    \end{tikzcd}
  \end{center}

  By Lemmas~\ref{pre_col_exact} and~\ref{canonical_iso_2},
  we know that the two columns on the right are
  weakly $\le k$-exact in degrees $\le m$ and for $c \ge c_0$ with bound~$1$.

  The result now follows from Lemma~\ref{lem:Tinv}, and Proposition~\ref{weakdualsnakelemma}.
\end{proof}

\begin{proposition}
  \label{double-complex-htpy}
  \lean{NSC_htpy}
  \uses{double_complex, BD_h_mul_suitable, spectral-htpy}
  \leanok
  Let $h$ be the homotopy packaged with $\BD$,
  and let $h^N$ denote the $n$-th iterated composition of $h$
  (see Def~\ref{BD_h_mul})
  which is a homotopy between
  $\pi^N$ and $\sigma^N \colon N \otimes \BD \to \BD$.

  Let $H \in \R_{\ge 0}$ be such that for $i = 0, \dots, m$
  the universal map $h^N_i$ is bound by $H$
  (see Def~\ref{universal_map_bound_by}).

  Then the double complex satisfies
  the normed homotopy homotopy condition (Def~\ref{spectral-htpy})
  for $m$, $H$, and $c_0$.
\end{proposition}

\begin{proof}
  \uses{exists_adept, BD_h_mul_suitable, row_one_iso}
  \leanok
  By Lemma~\ref{row_one_iso} we may replace the first row by
  \[
    C^{N \otimes \BD}_{\kappa/N}(\Hom(\Lambda, M))^\bullet.
  \]
  Now it is important to recall that we have chosen $k' \ge \kappa'_i$ for all $i = 0, \dots, m+1$.

  Our goal is to find,
  in degrees $\leq m$, a homotopy between the two maps from the first row
  \[
    \widehat{V}(\Hom(\Lambda,M)_{\leq c})^{T^{-1}}\to \widehat{V}(\Hom(\Lambda,M)_{\leq \kappa_1c}^2)^{T^{-1}}\to \ldots
  \]
  to the second row
  \[
    \widehat{V}(\Hom(\Lambda,M)_{\leq c/N}^N)^{T^{-1}}\to \widehat{V}(\Hom(\Lambda,M)_{\leq \kappa_1c/N}^{2N})^{T^{-1}}\to \ldots
  \]
  respectively induced by $\sigma^N$ and $\pi^N$ (which are maps $N \otimes \BD$

  By Definition~\ref{BD_h_mul} and Lemma~\ref{BD_h_mul_suitable}
  we can find this homotopy between the complex for $k'c$ and the complex for~$c$.
  (Here we use $k'\geq c_i'$ for $i=0,\ldots,m$.)
  By assumption, the norm of these maps is bounded by~$H$.

  Finally, it remains to establish the estimate (eq.~\ref{eq:homotopicmapsmall}) on the homotopic map.
  We note that this takes $x\in \widehat{V}(\Hom(\Lambda,M)_{\leq k'^2\kappa_ic}^{a_i})^{T^{-1}}$
  (with $i=q$ in the notation of (eq.~\ref{eq:homotopicmapsmall})) to the element
  \[
    y\in \widehat{V}(\Hom(\Lambda,M)_{\leq k'\kappa_ic/N}^{Na_i})^{T^{-1}}
  \]
  that is the sum of the $N$ pullbacks along the $N$ projection maps
  $\Hom(\Lambda,M)_{\leq k'\kappa_ic/N}^{Na_i}\to \Hom(\Lambda,M)_{\leq k'^2\kappa_ic}^{a_i}$.
  We note that these actually take image in $\Hom(\Lambda,M)_{\leq \kappa_ic}^{a_i}$ as $N\geq k'$,
  so this actually gives a well-defined map
  \[
    \widehat{V}(\Hom(\Lambda,M)_{\leq \kappa_ic}^{a_i})^{T^{-1}} \to
    \widehat{V}(\Hom(\Lambda,M)_{\leq k'\kappa_ic/N}^{Na_i})^{T^{-1}}.
  \]
  We need to see that this map is of norm $\leq \epsilon$.
  Now note that by our choice of $N$, we actually have $k'\kappa_ic/N\leq (r')^b \kappa_ic$,
  so this can be written as the composite of the restriction map
  \[
    \widehat{V}(\Hom(\Lambda,M)_{\leq \kappa_ic}^{a_i})^{T^{-1}} \to
    \widehat{V}(\Hom(\Lambda,M)_{\leq (r')^b \kappa_ic}^{a_i})^{T^{-1}}
  \]
  and
  \[
    \widehat{V}(\Hom(\Lambda,M)_{\leq (r')^b \kappa_ic}^{a_i})^{T^{-1}} \to
    \widehat{V}(\Hom(\Lambda,M)_{\leq k'\kappa_ic/N}^{Na_i})^{T^{-1}}.
  \]
  The first map has norm exactly $r^b$, by $T^{-1}$-invariance,
  and as multiplication by $T$ scales the norm with a factor of $r$ on $\widehat{V}$.
  (Here is where we use $r'>r$, ensuring different scaling behaviour of the norm on source and target.)
  The second map has norm at most $N$ (as it is a sum of $N$ maps of norm $\leq 1$).
  Thus, the total map has norm $\leq r^bN$. But by our choice of $N$, we have $r^bN\leq \epsilon$, giving the result.
\end{proof}

\begin{proof}[Proof of Theorem~\ref{explicit}]
  \proves{explicit}
  \leanok
  \uses{spectral,col_exact, double-complex-htpy}
  By induction, the first condition of Proposition~\ref{spectral}
  is satisfied for all $c\geq c_0$ with $c_0$ large enough
  (depending on $\Lambda$ but not $V$ or $S$).

  The second condition is Proposition~\ref{col_exact},
  and the third condition has been checked in Proposition~\ref{double-complex-htpy}.

  Thus, we can apply Proposition~\ref{spectral},
  and get the desired $\leq \max(k'^2,2k_0H)$-exactness in degrees $\leq m$ for $c\geq c_0$,
  where $k'$, $k_0$ and $H$ were defined only in terms of $k$, $m$, $r'$ and $r$,
  while $c_0$ depends on $\Lambda$ (but not on $V$ or $S$).
  This proves the inductive step.
\end{proof}

\begin{question}
  Can one make the constants explicit, and how large are they?
  \footnote{A back of the envelope calculation seems to suggest that $k$ is roughly doubly exponential in $m$, and that $N$ has to be taken of roughly the same magnitude.}
  Modulo the Breen-Deligne resolution, all the arguments give in principle explicit constants;
  and actually the proof of the existence of the Breen-Deligne resolution
  should be explicit enough to ensure the existence of bounds on the $c_i$ and $c_i'$.

  Answer: yes, we can do this.
  And we should write up a clean account asap,
  when we have cleaned up the proof in Lean.
\end{question}

% \subsection{Relevant material that should move to a better place}

% The next statement uses the definition of the completion of a condensed abelian group, see Definition~\ref{CLC}.
%
% \begin{proposition}
%   \label{prop:normedcompletion}
%   \uses{CLC}
% The condensed abelian group $\widehat{M}$ is canonically identified with the condensed abelian group associated to the topological abelian group $\widehat{M}_{\mathrm{top}}$ given by the completion of $M$ equipped with the topology induced by the norm. The norm defines a natural map of condensed sets
% \[
% ‖\cdot‖: \widehat{M}\to \mathbb R_{\geq 0}.
% \]
% \end{proposition}
%
% \begin{proof}
% Note that in the supremum norm any continuous function from $S$ to $\widehat{M}_{\mathrm{top}}$ can be approximated by locally constant functions arbitrarily well, and that the space of continuous functions from $S$ to $\widehat{M}_{\mathrm{top}}$ is complete with respect to the supremum norm. That $‖\cdot‖$ defines a map of condensed sets $\widehat{M}\to \mathbb R_{\geq 0}$ follows for example from this identification with $\underline{\widehat{M}_{\mathrm{top}}}$, as the norm is by definition a continuous map $\widehat{M}_{\mathrm{top}}\to \mathbb R_{\geq 0}$.
% \end{proof}


% vim: ts=2 et sw=2 sts=2


% vim: ts=2 et sw=2 sts=2
