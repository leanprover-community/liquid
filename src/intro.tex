\section{Introduction}
\label{intro}

The goal of this document is to provide a detailed account
of the proof of the following theorem,
along side a computer verification in the Lean theorem prover
(see Section~\ref{on-lean}).
This text is based on the lecture notes on Analytic Geometry~\cite{Analytic},
by Peter Scholze.
This text is meant as a blueprint for the Liquid Tensor Experiment.

\begin{theoremx}[Clausen--Scholze]
  \label{main-goal}
  Let $0 < p' < p \le 1$ be real numbers,
  let $S$ be a profinite set,
  and let $V$ be a $p$-Banach space.
  Let $\mathcal M_{p'}(S)$ be the space of $p'$-measures on $S$.
  Then
  \[
    \Ext^i_{\Cond(\Ab)}(\mathcal M_{p'}(S), V)=0
  \]
  for $i \ge 1$.
\end{theoremx}

We will explain this statement in more detail in Section~\ref{on-the-statement} below.

\begin{remark}
  Status report of the project as of 10-06-2021.
  The main ingredient in the proof of Theorem~\ref{main-goal}
  is the highly technical Theorem~\ref{first_target}.

  We have completed a computer verification of this first target in Lean,
  and this document contains an account of the proof.
  We are in the process of cleaning up and documenting this first step.

  Once that is done, we will work on the second step:
  deducing Theorem~\ref{main-goal} from Theorem~\ref{first_target}.
\end{remark}

\subsection{On the statement of the main goal}
\label{on-the-statement}

For the definition of condensed sets and condensed abelian groups,
we refer to~\cite{Condensed}.

A $p$-Banach space $V$ is a complete normed abelian group
that is also a real vector space,
satisfying $\|rv\| = |r|^p\|v\|$.

We will now explain the space $\mathcal M_{p'}(S)$.
TODO
(For now, see Definition~6.3 of~\cite{Analytic}.)

\subsection{On the Lean theorem prover and computer verification of proofs}
\label{on-lean}

The Lean theorem prover is developed by Leonardo de Moura
at Microsoft Research.

TODO: write a short paragraph about what computer verification means,
add pointers to further tutorials/material

\subsection{Generic remarks}

\begin{remark}
  In this text $\N$ denotes the natural numbers \emph{including} $0$.
\end{remark}

