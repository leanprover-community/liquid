\subsection{The MacLane $Q'$-construction}

In this subsection we will focus on the functorial complex induced by the Breen--Deligne package
described in Definition~\ref{BD_eg}.
This complex is also known as MacLane's $Q'$-construction.
(TODO: Rewrite the subsection on Breen--Deligne packages to reflect this.)

\begin{proposition}
  \label{homology-Qprime}
  \uses{BD_eg}
  \lean{breen_deligne.package.eval_additive, breen_deligne.data.eval_functor_preserves_filtered_colimits}
  \leanok
  For any $i\geq 0$, the functor $A\mapsto H_i(Q'(A))$ has the following properties:
  \begin{enumerate}
    \item It is additive, i.e.
      \[ H_i(Q'(A\oplus B))\cong H_i(Q'(A))\oplus H_i(Q'(B)).  \]
    \item It commutes with filtered colimits, i.e.~for a filtered inductive system $A_i$,
      \[ \varinjlim_i H_i(Q'(A))\cong H_i(Q'(\varinjlim_i A_i)). \]
  \end{enumerate}

  In particular, for torsion-free abelian groups $A$, there is a functorial isomorphism
  \[ H_i(Q'(A))\cong H_i(Q'(\mathbb Z))\otimes A.  \]
\end{proposition}

As the proof shows, we do not really need the $Q'$-construction here: Any Breen--Deligne package will do.

\begin{proof}
  \leanok
  Let us do the easy things first. Part (2) is clear as everything in sight commutes with filtered colimits.
  Assuming (1), we note that there is a natural map
  \[ H_i(Q'(\mathbb Z))\times A\to H_i(Q'(A)) \]
  induced by functoriality of $H_i(Q'(-))$. To check that this is bilinear and induces an isomorphism
  \[ H_i(Q'(\mathbb Z))\otimes A\cong H_i(Q'(A)), \]
  we can reduce to the case that $A$ is finitely generated by (2).
  In that case $A$ is finite free, and the result follows from (1).

  Thus, it remains to prove part (1), which has already been formalized.
  We recall that the direct sum of two abelian groups $M$ and $N$
  is characterized as the abelian group $P$ with maps $i_M: M\to P$, $i_N: N\to P$, $p_M: P\to M$, $p_N: P\to N$,
  satisfying $p_M i_M=\mathrm{id}_M$, $p_N i_N = \mathrm{id}_N$, $p_M i_N=0$, $p_N i_M = 0$, $\mathrm{id}_P = i_M p_M + i_N p_N$.
  Apply this to $M=H_i(Q'(A))$, $N=H_i(Q'(B))$ and $P=H_i(Q'(A\oplus B))$,
  with all maps induced by applying $H_i(Q'(-))$ to the similar maps for $A$, $B$ and $A\oplus B$.
  The fact that $H_i(Q'(-))$ is a functor already gives all identities except $\mathrm{id}_P = i_M p_M + i_N p_N$,
  and the only issue is the question whether $H_i(Q'(-))$ induces additive maps on morphism spaces.
  But if $f,g: C\to D$ are any two maps of abelian groups, then $H_i(Q'(f+g)) = H_i(Q'(f))+H_i(Q'(g))$,
  by reducing to the universal case of the two projections $D^2\to D$ and using the homotopy baked into Definition~\ref{BD_eg}.
\end{proof}

We note that by functoriality of the $Q'$-construction, it can also be applied to condensed abelian groups.

\begin{corollary}
  \label{Qprime-Cond}
  \lean{Condensed.exists_tensor_iso}
  \leanok
  \uses{BD_eg, homology-Qprime}
  For torsion-free condensed abelian groups $A$, there is a natural isomorphism
  \[ H_i(Q'(A))\cong H_i(Q'(\mathbb Z))\otimes A \]
  of condensed abelian groups.
\end{corollary}

Here, we only need to be able to tensor condensed abelian groups with (abstract) abelian groups.
(With more effort, one could prove that $H_i(Q'(\mathbb Z))$ is even finitely generated.)
In that case, the tensor product functor can be defined very naively by tensoring the values at any $S$ with the given abstract abelian group.

\begin{proof}
  \leanok
  Evaluating at $S\in \mathrm{ExtrDisc}$, we note that $S\mapsto H_i(Q'(A(S)))$ is already a condensed abelian group,
  and agrees with $H_i(Q'(\mathbb Z))\otimes A(S)$. Thus, the same is true after sheafification.
\end{proof}

If $A$ is a torsion-free condensed abelian group equipped with an endomorphism $f$,
then $Q'(A)$ is also equipped with the endomorphism $f$ induced by functoriality,
and by functoriality all previous assertions upgrade to $\mathbb Z[f]$-modules.
% We will need the following proposition in the proof of Theorem~\ref{thm:91prime}.

\begin{proposition}
  \label{Qprime-prop}
  \uses{BD_eg}
  \lean{breen_deligne.package.main_lemma}
  \leanok
  Let $M$ and $N$ be condensed abelian groups with endomorphisms $f_M$, $f_N$.
  Assume that $M$ is torsion-free (over $\mathbb Z$). Then
  \[ \mathrm{Ext}^i_{\mathbb Z[f]}(M,N)=0 \]
  for all $i\geq 0$ if and only if
  \[ \mathrm{Ext}^i_{\mathbb Z[f]}(Q'(M),N)=0 \]
  for all $i\geq 0$. More precisely, the first vanishes for $0\leq i\leq j$ if and only if the second vanishes for $0\leq i\leq j$.
\end{proposition}

At this point, we need to be able to talk about $\mathrm{Ext}$-groups of (bounded to the right) complexes of condensed abelian groups (against condensed abelian groups).

The statement is also true without the torsion-freeness assumption on $M$, but slightly more nasty to prove then (and not required for the application).

\begin{proof}
  \leanok
  \uses{homology-Qprime, Qprime-Cond}
  We induct on $j$.
  Consider first the case $j=0$; then any map $Q'(M)\to N$ factors uniquely over $H_0 Q'(M)[0]=M[0]$, yielding the result.
  Now assume that both sides vanish for $0\leq i<j$; we need to see that the vanishing of the $\mathrm{Ext}^i$'s is equivalent.
  Consider the triangle
  \[ \tau_{\geq 1} Q'(M)\to Q'(M)\to M[0]\to . \]
  Taking the corresponding long exact sequence of $\mathrm{Ext}$-groups against $N$, we see that it suffices to see that
  \[ \mathrm{Ext}^i_{\mathbb Z[f]}(\tau_{\geq 1} Q'(M),N)=0 \]
  for $0\leq i\leq j$. But we can prove by descending induction on $t$ that
  \[ \mathrm{Ext}^i_{\mathbb Z[f]}(\tau_{\geq t} Q'(M),N)=0.  \]
  This is trivially true for $t>i$. Now look at the triangle
  \[ \tau_{\geq t+1} Q'(M)\to \tau_{\geq t} Q'(M)\to H_t(Q'(M))[t]\to \]
  and the corresponding long exact sequence. It becomes sufficient to prove that
  \[ \mathrm{Ext}^i_{\mathbb Z[f]}(H_t(Q'(M))[t],N)=0 \]
  for $0\leq i\leq j$. Trivially,
  \[ \mathrm{Ext}^i_{\mathbb Z[f]}(H_t(Q'(M))[t],N)=\mathrm{Ext}^{i-t}_{\mathbb Z[f]}(H_t(Q'(M)),N).  \]
  Note that $t\geq 1$ here, so $i-t<j$ (and can be assumed $\geq 0$). Also $H_t(Q'(M))\cong H_t(Q'(\mathbb Z))\otimes M$. Thus, it suffices to show that for every abelian group $A$ and every $0\leq i<j$,
  \[ \mathrm{Ext}^i_{\mathbb Z[f]}(A\otimes M,N)=0.  \]
  If $A$ is free, then $A\otimes M$ is a direct sum of copies of $M$,
  and the result follows as $\mathrm{Ext}$ turns direct sums into products
  (and we assumed the vanishing of $\mathrm{Ext}^i_{\mathbb Z[f]}(M,N)$ for $0\leq i<j$).
  In general, one can pick a two-term free resolution of $A$ and use the long exact sequence.
\end{proof}

% vim: ts=2 et sw=2 sts=2

