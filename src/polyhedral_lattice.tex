\section{Polyhedral lattices}
\label{sec:polyhedral_lattice}

\begin{definition}
  \label{polyhedral_lattice}
  \lean{polyhedral_lattice}
  A \emph{polyhedral lattice} is a finite free abelian group~$\Lambda$
  equipped with a norm $‖\cdot‖_\Lambda \colon \Lambda\otimes \mathbb R\to \mathbb R$
  (so $\Lambda\otimes \mathbb R$ is a Banach space)
  that is given by the supremum of finitely many linear functions on $\Lambda$;
  equivalently, the ``unit ball''
  $\{\lambda\in \Lambda\otimes \mathbb R\mid ‖\lambda‖_\Lambda\leq 1\}$ is a polyhedron.
\end{definition}

Finally, we can prove the key combinatorial lemma,
ensuring that any element of $\Hom(\Lambda,\overline{\mathcal M}_{r'}(S))$
can be decomposed into $N$ elements whose norm is roughly $\tfrac 1N$ of the original element.

\begin{lemma}
  \label{combi}
  \lean{lem98}
  \leanok
  \uses{polyhedral_lattice}
  Let $\Lambda$ be a polyhedral lattice.
  Then for all positive integers $N$ there is a constant $d$
  such that for all $c>0$ one can write any
  $x\in \Hom(\Lambda,\overline{\mathcal M}_{r'}(S))_{\leq c}$ as
  \[
    x=x_1+\ldots+x_N
  \]
  where all $x_i\in \Hom(\Lambda,\overline{\mathcal M}_{r'}(S))_{\leq c/N+d}$.
\end{lemma}

As preparation for the proof, we have the following results.

\begin{lemma}[Gordan's lemma]
  \label{explicit_gordan}
  \lean{explicit_gordan}
  \leanok
  Let $\Lambda$ be a finite free abelian group,
  and let $S=\{\lambda_1, \ldots, \lambda_m\}\subseteq \Lambda$ be a finite set of elements.
  Let $M(S) \subset \Lambda^*:=\Hom(\Lambda, \mathbb Z)$ be the submonoid
  \(\{x \mid x(\lambda_i) \ge 0 \text{ for all \(i = 1, \dots, m\)}\}\).
  Then $M(S)$ is finitely generated as monoid.
\end{lemma}
This is a standard result, but the proof is rather intricate. The following subsection contains the proof.

\subsection{Proof of Gordan's Lemma}.

We begin by demonstrating how to bring commutative ring theory into the mix.

\begin{lemma} If $M$ is an additive abelian monoid and if $R$ is a nonzero commutative ring, then $M$ is finitely-generated as a monoid iff the monoid algebra $R[M]$ is finitely-generated as an $R$-algebra.
\end{lemma}
\begin{proof} Because $R$ is nonzero, we can think of $M$ as a subset of $R[M]$, and even as a multiplicative submonoid of $R[M]$.

  The easy direction: say $M$ is finitely-generated by $S\subseteq M$. The claim is that $S$ generates $R[M]$ as an $R$-algebra. Indeed, the sub-$R$-algebra of $R[M]$ generated by~$S$ contains~$M$, and is an $R$-module, so it contains the $R$-module generated by~$M$ which is all of $R[M]$.

  Conversely, say $R[M]$ is generated by a finite subset~$T$ as an $R$-algebra. Any $f\in R[M]$ has a support, which is a finite subset of~$M$, and $f$ is in the $R$-module (and hence the $R$-algebra) generated by its support, so by replacing the elements of~$T$ by their supports we may assume that $T\subseteq M$. We claim that~$S$ generates~$M$ as an additive monoid. This follows because the $R$-algebra generated by~$S$ equals the $R$-module generated by the submonoid~$M'$ of~$M$ spanned by~$S$, and if $M'\not=M$ then its $R$-module span cannot be $R[M]$ (again using that $R$ is nonzero).
\end{proof}

This result shows that Gordan's Lemma can be attacked using commutative algebra and the theory of graded rings. We say that a commutative ring $R$ is \emph{graded} by an additive abelian monoid~$M$ if $R=\bigoplus_{m\in M} R_m$ with $R_m$ a family of additive subgroups of $(R,+)$ indexed by~$M$, such that $1\in R_0$ and $R_{a}R_b\subseteq R_{a+b}$. Here are some lemmas about graded rings.

\begin{lemma} $R_0$ is a subring of~$R$ and multiplication on~$R$ turns $R_a$ into an $R_0$-submodule.
\end{lemma}
\begin{proof} Everything here follows immediately from the definitions.
\end{proof}

\begin{lemma} If $R$ is a Noetherian ring then $R_a$ is a Noetherian $R_0$-module.
\end{lemma}
\begin{proof} We show that any increasing chain $J_0\subseteq J_1\subseteq J_2\subseteq\cdots$ of $R_0$-submodules of $R_a$ stabilises. If $RJ_n$ denotes the ideal of~$R$ generated by $J_n$ then the $RJ_n$ are an increasing sequence of ideals of~$R$ and hence they stabilise. The result will then follow if we can show that $J_n=RJ_n\cap R_a$ can be recovered from $RJ_n$. The inclusion $J_n\subseteq RJ_n\cap R_a$ is clear, and the converse inclusion follows from the fact that if $t:=\sum_i r_i j_i\in R_a$ with $r_i\in R$ and $j_i\in J_n\subseteq R_a$, then, writing $x_a$ for the $a$th graded piece of $x\in R$ we deduce that $t=\sum_i(r_ij_i)_a=\sum_i(r_i)_0j_i\in J_n$.
\end{proof}

If $R$ is a commutative ring graded by $\mathbb{Z}$ then let $R_{\geq0}$ denote teh subring $\oplus_{n\geq0}R_n$ of $R$.

\begin{theorem} If $R$ is Noetherian and graded by $\mathbb{Z}$ then $R_{\geq0}$ is a finitely-generated $R_0$-algebra.
\end{theorem}
\begin{proof}
  Let $S$ denote the union of $R_n$ for $n>0$, and let $I$ denote the ideal of~$R$ generated by~$S$. Because $R$ is Noetherian, $I$ is generated by a finite subset $S_0$ of $S$. Let $N$ be the maximum of the degrees of the elements of $S_0$ (or 37 if $S_0$ is empty). By the previous lemma, $R_n$ is a Noetherian $R_0$-module and hence finitely-generated as an $R_0$-module by a subset $T_n$. We claim that~$X$, the union of $S_0$ and the $T_n$ for $0\leq n\leq N$, generates $R_{\geq0}$ as an $R_0$-algebra.

  It suffices to prove that the $R_0$-algebra $R_0[X]$ generated by~$X$ contains $R_n$ for all $n\geq0$. For $n\leq N$ this is clear. For $n>N$ we proceed by strong induction. Say $f\in R_n$ and we know $R_m$ for $m<n$ are all in $R_0[X]$. Because $f\in I$ we can write $f=\sum_i r_if_i$ with $f_i\in S_0$ homogeneous of degree $j_i$. Taking $n$th components and observing that $f$ and $f_i$ are homogeneous we deduce $f=\sum_i(r_i)_{n-j_i}f_i$. By our inductive hypothesis, $(r_i)_{n-j_i}$ and $f_i$ are in $R[X]$, and we are home.
\end{proof}

We are finally ready to prove Gordan's Lemma.
  
\begin{proof}[Proof of Gordan's Lemma]
  We want to prove that if $\Lambda$ is a finite free $\mathbb{Z}$-module, then for all finite subsets $S\subseteq\Lambda$, the submonoid $M(S)\subseteq\Lambda^*$ is a finitely-generated additive monoid. We proceed by induction on the rank of $\Lambda$. The base case is trivial. For the inductive step, we proceed by a second induction, this time on the size of~$S$. Again the base case is trivial, because $\mathbb{Z}^n$ is generated as an additive monoid by $\pm e_i$ with $e_i$ the standard $\mathbb{Z}$-basis of $\mathbb{Z}^n$. It suffices to deal with the case where $S=S'\cup\{\ell\}$, where we may assume $M(S')$ is finitely-generated.

Note that if $\ell=0$ then $M(S)=M(S')$ and we are home, so we may assume $\ell\not=0$. We now choose a field $k$, for example the rationals. Our inductive hypothesis tells us that $M(S')$ is finitely-generated, and hence the monoid algebra $R:=k[M(S')]$ is a finitely-generated $k$-algebra and hence in particular a Noetherian ring. We grade this ring by $\mathbb{Z}$ in the following way: the element $\ell$ induces a map $M(S')\to\mathbb{Z}$ and we let $R_n$ be the $k$-subspace of $R$ spanned by the elements of $M(S')$ sent to $n$. This is easily checked to be a grading, and furthermore $R_{\geq0}$ is $k[M(S)]$. It thus suffices to prove that $R_{\geq0}$ is a finitely-generated $k$-algebra. We have proved in a previous lemma that it is a finitely-generated $R_0$-algebra, so what remains is to prove that $R_0$ is a finitely-generated $k$-algebra. Now $R_0$ is the monoid algebra for the monoid $M_0:=\{\phi\in M(S)\,|\,\phi(\ell)=0\}\subseteq\Lambda^*$, and we will show that this is finitely generated by our inductive hypothesis.

Define $\Lambda'$ to be the torsion-free quotient of $\Lambda/\langle \ell\rangle$. Its dual $\Lambda'^*$ is naturally isomorphic to $\{\phi\in\Lambda^*\,|\,\phi(\ell)=0\}$, and because $\ell\not=0$ these groups are finite and free of rank one less than $\Lambda$. If $T$ is the image of $S'$ in $\Lambda'$ then $M_0$ is isomorphic to $M(S')$, so must be finitely-generated by the inductive hypothesis. 

  
\end{proof}


\begin{lemma}
  \label{combi_aux}
  \lean{lem97}
  \leanok
  Let $\Lambda$ be a finite free abelian group,
  let $N$ be a positive integer,
  and let $\lambda_1,\ldots,\lambda_m\in \Lambda$ be elements.
  Then there is a finite subset $A\subset \Lambda^\vee$
  such that for all $x\in \Lambda^\vee=\Hom(\Lambda,\mathbb Z)$
  there is some $x'\in A$ such that $x-x'\in N\Lambda^\vee$
  and for all $i=1,\ldots,m$,
  the numbers $x'(\lambda_i)$ and $(x-x')(\lambda_i)$ have the same sign,
  i.e.~are both nonnegative or both nonpositive.
\end{lemma}

\begin{proof}
  \uses{explicit_gordan}
  \leanok
  It suffices to prove the statement for all $x$ such that $\lambda_i(x)\geq 0$ for all $i$;
  indeed, applying this variant to all $\pm \lambda_i$, one gets the full statement.

  Thus, consider the submonoid $\Lambda^\vee_+\subset \Lambda^\vee$
  of all $x$ that pair nonnegatively with all $\lambda_i$.
  This is a finitely generated monoid by Lemma~\ref{explicit_gordan};
  let $y_1,\ldots,y_M$ be a set of generators.
  Then we can take for $A$ all sums $n_1y_1+\ldots+n_My_M$ where all $n_j\in \{0,\ldots,N-1\}$.
\end{proof}

\begin{lemma}
  \label{exists_partition}
  \lean{exists_partition}
  \leanok
  Let $x_0, x_1, \dots$ be a sequence of reals,
  and assume that $\sum_{i=0}^\infty x_i$ converges absolutely.
  For every natural number $N > 0$,
  there exists a partition $\mathbb N = A_1 \sqcup A_2 \sqcup \dots \sqcup A_N$
  such that for each $j = 1,\dots,N$ we have
  $\sum_{i \in A_j} x_i \le (\sum_{i=0}^\infty x_i)/N + 1$
\end{lemma}

\begin{proof}
  \leanok
  Define the $A_j$ recursively:
  assume that the natural numbers $0, \dots, n$
  have been placed into the sets $A_1, \dots, A_N$.
  Then add the number $n+1$ to the set $A_j$ for which
  \[
    \sum_{i=0, i\in A_j}^n x_i
  \]
  is minimal.
\end{proof}

\begin{lemma}
  \label{lem98_int}
  \lean{lem98_int}
  \leanok
  For all natural numbers $N > 0$,
  and for all $x\in \overline{\mathcal M}_{r'}(S)_{\leq c}$
  one can decompose $x$ as a sum
  \[
    x=x_1+\ldots+x_N
  \]
  with all $x_i\in \overline{\mathcal M}_{r'}(S)_{\leq c/N+1}$.
\end{lemma}

\begin{proof}
  \leanok
  \uses{exists_partition}
  Choose a bijection $S \times \mathbb N \cong \mathbb N$,
  and transport the result from Lemma~\ref{exists_partition}.
\end{proof}

\begin{proof}[{Proof of Lemma~\ref{combi}}]
  \proves{combi}
  \uses{combi_aux, lem98_int}
  \leanok
  Pick $\lambda_1,\ldots,\lambda_m\in \Lambda$ generating the norm. We fix a finite subset $A\subset \Lambda^\vee$ satisfying the conclusion of the previous lemma. Write
  \[
  x=\sum_{n\geq 1, s\in S} x_{n,s} T^n [s]
  \]
  with $x_{n,s}\in \Lambda^\vee$. Then we can decompose
  \[
  x_{n,s} = N x_{n,s}^0 + x_{n,s}^1
  \]
  where $x_{n,s}^1\in A$ and we have the same-sign property of the last lemma. Letting $x^0 = \sum_{n\geq 1, s\in S} x_{n,s}^0 T^n [s]$, we get a decomposition
  \[
  x = Nx^0 + \sum_{a\in A} a x_a
  \]
  with $x_a\in \overline{\mathcal M}_{r'}(S)$ (with the property that in the
  basis given by the $T^n [s]$, all coefficients are $0$ or $1$). Crucially,
  we know that for all $i=1,\ldots,m$, we have
  \[
  ‖x(\lambda_i)‖ = N ‖x^0(\lambda_i)‖ + \sum_{a\in A} |a(\lambda_i)| ‖x_a‖
  \]
  by using the same sign property of the decomposition.

  Using this decomposition of $x$, we decompose each term into $N$ summands.
  This is trivial for the first term $Nx^0$,
  and each summand of the second term decomposes with $d = 1$ by Lemma~\ref{lem98_int}.
  (It follows that in general one can take for $d$
  the supremum over all $i$ of $\sum_{a\in A} |a(\lambda_i)|$.)
\end{proof}

\begin{definition}
  \label{rescaled-sum}
  \uses{polyhedral_lattice}
  \leanok
  \lean{rescale.polyhedral_lattice}
  \lean{finsupp.polyhedral_lattice}
  Let $\Lambda$ be a polyhedral lattice, and let $N > 0$ be a natural number.
  (We think of $N$ as being fixed once and for all,
  and thus it does not show up in the notation below.)

  By $\Lambda'$ we denote $\Lambda^N$ endowed with the norm
  \[
	  ‖(\lambda_1,\ldots,\lambda_N)‖_{\Lambda'} = \tfrac 1N(‖\lambda_1‖_\Lambda+\ldots+‖\lambda_N‖_\Lambda).
  \]
  This is a polyhedral lattice.
\end{definition}

\begin{lemma}
  \label{polyhedral-quotient}
  \uses{rescaled-sum}
  \lean{polyhedral_lattice.conerve.obj.polyhedral_lattice}
  \leanok
  For any $m\geq 1$, let $\Lambda'^{(m)}$ be given by $\Lambda'^m / \Lambda\otimes (\mathbb Z^m)_{\sum=0}$;
  for $m=0$, we set $\Lambda'^{(0)} = \Lambda$.
  Then $\Lambda'^{(m)}$ is a polyhedral lattice.
\end{lemma}

\begin{proof}
  WIP
\end{proof}

\begin{definition}
  \label{cosimplicial-lattice}
  \uses{polyhedral-quotient}
  \lean{PolyhedralLattice.cosimplicial}
  \leanok
  For any $m\geq 1$, let $\Lambda'^{(m)}$ be given by $\Lambda'^m / \Lambda\otimes (\mathbb Z^m)_{\sum=0}$;
  for $m=0$, we set $\Lambda'^{(0)} = \Lambda$.
  Then $\Lambda'^{(\bullet)}$ is a cosimplicial polyhedral lattice,
  the \v{C}ech conerve of $\Lambda\to \Lambda'$.

  In particular, $\Lambda'^{(0)} = \Lambda \to \Lambda' = \Lambda'^{(1)}$
  is the diagonal embedding.
\end{definition}

\begin{definition}
  \label{Hom}
  \uses{polyhedral_lattice, profinitely_filtered_pseudo_normed_group_with_Tinv}
  \lean{polyhedral_lattice.add_monoid_hom.profinitely_filtered_pseudo_normed_group_with_Tinv}
  \leanok
  Let $\Lambda$ be a polyhedral lattice,
  and $M$ a profinitely filtered pseudo-normed group.

  Endow $\Hom(\Lambda, M)$ with the subspaces
  \[
    \Hom(\Lambda, M)_{\leq c} =
    \{f \colon \Lambda \to M \mid
      \forall x \in \Lambda, f(x) \in M_{\leq c‖x‖} \}.
  \]
  As $\Lambda$ is polyhedral, it is enough to check the given condition on~$f$
  for a finite collection of $x$ that generate the norm.

  These subspaces are profinite subspaces of $M^\Lambda$,
  and thus they make $\Hom(\Lambda, M)$ ito a profinitely filtered pseudo-normed group.

  If $M$ has an action of $T^{-1}$, then so does $\Hom(\Lambda, M)$.
\end{definition}

% vim: ts=2 et sw=2 sts=2

