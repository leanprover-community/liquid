\subsection{Some normed homological algebra}%
\label{sec:some_normed_homological_algebra}

It will be convenient to use the following definition generalizing the notion
of a bound on the norm of a right inverse of a normed group morphism (note that
the morphisms we consider have no reason to have a right inverse, even when
they are surjective).

\begin{definition}
  \label{def:surjective_with_bound}
  \lean{normed_group_hom.surjective_on_with}
  \leanok
  Let $G$ and $H$ be semi-normed groups, let $K$ be a subgroup of $H$ and $C$ be
  a positive real number.
  A morphism $f : G → H$ is $C$-surjective onto $K$ if, for all $x$ in $K$,
  there exists some $g$ in $G$ such that $f(g) = x$ and $\|g\| ≤ C\|x\|$.
  If $K = H$ we simply say $f$ is $C$-surjective.
\end{definition}

The following controlled surjectivity lemma will be used to prove
Lemma~\ref{lem:controlled_exactness} and Lemma~\ref{lem:Tinv}.

\begin{lemma}
  \label{lem:closure_surjective}
  \uses{def:surjective_with_bound}
  \lean{controlled_closure_of_complete}\leanok
  Let $G$ and $H$ be normed groups.
  Let $K$ be a subgroup of $H$ and $f$ a morphism from $G$ to~$H$.
  Assume that $G$ is complete and $f$ is $C$-surjective onto $K$.
  Then $f$ is $(C + ε)$-surjective onto
  the topological closure of $K$ for every positive $ε$.
\end{lemma}

\begin{proof}\leanok
  Let $x$ be any element of the closure of $K$.
  First note the conclusion is trivial when $x = 0$, so we can assume
  $x ≠ 0$. Then write $x$ as a sum
  $\sum_{i \ge 0} x_i$ with all $x_i \in K$, $\|x - x_0\| ≤ ε_0$ and
  $‖x_i‖\leq \epsilon_i$ for $i>0$ for some sequence of positive numbers
  $\epsilon_i$ to be chosen later.
  By assumption, we can then lift each $x_i$ to $g_i$ such that
  $f(g_i) = x_i$ and $‖g_i‖\leq C‖x_i‖$, and then set
  $g = \sum g_i$. Because $G$ is complete,
  this sum converges provided the $ε_i$ sequence converges fast enough to zero.
  We then have $f(g) = x$ and
  \[
    ‖g‖ ≤ C\sum_{i \geq 0} ‖x_i‖ ≤
    C(\|x\| + ε_0) + C\sum_{i>0} ε_i ≤
    (C + ε)‖x‖
  \]
  where the last inequality holds provided the $ε_i$ sequence converges fast
  enough to zero. For instance $ε_i = ε∥x∥/(2^{i+1}C)$ satisfies all
  our constraints on the $ε_i$ sequence (in particular they are positive
  because $x ≠ 0$).
\end{proof}

The first application of the above lemma is a completion result for
a quantitative version of being a complex.

\begin{lemma}\label{lem:controlled_exactness}
  \uses{def:surjective_with_bound}
  \lean{controlled_exactness}\leanok
  Let $f : M_0 → M_1$ and $g : M_1 → M_2$ be bounded maps between normed groups.
  Assume there are positive constants $C$ and $D$ such that:
  \begin{itemize}
    \item
      $f$ is $C$-surjective onto $\ker g$.
    \item
      $g$ is $D$-surjective onto its image.
  \end{itemize}
  Then for every positive $ε$, $\widehat{f}$ is $(C + ε)$-surjective onto
  $\ker \widehat{g}$.
\end{lemma}

\begin{proof}\leanok
  \uses{lem:closure_surjective}
  Since $f$ is $C$-surjective onto $\ker g$, $\widehat{f}$ is $C$-surjective
  onto $\ker g$ seen as a subset of $\widehat{M_1}$. Hence this lemma will
  follow directly from Lemma~\ref{lem:closure_surjective}
  once we'll have proven that $\ker g$ is dense in $\ker \widehat{g}$.
  Let $\widehat y$ be an element of $\ker \widehat{g}$.
  Pick any $\delta > 0$ and take $y\in M_1$ such that
  $‖\widehat{y}-y‖\leq \delta$. Let $z=g(y)\in M_2$, which has norm
  $‖z‖=‖g(y)‖=‖g(y-\widehat{y})‖$ bounded by
  $C_{g}\delta$, where $C_{g}$ is the norm of $g$. We can thus find some
  $y'\in M_1$ with $‖y'‖\leq DC_{g}\delta$ and $g(y')=z$. Replacing $y$ by
  $y-y'$, we can thus find $y\in \ker(g: M_1\to M_2)$ such that still
  $‖\widehat{y}-y‖\leq (1+DC_{g})\delta$; as $\delta$ was arbitrary, this
  gives the desired density.
\end{proof}

\begin{definition}
  \label{system_of_complexes}
  \lean{system_of_complexes}
  \leanok
  A \emph{system of complexes} of normed abelian groups
  is for each $c \in \mathbb R_{\ge 0}$
  a complex
  \[
  C_c^\bullet: C_c^0\to C_c^1\to\ldots
  \]
  of normed abelian groups together with maps of complexes
  $\mathrm{res}_{c',c}: C_{c'}^\bullet\to C_c^\bullet$,
  for $c' ≥ c$,
  satisfying $\mathrm{res}_{c,c}=\mathrm{id}$ and the obvious associativity condition.
  In other words, a functor from $(\mathbb R_{\ge 0})^{\mathrm{op}}$ to
  cochain complexes of semi-normed groups.
\end{definition}

By convention, for every system of complexes $C_\bullet^\bullet$,
we will set $C^{-1}_c = 0$ for all $c$.
This will come up each time we write $C^{i-1}_c$ and $i$ could be $0$.

In this subsection, given $x ∈ C^•_{c'}$ and $c_0\leq c ≤ c'$ we will use the notation
$x_{|c} := \mathrm{res}_{c', c}(x)$.

\begin{definition}
  \label{admissible}
  \lean{system_of_complexes.admissible}
  \leanok
  \uses{system_of_complexes}
  A system of complexes is \emph{admissible}
  if all differentials and maps $\mathrm{res}_{c',c}^i$ are norm-nonincreasing.
\end{definition}

Throughout the rest of this subsection, $k$ (and $k'$, $k''$) will denote reals at least 1,
$m$ will be a non-negative integer, and $K$, $K'$, $K''$ will denote non-negative reals.

\begin{definition}
  \label{norm_exact_complex}
  \lean{system_of_complexes.norm_exact_complex}
  \leanok
  A cochain complex~$C$ of semi-normed groups is \emph{normed exact} if
  for all $i \ge 0$, all $\varepsilon > 0$, and all $x \in C^i$ with $d(x) = 0$
  there exists a $y \in C^{i-1}$ such that $d(y) = x$ and $\|y\| \le (1 + \varepsilon)\|x\|$.
\end{definition}

\begin{definition}
  \label{is_bounded_exact}
  \lean{system_of_complexes.is_bounded_exact}
  \leanok
  \uses{system_of_complexes}
  Let $C_\bullet^\bullet$ be a system of complexes.
  For an integer $m\geq 0$ and reals $k \ge 1$, $K \ge 0$ and $c_0 \ge 0$,
  we say the datum $C_\bullet^\bullet$ is
  \emph{$\leq k$-exact in degrees $\leq m$ and for $c\geq c_0$ with bound $K$} if the following condition is satisfied.
  For all $c\geq c_0$ and all $x\in C_{kc}^i$ with $i\leq m$
  there is some $y\in C_c^{i-1}$ such that
  \[
    ‖x_{|c} - dy‖ ≤ K ‖dx‖.
  \]
\end{definition}

We will also need a version where the inequality is relaxed by some arbitrary small additive constant.

\begin{definition}
  \label{is_weak_bounded_exact}
  \lean{system_of_complexes.is_weak_bounded_exact}
  \leanok
  \uses{system_of_complexes}
  Let $C_\bullet^\bullet$ be a system of complexes.
  For an integer $m\geq 0$ and reals $k \ge 1$, $K \ge 0$ and $c_0 \ge 0$,
  the datum $(C_c^\bullet)_c$ is
  \emph{weakly $\leq k$-exact in degrees $\leq m$ and for $c\geq c_0$ with bound $K$} if the following condition is satisfied.
  For all $c\geq c_0$, all $x\in C_{kc}^i$ with $i\leq m$ and any $ε > 0$
  there is some $y\in C_c^{i-1}$ such that
  \[
    ‖x_{|c} - dy‖ ≤ K ‖dx‖ + ε.
  \]
\end{definition}

We first note that the difference between those two definitions is only about
cocyles if we are ready to lose a tiny something on the norm bound $K$.

\begin{lemma}
  \label{is_bounded_exact_of_weakly}
  \lean{system_of_complexes.is_weak_bounded_exact.to_exact}
  \leanok
  \uses{is_bounded_exact, is_weak_bounded_exact}
  Let $C_\bullet^\bullet$ be a system of complexes. If $C_\bullet^\bullet$ is
  weakly $\leq k$-exact in degrees $\leq m$ and for $c\geq c_0$ with bound $K$ and if,
  for all $c\geq c_0$ and all $x\in C_{kc}^i$ with $i\leq m$ such that $dx = 0$
  there is some $y\in C_c^{i-1}$  such that
  $x_{|c} = dy$ then, for every positive $δ$,
  $C_\bullet^\bullet$ is $\leq k$-exact in degrees $\leq m$ and for $c\geq c_0$ with
  bound $K + δ$.
\end{lemma}

\begin{proof}
  \leanok
  Let $δ$ be some positive real number.
  Let $x$ be an element of $C_{kc}^i$ for some $c ≥ c_0$ and $i ≤ m$. If $dx = 0$
  then the assumption we made about exact elements is exactly what we want.

  Assume now that $dx ≠ 0$. The weak exactness assumption applied to $ε = δ‖dx‖$
  gives some $y\in C_c^{i-1}$ such that
  \begin{align*}
    ‖x_{|c} - dy‖ &≤ K‖dx‖ + δ‖dx‖ \\
                  &= (K + δ)‖dx‖
  \end{align*}
\end{proof}

\begin{lemma}
  \label{weak_exact_of_factor_exact}
  \lean{system_of_complexes.weak_exact_of_factor_exact}
  \leanok
  \uses{is_weak_bounded_exact, norm_exact_complex}
  Let $k \ge 1$, $c_0 \ge 0$ be real numbers, and $m \in \mathbb N$.
  Let $C_\bullet^\bullet$ be a system of complexes,
  and for each $c \ge 0$ let $D_c$ be a cochain complex of semi-normed groups.
  Let $f_c \colon C^\bullet_{kc} \to D^\bullet_c$
  and $g_c \colon D^\bullet_c \to C^\bullet_c$ be
  norm-nonincreasing morphisms of cochain complexes of semi-normed groups
  such that $g_c \circ f_c$ is the restriction map $C^\bullet_{kc} \to C^\bullet_c$.
  Assume that for all $c \ge c_0$ the cochain complex $D_c$ is normed exact.
  Then $C_\bullet^\bullet$ is weakly $\le k$-exact in degrees $\le m$ and for $c \ge c_0$ with bound~$1$.
\end{lemma}

\begin{proof}
  \leanok
  Fix $c \ge c_0$, $i \le m$, $x \in C_{kc}^i$, and $\varepsilon > 0$.
  Denote by $\delta$ the positive real number
  $\frac{\varepsilon}{\|x\| + 1}$.

  Clearly $f(d(x))$ is killed by $d$,
  so by normed exactness of $D_c$ we find $x' \in D_c^i$
  such that $d(x') = f(d(x))$ and $\|x'\| \le (1 + \delta)\|f(d(x))\|$.
  Similarly $d(f(x) - x') = 0$,
  so by exactness of $D_c$ we find $y \in D_c^{i-1}$
  such that $d(y) = f(x) - x'$.

  We are done if we show that
  $\|x_{|c} - d(g(y))\| \le \|d(x)\| + \varepsilon$.
  Observe that $x_{|c} - d(g(y)) = g(f(x)) - g(d(y)) = g(x')$,
  and therefore we shall show $\|g(x')\| \le \|d(x)\| + \varepsilon$.

  Now we use that $f$ and $g$ are norm-nonincreasing to calculate
  \[
    \|g(x')\| \le \|x'\| \le (1 + \delta) \|f(d(x))\| \le (1 + \delta) \|d(x)\|.
  \]
  Finally, we have $(1 + \delta) \|d(x)\| \le \|d(x)\| + \varepsilon$
  by our choice of $\delta$.
\end{proof}

% A more important observation is that, in both definitions, we can also ask some
% control on the norm of $y$ if we are ready to square the restriction depth factor $k$.
%
% \begin{lemma}
%   \label{is_weak_bdd_exact_controlled_y}
%   \uses{is_weak_bounded_exact}
%   % This statement exists in some comment somewhere
%   % \lean{system_of_complexes.is_weak_bounded_exact.controlled_y}
%   Let $C_\bullet^\bullet$ be a system of complexes which is
%   weakly $\leq k$-exact in degrees $\leq m$ and for $c\geq c_0$ with bound $K$.
%   For all $c\geq c_0$, all $x\in C_{k²c}^i$ with $i\leq m$, all $ε > 0$ and all $δ > 0$
%   there is some $y\in C_c^{i-1}$ such that
%   \[
%     ‖x_{|c} - dy‖ ≤  K ‖dx‖ + ε
%     \quad \text{and} \quad
%     ‖y‖ ≤ K(K + 1)‖x‖ + δ.
%   \]
% \end{lemma}
%
% \begin{proof}
%   Fix $x$, $ε$ and $δ$. The weak exactness assumption applied to $x$ and some $η$ to be chosen later gives us
%   $w ∈ C_{kc}^{i-1}$ such that
%   \[
%     ‖x_{|kc} - dw‖ ≤  K ‖dx‖ + η.
%   \]
%   Then the weak exactness assumption applied to $w$ and some $τ$ to be chosen later gives us
%   $z ∈ C_{c}^{i-2}$ such that
%   \[
%     ‖w_{|c} - dz‖ ≤  K ‖dw‖ + τ.
%   \]
%   We set $y = w_{|c} - dz$. Since $dy = dw_{|c}$, we get the first required estimate as long
%   as $η ≤ ε$. And we have:
%   \begin{align*}
%     ‖y‖ &≤ K ‖dw‖ + τ \\
%         &≤ K (‖x_{|kc}‖ + K‖dx‖ + η) + τ \\
%         &≤ K(K + 1) ‖x‖ + Kη + τ
%   \end{align*}
%   which is fine as long as $Kη + τ ≤ δ$.
%   So we set $η = \min(ε, δ/(2K))$ (interpreted as $ε$ if $K=0$) and $τ = δ/2$.
% \end{proof}

% \begin{lemma}
%   \label{completion_is_weakexact}
%   \uses{is_weak_bounded_exact, admissible}
%   % This statement exists in some comment somewhere
%   % \lean{system_of_complexes.is_weak_bounded_exact.completion}
%   Let $M_\bullet^\bullet$ be an admissible collection
%   of complexes of normed abelian groups.
%
%   Assume that $M^\bullet_\bullet$ is weakly $\leq k$-exact in degrees $\leq m$ for $c\geq c_0$ with bound $K$.
%   Then the completion $\overline{M^\bullet_\bullet}$ is weakly $\leq k^2$-exact in degrees $\leq m$ for $c\geq c_0$ with bound $K$.
% \end{lemma}
%
% \begin{proof}
%   \uses{is_weak_bdd_exact_controlled_y}
% Let $x \in \overline{M^i_{k^2c}}$, where $c \geq c_0$ and $i \leq m$ and let $\epsilon > 0$.
% We can write $x = \sum_j x^j$ where
% \begin{itemize}
%  \item $x^j \in M^i_{k^2c}$ for all $j \geq 0$,
%  \item $‖x - x^0‖ ≤ ε_0$ for some positive $ε_0$ to be chosen later. This implies that $‖dx - dx^0‖ ≤ ε_0$ and
%    in particular $‖dx^0‖ ≤ ‖dx‖ + ε_0$,
%  \item $‖x^j‖ ≤ ε_j$ if $j > 0$, for some positive $ε_j$ to be chosen later. This implies $‖dx^j‖ ≤ ε_j$
%    for all $j > 0$.
% \end{itemize}
%
% Using Lemma~\ref{is_weak_bdd_exact_controlled_y}, we get a sequence $y^j$ in $M^{i-1}_c$ such that
%   \[
%     ‖x^j_{|c} - dy^j‖ ≤  K ‖dx^j‖ + δ_j
%     \quad \text{and} \quad
%     ‖y^j‖ ≤ K(K + 1)‖x^j‖ + τ_j.
%   \]
% for positive sequences $δ$ and $τ$ to be chosen later.
%
% Since $M^{i-1}_c$ is complete, the series $\sum y^j$ converges as soon as we can guarantee that
% $\sum ‖y^j‖$ converges. Our estimates ensure this convergence as soon as the
% sum of the $K(K + 1)ε_j + τ_j$ converges so here we only need $ε$ and $τ$ to be
% summable.
%
% We then set $y = ∑ y^j$ and compute:
% \begin{align*}
%   ‖x_{|c} - dy‖ &= \left\|∑_{j ≥ 0} x^j_{|c} - dy^j\right\| \\
%     &≤ ∑_{j ≥ 0} \left\|x^j_{|c} - dy^j\right\| \\
%     &≤ ∑_{j ≥ 0} K‖dx^j‖ + δ_j \\
%     &≤ K‖dx‖ + Kε_0 + δ_0 +  ∑_{j > 0} (Kε_j + δ_j)
% \end{align*}
%
% So everything is fine as long as $∑_{j ≥ 0} (Kε_j + δ_j) ≤ ε$, say $ε_j = ε2^{-j-2}/K$ and $δ_j = ε2^{-j-2}$.
% \end{proof}

\begin{lemma}
  \label{weakexact_implies_exact}
  \lean{system_of_complexes.is_weak_bounded_exact.strong_of_complete}
  \leanok
  \uses{is_weak_bounded_exact, is_bounded_exact, admissible}
  Let $M^\bullet_\bullet$ be an admissible collection
  of complexes of complete normed abelian groups.

  Assume that $M^\bullet_c$ is weakly $\leq k$-exact in degrees $\leq m$ for $c\geq c_0$ with bound $K$.
  Then $M^\bullet_c$, for every $δ > 0$, it is $\leq k^2$-exact in degrees $\leq m$ for $c\geq c_0$
  with bound $K+δ$.
\end{lemma}

\begin{proof}
  \uses{is_bounded_exact_of_weakly}
  \leanok
  Lemma~\ref{is_bounded_exact_of_weakly} ensures we only need to care about cocycles
  of $M$. More precisely, let $x$ be a cocycle in $M^i_{k^2c}$ for some $i ≤ m$ and $c ≥ c_0$.
  We need to find $y \in M^{i-1}_c$ such that $dy = x_{|c}$.

  By weak $\leq k$-exactness applied to $x$ and a sequence $ε_j$ to be chosen later, we can find
  a sequence $w^j \in M^{i-1}_{kc}$ such that
  \[
    ‖x_{kc} - dw^j‖ ≤ ε_j.
  \]
  Then, by weak $\leq k$-exactness applied to each $w^{j + 1} - w^j$ and a sequence $δ_j$ to be chosen later, we can find
  a sequence $z^j \in M^{i-2}_{c}$ such that
  \[
    ‖(w^{j+1} - w^j)_{|c} - dz^j‖ ≤ K‖dw^{j+1} - dw^j‖ + δ_j.
  \]
  We set $y^j := w^j_{|c} - \sum_{l=0}^{j-1} dz^l ∈ M^{i-1}_c$.


  We have
  \begin{align*}
    ‖y^{j + 1} - y^j‖ &=  \left\|(w^{j + 1} - w^j)_{|c} - dz^j\right\| \\
                      &≤  K‖dw^{j+1} - dw^j‖ + δ_j \\
                      &≤  2Kε_j + δ_j.
  \end{align*}
  So $y^j$ is a Cauchy sequence as long as we make sure $2Kε_j + δ_j ≤ 2^{-j}$ for instance.
  Since $M^{i-1}_c$ is complete, this sequence converges to some $y$.
  Because $dy^j = dw^j_{|c}$, we get that $‖x_{|c} - dy^j‖ ≤ ε_j$ and in the limit $x_{|c} = dy$.
\end{proof}

\begin{proposition}
  \label{weaksnakelemma}
  \leanok
  \lean{weak_normed_snake}
  \uses{is_weak_bounded_exact, admissible}
  Let $M^\bullet_\bullet$ and $M'^\bullet_\bullet$ be two admissible collections
  of complexes of complete normed abelian groups.
  For each $c\geq c_0$ let $f^\bullet_c: M^\bullet_c\to M'^\bullet_c$ be a collection of maps
  between these collections of complexes
  that are norm-nonincreasing and which all commute with all restriction maps,
  and assume that there exists these maps satisfy
  \[
    ‖x_{|c}‖ ≤ K''‖f(x)‖
  \]
  for all $i ≤ m+1$ and all $x\in M^i_{k''c}$.
  Let $N^\bullet_c=M'^\bullet_c/M^\bullet_c$
  be the collection of quotient complexes, with the quotient norm;
  this is again an admissible collection of complexes.

  Assume that $M^\bullet_c$ (resp. $M'^\bullet_c$) is weakly $\leq k$-exact
  (resp. $≤ k'$-exact) in degrees $\leq m$ for $c\geq c_0$ with bound $K$
  (resp. $K'$).
  Then $N^\bullet_c$ is weakly $\leq kk'k''$-exact in degrees $\leq m-1$ for $c\geq c_0$
  with bound $K'(KK'' + 1)$.
\end{proposition}

\begin{proof}
\leanok
\def\ndn{\left\|dn\right\|}
Let $n \in N^i_{kk'k''c}$ for $i\leq m-1$.
We fix $ε > 0$. We need to find an element $y \in N^{i-1}_c$ such that
\[
  ‖n_{|c} - dy‖ \leq K'(KK'' + 1)‖dn‖ + \epsilon.
\]

Pick any preimage $m' \in M'^i_{kk'k''c}$ of $n$. In particular $dm'$ is
a preimage of $dn$.
By definition of the quotient norm,
we can find $m_1 ∈ M^{i+1}_{kk'k''c}$ and $m_1'' ∈ (M')^{i+1}_{kk'k''c}$ such that
\[
dm' = f(m_1) + m_1''
\]
with $‖m_1''‖ \leq ‖dn‖ + ε_1$, for some positive $ε_1$ to be chosen later.

Applying the differential to the last displayed equation, and using that this
kills the image of $d$, and that $f$ is a map of complexes, we see that
\[
f(dm_1) = -dm_1''.
\]
Using the norm bound on $f$, we get
\[\begin{aligned}
  ‖dm_{1|kk'c}‖ &≤ K''‖f(dm_1)‖ = K''‖dm_1''‖\\
                &≤ K''‖m_1''‖ ≤ K''‖dn‖ + K''ε_1.
\end{aligned}\]
On the other hand, weak exactness of $M$ applied to $m_{1|kk'c}$
gives $m_0 ∈ M^i_{k'c}$ such that
\[
  ‖m_{1|kk'c|k'c} - dm_0‖ \leq K‖dm_{1|kk'c}‖ + ε_1
\]
which combines with the previous estimate to give:
\[
  ‖m_{1|k'c} - dm_0‖ \leq K K'' \left\|d n\right\| + (KK'' + 1)ε_1.
\]
Now let $m'_{\mathrm{new}} = m'_{|k'c} - f(m_0) \in M'^i_{k'c}$; this is a lift of $n_{|k'c}$.
Then
\[
dm'_{\mathrm{new}} = dm'_{|k'c} - f(m_{1|k'c}) + f(m_{1|k'c} - dm_0) = m''_{1|k'c} + f(m_{1|k'c} - dm_0).
\]
In particular,
\[
‖dm'_{\mathrm{new}}‖ ≤ (KK'' + 1)\ndn + (KK'' + 2) ε_1.
\]
Now weak exactness of $M'$ gives $x \in M'^{i-1}_c$ such that
\[
  ‖m'_{\mathrm{new}|c} - dx‖ ≤ K'‖dm'_{\mathrm{new}}‖ + ε_1 \leq
    K'((K K'' + 1) \ndn + (KK'' + 2) ε_1) + ε_1.
\]
In particular, letting $y \in N^{i-1}_c$ be the image of $x$, we get
\[
  ‖n_{|c} - dy‖ ≤ K'(K K'' + 1)\ndn + (K'(K  K'' + 2) + 1) ε_1,
\]
which is exactly what we wanted if we choose
$ε_1 = ε/(K'(K  K'' + 2) + 1)$.
\end{proof}

% \begin{proposition}
%   \label{snakelemma}
%   \uses{is_bounded_exact, admissible}
%   Let $M^\bullet_\bullet$ and $M'^\bullet_\bullet$ be two admissible collections
%   of complexes of complete normed abelian groups.
%   For $c\geq c_0$ let $f^\bullet_c: M^\bullet_c\to M'^\bullet_c$ be a collection of maps
%   between these collections of complexes
%   that is strictly compatible with the norm and commutes with restriction maps,
%   and assume that it satisfies
%   \[
%     ‖x_{|c}‖ ≤ K''‖f(x)‖
%   \]
%   for all $i ≤ m+1$ and all $x\in M^i_{k''c}$.
%   Let $N^\bullet_c=M'^\bullet_c/\overline{M^\bullet_c}$
%   be the collection of quotient complexes, with the quotient norm;
%   this is again an admissible collection of complexes.
%
%   Assume that $M^\bullet_c$ (resp. $M'^\bullet_c$) is $\leq k$-exact
%   (resp. $≤ k'$-exact) in degrees $\leq m$ for $c\geq c_0$ with bound $K$
%   (resp. $K'$).
%   Then, for every $δ > 0$, $N^\bullet_c$ is $\leq (kk'k'')^2$-exact in
%   degrees $\leq m-1$ for $c\geq c_0$
%   with bound $K'(KK'' + 1) + δ$.
% \end{proposition}
%
% \begin{proof}
%   \uses{weaksnakelemma, weakexact_implies_exact}
%   The exactness assumptions on $M$ and $M'$ give the corresponding
%   weak exactness condition. Hence Proposition~\ref{weaksnakelemma}
%   ensures that $N^\bullet_c$ is weakly $\leq kk'k''$-exact in degrees
%   $\leq m-1$ for $c\geq c_0$ with bound $K'(KK'' + 1)$.
%   Since $N^\bullet_c$ is a complex of complete groups,
%   Lemma~\ref{weakexact_implies_exact} gives the required exactness.
% \end{proof}

We also need the `dual' version of \ref{weaksnakelemma}, for kernels instead of cokernels.
Note that this is actually a bit easier to prove.
(Should we put it first in the presentation?)

\begin{proposition}
  \label{weakdualsnakelemma}
  \leanok
  \lean{weak_normed_snake_dual}
  \uses{is_weak_bounded_exact, admissible}
  Let $M^\bullet_\bullet$ and $M'^\bullet_\bullet$ be two admissible collections
  of complexes of complete normed abelian groups.
  For each $c\geq c_0$ let $f^\bullet_c: M^\bullet_c\to M'^\bullet_c$ be a collection of maps
  between these collections of complexes which all commute with all restriction maps.
  Assume moreover that we are given two constants $r_1, r_2 \geq 0$ such that:
  \begin{itemize}
   \item for all $i, c\geq c_0$ and all $x\in M^i_c$
    \[
    ‖f(x)‖ ≤ r_1‖x‖;
    \]
   \item for all $i ≤ m+1, c \geq c_0$ and all $y\in M'^i_c$, there exists $x\in M^i_c$ such that
   \[
    f(x) = y \mbox{ and } ‖x‖ ≤ r_2‖y‖.
   \]

  \end{itemize}

  Let $N^\bullet_c$ be the collection of kernel complexes, with the induced norm;
  this is again an admissible collection of complexes.

  Assume that $M^\bullet_c$ (resp. $M'^\bullet_c$) is weakly $\leq k$-exact
  (resp. $≤ k'$-exact) in degrees $\leq m$ for $c\geq c_0$ with bound $K$
  (resp. $K'$).
  Then $N^\bullet_c$ is weakly $\leq kk'$-exact in degrees $\leq m-1$ for $c\geq c_0$
  with bound $K + r_1r_2KK'$.
\end{proposition}

\begin{proof}
\leanok
%\def\ndn{\left\|dn\right\|}
Let $n \in N^i_{kk'c} \subseteq M^i_{kk'c}$ for $i\leq m-1$ and let $ε > 0$. We need to find an element $y \in N^{i-1}_c$ such that
\[
  ‖n_{|c} - dy‖ \leq K + r_1r_2KK'‖dn‖ + \epsilon.
\]
By weak exactness of $M^\bullet_\bullet$, we can find $m \in M^{i-i}_{k'c}$ such that
\[
  ‖n_{|k'c} - dm‖ \leq K‖dn‖ + \epsilon_1,
\]
where $\epsilon_1 > 0$ to be chosen later. By weak exactness of $M'^\bullet_\bullet$, we can find $m' \in M'^{i-2}_c$ such that
\[
  ‖f(m)_{|c} - dm'‖ \leq K'‖df(m)‖ + \epsilon_2,
\]
where $\epsilon_2 > 0$ to be chosen later. Let $m_1 \in M^{i-2}_c$ be a lift of $m'$ and let $m_2 \in M^{i-1}_c$ be such that
\[
f(m_2) = f(m_{|c} - dm_1) \mbox{ and } ‖m_2‖ \leq r_2 ‖f(m_{|c} - dm_1)‖.
\]
Set $y = m_{|c} - dm_1 - m_2 \in M^{i-1}_c$. By construction $f(y) = 0$, so $y \in N^{i-1}_c$. We compute
\begin{gather*}
‖n_{|c} - dy‖ = ‖n_{|c} - dm_{|c} + d^2m_1 - dm_2‖ = ‖n_{|c} - dm_{|c} - dm_2‖ \leq \\
‖n_{|c} - dm_{|c} ‖ + ‖dm_2‖ = ‖(n_{|k'c} - dm)_{|c}‖ + ‖dm_2‖ \leq ‖(n_{|k'c} - dm)‖ + ‖dm_2‖ \leq \\
K‖dn‖ + \epsilon_1 + ‖dm_2‖.
\end{gather*}
Where we have used the defining property of $m$ and admissibility of $M^\bullet_\bullet$. By the same assumption and since $f(n_{|k'c}) = f(n)_{|k'c} = 0$, we have
\begin{gather*}
‖dm_2‖ \leq ‖m_2‖ \leq  r_2 ‖f(m_{|c} - dm_1)‖ = r_2 ‖f(m)_{|c} - df(m_1)‖ = r_2 ‖f(m)_{|c} - dm'‖ \leq \\
r_2(K'‖df(m)‖ + \epsilon_2) = r_2(K'‖f(dm)‖ + \epsilon_2) = r_2(K'‖f(n_{|k'c}) - f(dm)‖ + \epsilon_2) = \\
r_2(K'‖f(n_{|k'c} - dm)‖ + \epsilon_2) \leq r_2(K'r_1‖n_{|k'c} - dm‖ + \epsilon_2) \leq r_2(K'r_1(K‖dn‖ + \epsilon_1) + \epsilon_2)
\end{gather*}
In particular we get
\begin{gather*}
‖n_{|c} - dy‖ \leq K‖dn‖ + \epsilon_1 + r_2(K'r_1(K‖dn‖ + \epsilon_1) + \epsilon_2) =
\\(K + r_1r_2KK')‖dn‖ + \epsilon_1(1+r_1r_2K') + r_2\epsilon_2.
\end{gather*}
Now let
\[
\epsilon_1 = \frac{\epsilon}{2(1+r_1r_2K')} \mbox{ and } \epsilon_2 = \begin{cases}
                                                                    \frac{\epsilon}{2r_2} \mbox{ if } r_2 \neq 0 \\
                                                                    1 \mbox{ if } r_2 = 0
                                                                      \end{cases}
\]
In any case $r_2 \epsilon_2 \leq \frac{\epsilon}{2}$ and so
\[
‖n_{|c} - dy‖ \leq (K + r_1r_2KK')‖dn‖ + \epsilon
\]
as required.

If $i=0$, then all $m$, $m'$, $m_1$ and $m_2$ are $0$, so $y=0$ as required.
\end{proof}

Consider a system of double complexes $M^{p,q}_c$, $p,q\geq 0$, $c\geq c_0$,
\begin{center}
  \begin{tikzcd}
    M^{0,0}_c \ar[r]{d'^{0,0}_c}\ar[d]{d^{0,0}_c} & M^{0,1}_c\ar[r]{d'^{0,1}_c}\ar[d]{d^{0,1}_c} & M^{0,2}_c\ar[r]{d'^{0,2}_c}\ar[d]{d^{0,2}_c} & \ldots\\
M^{1,0}_c\ar[r]{d'^{1,0}_c}\ar[d]{d^{1,0}_c} & M^{1,1}_c\ar[r]{d'^{1,1}_c}\ar[d]{d^{1,1}_c} & M^{1,2}_c\ar[r]{d'^{1,2}_c}\ar[d]{d^{1,2}_c} & \ldots\\
M^{2,0}_c\ar[r]{d'^{2,0}_c}\ar[d]{d^{2,0}_c} & M^{2,1}_c\ar[r]{d'^{2,1}_c}\ar[d]{d^{2,1}_c} & \ddots\\
\vdots & \vdots
  \end{tikzcd}
\end{center}
of complete normed abelian groups.

\begin{definition}
  \label{spectral-htpy}
  \lean{system_of_double_complexes.normed_spectral_homotopy}
  \uses{system_of_complexes}
  \leanok
  We say that the system of double complexes $M^{p,q}_c$
  satisfies the \emph{normed spectral homotopy condition}
  for $m \in \N$ and $H, c_0 \in \R_{\ge 0}$
  if the following condition is satisfied:

  For $q=0,\ldots,m$ and $c\geq c_0$,
  there is a map $h^q_{k'c} \colon M^{0,q+1}_{k'c}\to M^{1,q}_c$ with
  \[
    \|h^q_{k'c}(x)\|_{M^{1,q}_c} \leq H\|x\|_{M^{0,q+1}_{k'c}}
  \]
  for all $x\in M^{0,q+1}_{k'c}$,
  and such that for all $c\geq c_0$ and $q=0,\ldots,m$ the ``homotopic'' map
  \[
    \mathrm{res}_{k'^2c,k'c}^{1,q}\circ d^{0,q} +
    h^q_{k'^2c}\circ d'^{0,q}_{k'^2c} +
    d'^{1,q-1}_{k'c}\circ h^{q-1}_{k'^2c} \colon
      M^{0,q}_{k'^2c}\to M^{1,q}_{k'c}
  \]
  factors as a composite of the restriction $\mathrm{res}_{k'^2c,c}^{0,q}$ and a map
  \[
    \delta^{0,q}_c \colon M^{0,q}_c\to M^{1,q}_{k'c}
  \]
  that is a map of complexes (in degrees $\leq m$), and satisfies the estimate
  \begin{equation}\label{eq:homotopicmapsmall}
  \|\delta^{0,q}_c(x)\|_{M^{1,q}_{k'c}}\leq \epsilon \|x\|_{M^{0,q}_c}
  \end{equation}
  for all $x\in M^{0,q}_c$.
\end{definition}

\begin{proposition}
  \label{spectral}
  \uses{admissible, is_weak_bounded_exact, spectral-htpy}
  \lean{system_of_double_complexes.normed_spectral}
  \leanok
  Fix an integer $m\geq 0$ and constants $k$, $K$.
  Then there exists an $\epsilon>0$ and constants $k_0$, $K_0$,
  depending (only) on $k$, $K$ and $m$, with the following property.

  Let $M^{p,q}_c$ be a system of double complexes as above,
  and assume that it is admissible.
  Assume further that there is some $k'\geq k_0$ and some $H>0$, such that
  \begin{enumerate}
	  \item for $i=0,\ldots,m+1$, the rows $M^{i,q}_c$ are weakly $\leq k$-exact in degrees $\leq m-1$ for $c\geq c_0$ with bound $K$;
	  \item for $j=0,\ldots,m$, the columns $M^{p,j}_c$ are weakly $\leq k$-exact in degrees $\leq m$ for $c\geq c_0$ with bound $K$;
    \item it satisfies the normed spectral homotopy condition for $m$, $H$ and $c_0$.
  \end{enumerate}
  Then the first row is weakly $\leq k'^2$ exact in degrees $\leq m$ for $c\geq c_0$ with bound $2K_0H$.
\end{proposition}

We note that the bound on the homotopy is of a peculiar nature, in that the bound only depends on a deep restriction of $x$.

\begin{proof}[{Proof of Proposition~\ref{spectral}}]
  \proves{spectral}
  \uses{weaksnakelemma}
  \leanok
  First, we treat the case $m=0$.
  If $m=0$, we claim that one can take $\epsilon=\tfrac 1{2k}$ and $k_0=k$.
  We have to prove exactness at the first step.
  Let $x_{k'^2c}\in M^{0,0}_{k'^2c}$ and
  denote $x_{k'c}=\mathrm{res}_{k'^2c,k'c}^{0,0}(x)$
  and $x_c=\mathrm{res}_{k'^2c,c}^{0,0}(x)$.
  Then by assumption (2) (and $k'\geq k$), we have
  \[
  ‖x_c‖_{M^{0,0}_c}\leq k‖d^{0,0}_{k'c}(x_{k'c})‖_{M^{1,0}_{k'c}}.
  \]
  On the other hand, by (3),
  \[
  ‖\mathrm{res}_{k'^2c,k'c}^{1,0}(d^{0,0}_{k'^2c}(x))\pm h^0_{k'^2c}(d'^{0,0}_{k'^2c}(x))‖_{M^{1,0}_{k'c}}\leq \epsilon ‖x_c‖_{M^{0,0}_c}.
  \]
  In particular, noting that $\mathrm{res}_{k'^2c,k'c}^{1,0}(d^{0,0}_{k'^2c}(x)) = d^{0,0}_{k'c}(x_{k'c})$, we get
  \[
  ‖x_c‖_{M^{0,0}_c}\leq k‖d^{0,0}_{k'c}(x_{k'c})‖_{M^{1,0}_{k'c}}\leq k\epsilon ‖x_c‖_{M^{0,0}_c} + kH ‖d'^{0,0}_{k'^2c}(x)‖_{M^{0,1}_{k'^2c}}.
  \]
  Thus, taking $\epsilon=\tfrac 1{2k}$ as promised, this implies
  \[
  ‖x_c‖_{M^{0,0}_c}\leq 2kH ‖d'^{0,0}_{k'^2c}(x)‖_{M^{0,1}_{k'^2c}}.
  \]
  This gives the desired $\leq \max(k'^2,2k_0H)$-exactness in degrees $\leq m$ for $c\geq c_0$.

  Now we argue by induction on $m$.
  Consider the complex $N^{p,q}$ given by $M^{p,q+1}$ for $q\geq 1$
  and $N^{p,0} = M^{p,1}/\overline{M^{p,0}}$
  (the quotient by the closure of the image, which is also the completion of $M^{p,1}/M^{p,0}$),
  equipped with the quotient norm.
  Using the normed version of the snake lemma,
  Proposition~\ref{weaksnakelemma},
  one checks that this satisfies the assumptions for $m-1$,
  with $k$ replaced by $\max(k^4,k^3+k+1)$.
\end{proof}

% vim: ts=2 et sw=2 sts=2
